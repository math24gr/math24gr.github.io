
\documentclass[a4paper,10pt]{report}

%\usepackage[landscape]{geometry}
\usepackage[cm-default]{fontspec}
\setromanfont{FreeSerif}
\setsansfont{FreeSans}
\setmonofont{FreeMono}
\usepackage[utf8x]{inputenc}
\usepackage{fontspec}
\usepackage{xunicode}
\usepackage{xltxtra}
\usepackage{xgreek}
\usepackage{amsmath}
\usepackage{unicode-math}
\usepackage{ulem}
\usepackage{color}
\usepackage{verbatim}
\usepackage{nopageno}
\usepackage{graphicx}
\usepackage{textpos}
\setlength{\TPHorizModule}{1cm}
\setlength{\TPVertModule}{1cm}
%\usepackage[colorgrid,texcoord]{eso-pic}
\usepackage[outline]{contour}
\usepackage{wrapfig}
\usepackage{url}
\usepackage{color}
\usepackage{tikz}
\usepackage{fancybox,fancyhdr}
\usepackage{subfigure}
\usepackage{pstricks}
\usepackage{epsfig}
\usepackage{multicol}
\usepackage{listings}
\usepackage{enumerate}
\usepackage{hyperref}
\hypersetup{
  bookmarks=true,
  bookmarksopen=true,
  pdfborder=false,
  pdfpagemode=UseNone,
  raiselinks=true,
  pdfhighlight={/P},
  colorlinks,
  citecolor=black,
  filecolor=black,
  linkcolor=black,
  urlcolor=black
}


\usepackage{fontspec}
\usepackage{xunicode}
\usepackage{xltxtra}


% Margins
%\setlength{\textwidth}{24cm}

%\setlength{\voffset}{0in}
%\setlength{\textheight}{6.8in}
% Colors
\definecolor{cbrown}{rgb}{0.49,0.24,0.07}
\definecolor{cmpez}{rgb}{0.92,0.88,0.79}
\definecolor{cmpez2}{rgb}{0.62,0.33,0.34}
\definecolor{cwhite}{rgb}{1,1,1}
\definecolor{cblack}{rgb}{0,0,0}
\definecolor{cred}{rgb}{0.9,0.15,0.15}
\definecolor{clblue}{rgb}{0.57,0.85,0.97}
\definecolor{corange}{rgb}{0.86,0.49,0.18}
\definecolor{clorange}{rgb}{0.95,0.84,0.61}
\definecolor{cyellow}{rgb}{1,0.95,0.16}
\definecolor{cgreen}{rgb}{0.69,0.8,0.31}
\definecolor{clbrown}{rgb}{0.78,0.67,0.43}
\definecolor{cmagenta}{rgb}{0.79,0.05,0.54}
\definecolor{cgray}{rgb}{0.76,0.73,0.63}
\definecolor{bluesite}{rgb}{0.05,0.43,0.67}
\definecolor{redsite}{rgb}{0.90,0.15,0.15}
\definecolor{backsite}{rgb}{0.07,0.07,0.07}
% ----------- from Costas -----------------------------

%---------------   Main Fonts    -----------------%   

% \setmainfont[Mapping=tex-text]{Calibri}
% \setmainfont[Mapping=tex-text]{Times New Roman}
%\setmainfont[Mapping=tex-text]{Droid Serif}
%\setmainfont[Mapping=tex-text]{Cambria}
%\setmainfont[Mapping=tex-text]{cm-unicode}
% \setmainfont[Mapping=tex-text]{Gentium}
% \setmainfont[Mapping=tex-text]{GFS Didot}
% \setmainfont[Mapping=tex-text]{Comic Sans MS}
 \setmainfont[Mapping=tex-text]{Ubuntu}
% \setmainfont[Mapping=tex-text]{Myriad Pro}
% \setmainfont[Mapping=tex-text]{CMU Concrete}
%\setmainfont[Mapping=tex-text]{DejaVu Sans}
%\setmainfont[Mapping=tex-text]{KerkisSans}
%\setmainfont[Mapping=tex-text]{KerkisCaligraphic}
% \setmainfont[Mapping=tex-text]{Segoe Print}
% \setmainfont[Mapping=tex-text]{Gabriola}

%------------------------------------------------%

% ------------ Mathematics Fonts ----------------%
%\setmathfont{Asana-Math.ttf}
% -----------------------------------------------%
% Misc
\author{Κασωτάκη Ε. - Σμαραγδάκης Κ.}
\title{www.math24.gr}
% ---------------- Formation --------------------%


\setlength\topmargin{-0.7cm}
\addtolength\topmargin{-\headheight}
\addtolength\topmargin{-\headsep}
\setlength\textheight{26cm}
\setlength\oddsidemargin{-0.54cm}
\setlength\evensidemargin{-0.54cm}
%\setlength\marginparwidth{1.5in}
\setlength\textwidth{17cm}

\RequirePackage[avantgarde]{quotchap}
\renewcommand\chapterheadstartvskip{\vspace*{0\baselineskip}}
\RequirePackage[calcwidth]{titlesec}
\titleformat{\section}[hang]{\bfseries}
{\Large\thesection}{12pt}{\Large}[{\titlerule[0.9pt]}]
%--------------------------------------------------%
%\usepackage{draftwatermark}
%\SetWatermarkText{www.math24.gr}
%\SetWatermarkLightness{0.9}
%\SetWatermarkScale{0.6}
%--------------------------------------------------%


\usepackage{xcolor}
\usepackage{amsthm}
\usepackage{framed}
\usepackage{parskip}

\colorlet{shadecolor}{bluesite!20}

\newtheorem*{orismos}{Ορισμός}

\newenvironment{notation}
  {\begin{shaded}\begin{theorem}}
  {\end{theorem}\end{shaded}}




% Document begins
\begin{document}
\pagestyle{fancy}
\fancyhead{}
\fancyfoot{}
\renewcommand{\headrulewidth}{0pt}
\renewcommand{\footrulewidth}{0pt}


\fancyhead[LO,LE]{
 %\textblockcolor{backsite}
 %\begin{textblock}{5}(-2,-0.55)
  %\rule{0cm}{1cm}
 %\end{textblock}
 \textblockcolor{bluesite}
 \begin{textblock}{5}(-1.5,-0.55)
  \rule{0cm}{1cm}
 \end{textblock}
 %\textblockcolor{bluesite}
 %\begin{textblock}{14}(5.5,-0.55)
 % \rule{0cm}{1cm}
 %\end{textblock}
 \begin{textblock}{0}(-1,-0.25)
 \color{cwhite} \begin{Large}www.math24.gr\end{Large}
 \end{textblock}
\begin{textblock}{0}(16,-0.4)
 \color{cwhite} \includegraphics[height=1.5cm]{math24_logo.png}
 \end{textblock}
\textblockcolor{white}
\begin{textblock}{17}(-1,28)
 \color{backsite} \begin{small}Επιμέλεια: Κασωτάκη Ειρήνη (ikasotaki@gmail.com) - Σχολικό έτος 2018 -  2019  \end{small}
 \end{textblock}
}
 \begin{shaded}
\begin{center}
\huge \textbf{Άλγεβρα Α' Λυκείου}\\
\end{center} 
\begin{center}
\textbf{Επαναληπτικές Ασκήσεις στην ´Εννοια της Συνάρτησης } \hfill \textbf{}
\end{center}
%\subsection*{}
\end{shaded}
%\vspace{2em
%\begin{shaded}
%\begin{center}
%\huge \textbf{Επαναληπτικές Ασκήσεις για το Κεφάλαιο 1}\\
%Πρόσθεση ρητών αριθμών
%\end{center} 
%\textbf{Μαθηματικά Α' Γυμνασίου} \hfill \textbf{Ημερομηνία Παράδοσης : \hspace{2em} }
%\subsection*{Ονοματεπώνυμο :\hfill  \hspace{5em}}
%\end{shaded}
\vspace{2em}
\begin{itemize}
\item Πεδίο ορισμού συναρτήσεων
%Εξισώσεις α' βαθμού με απόλυτες τιμές
%Κλασματικές εξισώσεις που ανάγονται σε α' βαθμού
%Εξισώσεις με ταυτότητες που ανάγονται σε α' βαθμού
\end{itemize}

\section*{Άσκηση 1  \hfill \small{}}
Να βρείτε το πεδίο ορισμού των παρακάτω συναρτήσεων:
\begin{enumerate}[1)]
 \item $f(x)=\dfrac{3}{x+4}$
 \item $f(x)=\dfrac{4}{x+7}$
 \item $f(x)=\dfrac{4}{x-5}+1$
 \item $f(x)=\dfrac{3}{x-8}$
 \item $f(x)=\dfrac{3x}{2x+5}+5$
 \item $f(x)=\dfrac{2x+1}{3x+8}$   
 \item $f(x)=\dfrac{x+2}{x-5}+\dfrac{3x}{2x+4}$
 \item $f(x)=\dfrac{5x}{x+4}-\dfrac{2}{3x-12}$ 
 \item $f(x)=\dfrac{3x}{4}+1$
 \item $f(x)=\dfrac{x^{2}-3x}{2}$  
\end{enumerate}


\section*{Άσκηση 2  \hfill \small{}}
Να βρείτε το πεδίο ορισμού των παρακάτω συναρτήσεων:
\begin{enumerate}[1)]
 \item $f(x)=\dfrac{x^{2}-25}{x^{2}-5x}$
 \item $f(x)=\dfrac{x^{2}-9}{x^{2}-3x}$
 \item $f(x)=\dfrac{3}{x^{2}+4x+4}$
 \item $f(x)=\dfrac{x^{2}}{x^{2}-6x+9}$
 \item $f(x)=\dfrac{x}{x^{2}-5x+6}$
 \item $f(x)=\dfrac{3x+1}{x^{2}-3x-4}$
 \item $f(x)=\dfrac{3x}{x^{2}+2}$
 \item $f(x)=\dfrac{4x+1}{x^{2}+3}$
 \item $f(x)=\dfrac{2x}{x^{2}-x+5}$
 \item $f(x)=\dfrac{x^{2}+1}{x^{2}+2+4}$
\end{enumerate}

\section*{Άσκηση 3  \hfill \small{}}
Να βρείτε το πεδίο ορισμού των παρακάτω συναρτήσεων:
\begin{enumerate}[1)]
 \item $f(x)=\sqrt{x-5}$
 \item $f(x)=\sqrt{x-7}$
 \item $f(x)=\sqrt{x-2}+\sqrt{x-5}$
 \item $f(x)=\sqrt{x}+\sqrt{x+3}$
 \item $f(x)=\sqrt{2x-8}$
 \item $f(x)=\sqrt{3x-10}$
 \item $f(x)=\sqrt{2x-4}-\sqrt{3x-9}$
 \item $f(x)=\sqrt{3x+6}+\sqrt{4x+20}$
\end{enumerate}


\section*{Άσκηση 4  \hfill \small{}}
Να βρείτε το πεδίο ορισμού των παρακάτω συναρτήσεων:
\begin{enumerate}[1)]
 \item $f(x)=\sqrt{x^{2}-16}$
 \item $f(x)=\sqrt{x^{2}-25}$ 
 \item $f(x)=\sqrt{x^{2}-4x+4}$ 
 \item $f(x)=\sqrt{x^{2}+10+25}$
 \item $f(x)=\sqrt{x^{2}+3x-4}$
 \item $f(x)=\sqrt{x^{2}-x-6}$
 \item $f(x)=\sqrt{x^{2}-x+6}$
 \item $f(x)=\sqrt{x^{2}+2x+5}$
\end{enumerate}



\section*{Άσκηση 5  \hfill \small{}}
Να βρείτε το πεδίο ορισμού των παρακάτω συναρτήσεων:
\begin{enumerate}[1)]
 \item $f(x)=\dfrac{1}{x-1}+\sqrt{x-3}$
 \item $f(x)=\dfrac{1}{x-2}+\sqrt{x-2}$
 \item $f(x)=\dfrac{3}{x-4}$
 \item $f(x)=\dfrac{2}{\sqrt{x}-4}$
 \item $f(x)=\dfrac{1}{x^{2}-4}+\sqrt{x^{2}-4x+4}$
\end{enumerate}

\section*{Άσκηση 6  \hfill \small{}}
Να βρείτε το πεδίο ορισμού των παρακάτω συναρτήσεων:
\begin{enumerate}[1)]
\item  $$ f(x)=\begin{cases} 
                \ x^{2}   &, αν\quad x< 1 \\
                  x+4     &, αν\quad x\ge 1
               \end{cases} $$
\item  $$ f(x)=\begin{cases} 
                \ 3x+1   , & αν\quad x\le 0 \\
                  x^{2}+2    , & αν\quad 0<x<5
               \end{cases} $$ 
\item  $$ f(x)=\begin{cases} 
                \ 2x+1       , & αν\quad <1x<2 \\
                  2x-1       , & αν\quad 2\le x<10
               \end{cases} $$                
\item  $$ f(x)=\begin{cases} 
                \ x^{3}      , & αν\quad \le x<3 \\
                  x^{2}+1    , & αν\quad 3<x<10
               \end{cases} $$  
\item  $$ f(x)=\begin{cases} 
                \ 3x+1      , & αν\quad x\le 0 \\
                  2x^{2}    , & αν\quad 0<x<5
               \end{cases} $$                              
\end{enumerate} 

\section*{Άσκηση 7  \hfill \small{}}
Δίνεται η συνάρτηση $f(x)=x^{3}-1$. Να υπολογίσετε τις τιμές $f(-1)$,$f(0)$ και $f(2)$.

\section*{Άσκηση 8  \hfill \small{}}
Δίνεται η συνάρτηση $f(x)=2x^{2}-4x+5$. Να υπολογίσετε τις τιμές $f(-3)$,$f(1)$ και $f(5)$.

\section*{Άσκηση 9  \hfill \small{}}
Δίνεται η συνάρτηση 
$$ f(x)=\begin{cases} 
                \ 2x^{2}-1   & αν\quad x< 0 \\
                  2x+1     & αν\quad x\ge 0
               \end{cases} $$
Να υπολογίσετε τις τιμές $f(-5)$,$f(0)$ και $f(3)$.

\section*{Άσκηση 10  \hfill \small{}}
Δίνεται η συνάρτηση 
$$ f(x)=\begin{cases} 
                \ 3x-5   & αν\quad x\le 1 \\
                  2x^{3}+1     & αν\quad 1<x\le 5
               \end{cases} $$
Να υπολογίσετε τις τιμές $f(-5)$,$f(1)$ και $f(2)$.

\section*{Άσκηση 11  \hfill \small{}}
Δίνεται η συνάρτηση 
$$ f(x)=\dfrac{x^{2}-4}{x-2} $$
\begin{enumerate}[1)]
\item Να βρούμε το πεδίο ορισμού της $f$.
\item Να υπολογίσουμε τις τιμές $f(-2)$, $f(0)$ και $f(1)$.
\end{enumerate}

\section*{Άσκηση 12  \hfill \small{}}
Δίνεται η συνάρτηση 
$$ f(x)=\begin{cases} 
                \ x^{2}+5x+1   & αν\quad x< 2 \\
                  \sqrt{x-2}     & αν\quad x\ge 2
               \end{cases} $$
\begin{enumerate}[1)]
\item Να βρείτε το πεδίο ορισμού της $f$.
\item Να υπολογίσετε τις τιμές $f(0)$, $f(2)$ και $f(6)$.
\end{enumerate}

\section*{Άσκηση 13  \hfill \small{}}
Δίνεται η συνάρτηση 
$$ f(x)=\dfrac{1}{x^{2}+1}+5 $$
Να βρείτε τις τιμές του $x$ για  τις οποίες ισχύει $f(x)=6$.

\section*{Άσκηση 14  \hfill \small{}}
Δίνεται η συνάρτηση 
$$ f(x)=\begin{cases} 
                \ 2x^{3}+1   & αν\quad x\le 1 \\
                  \ x^{2}-5x+4     & αν\quad x>1
               \end{cases} $$
Να βρείτε τις τιμές του $x$ για  τις οποίες ισχύει $f(x)=0$.


\section*{Άσκηση 15  \hfill \small{}}
Δίνεται η συνάρτηση 
$$ f(x)=\dfrac{x^{2}-8x+6}{2x^{2}-10x+8} $$
\begin{enumerate}[1)]
\item Να βρείτε το πεδίο ορισμού της $f$.
\item Να υπολογίσετε τις τιμές $f(-1)$, $f(0)$ και $f(3)$.
\item Να λύσετε την εξίσωση $f(x)=0$.
\end{enumerate}


\section*{Άσκηση 16  \hfill \small{}}
Δίνεται η συνάρτηση 
$$ f(x)=\begin{cases} 
                \ x^{2}-5x+4   & αν\quad x<0 \\
                  \ x^{2}-8x+16     & αν\quad x\ge0
               \end{cases} $$
\begin{enumerate}[1)]
\item Να βρείτε το πεδίο ορισμού της $f$.
\item Να υπολογίσετε τις τιμές $f(-1)$, $f(0)$ και $f(2)$.
\item Να λύσετε την εξίσωση $f(x)=0$.
\end{enumerate}


\section*{Άσκηση 17  \hfill \small{}}
Δίνεται η συνάρτηση 
$$ f(x)=\begin{cases} 
                \ x^{2}+3x-4   & αν\quad x<1 \\
                  \ 2x^{2}-3x-2     & αν\quad x\ge 1
               \end{cases} $$
\begin{enumerate}[1)]
\item Να βρείτε το πεδίο ορισμού της $f$.
\item Να υπολογίσετε τις τιμές $f(-2)$, $f(1)$ και $f(2)$.
\item Να λύσετε την εξίσωση $f(x)=0$.
\end{enumerate}

\end{document}
