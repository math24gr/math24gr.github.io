
\documentclass[a4paper,10pt]{report}

%\usepackage[landscape]{geometry}
\usepackage[cm-default]{fontspec}
\setromanfont{FreeSerif}
\setsansfont{FreeSans}
\setmonofont{FreeMono}
\usepackage[utf8x]{inputenc}
\usepackage{fontspec}
\usepackage{xunicode}
\usepackage{xltxtra}
\usepackage{xgreek}
\usepackage{amsmath}
\usepackage{unicode-math}
\usepackage{ulem}
\usepackage{color}
\usepackage{verbatim}
\usepackage{nopageno}
\usepackage{graphicx}
\usepackage{textpos}
\setlength{\TPHorizModule}{1cm}
\setlength{\TPVertModule}{1cm}
%\usepackage[colorgrid,texcoord]{eso-pic}
\usepackage[outline]{contour}
\usepackage{wrapfig}
\usepackage{url}
\usepackage{color}
\usepackage{tikz}
\usepackage{fancybox,fancyhdr}
\usepackage{subfigure}
\usepackage{pstricks}
\usepackage{epsfig}
\usepackage{multicol}
\usepackage{listings}
\usepackage{enumerate}
\usepackage{hyperref}
\hypersetup{
  bookmarks=true,
  bookmarksopen=true,
  pdfborder=false,
  pdfpagemode=UseNone,
  raiselinks=true,
  pdfhighlight={/P},
  colorlinks,
  citecolor=black,
  filecolor=black,
  linkcolor=black,
  urlcolor=black
}


\usepackage{fontspec}
\usepackage{xunicode}
\usepackage{xltxtra}


% Margins
%\setlength{\textwidth}{24cm}

%\setlength{\voffset}{0in}
%\setlength{\textheight}{6.8in}
% Colors
\definecolor{cbrown}{rgb}{0.49,0.24,0.07}
\definecolor{cmpez}{rgb}{0.92,0.88,0.79}
\definecolor{cmpez2}{rgb}{0.62,0.33,0.34}
\definecolor{cwhite}{rgb}{1,1,1}
\definecolor{cblack}{rgb}{0,0,0}
\definecolor{cred}{rgb}{0.9,0.15,0.15}
\definecolor{clblue}{rgb}{0.57,0.85,0.97}
\definecolor{corange}{rgb}{0.86,0.49,0.18}
\definecolor{clorange}{rgb}{0.95,0.84,0.61}
\definecolor{cyellow}{rgb}{1,0.95,0.16}
\definecolor{cgreen}{rgb}{0.69,0.8,0.31}
\definecolor{clbrown}{rgb}{0.78,0.67,0.43}
\definecolor{cmagenta}{rgb}{0.79,0.05,0.54}
\definecolor{cgray}{rgb}{0.76,0.73,0.63}
\definecolor{bluesite}{rgb}{0.05,0.43,0.67}
\definecolor{redsite}{rgb}{0.90,0.15,0.15}
\definecolor{backsite}{rgb}{0.07,0.07,0.07}
% ----------- from Costas -----------------------------

%---------------   Main Fonts    -----------------%   

% \setmainfont[Mapping=tex-text]{Calibri}
% \setmainfont[Mapping=tex-text]{Times New Roman}
%\setmainfont[Mapping=tex-text]{Droid Serif}
%\setmainfont[Mapping=tex-text]{Cambria}
%\setmainfont[Mapping=tex-text]{cm-unicode}
% \setmainfont[Mapping=tex-text]{Gentium}
% \setmainfont[Mapping=tex-text]{GFS Didot}
% \setmainfont[Mapping=tex-text]{Comic Sans MS}
 \setmainfont[Mapping=tex-text]{Ubuntu}
% \setmainfont[Mapping=tex-text]{Myriad Pro}
% \setmainfont[Mapping=tex-text]{CMU Concrete}
%\setmainfont[Mapping=tex-text]{DejaVu Sans}
%\setmainfont[Mapping=tex-text]{KerkisSans}
%\setmainfont[Mapping=tex-text]{KerkisCaligraphic}
% \setmainfont[Mapping=tex-text]{Segoe Print}
% \setmainfont[Mapping=tex-text]{Gabriola}

%------------------------------------------------%

% ------------ Mathematics Fonts ----------------%
%\setmathfont{Asana-Math.ttf}
% -----------------------------------------------%
% Misc
\author{Κασωτάκη Ε. - Σμαραγδάκης Κ.}
\title{www.math24.gr}
% ---------------- Formation --------------------%


\setlength\topmargin{-0.7cm}
\addtolength\topmargin{-\headheight}
\addtolength\topmargin{-\headsep}
\setlength\textheight{26cm}
\setlength\oddsidemargin{-0.54cm}
\setlength\evensidemargin{-0.54cm}
%\setlength\marginparwidth{1.5in}
\setlength\textwidth{17cm}

\RequirePackage[avantgarde]{quotchap}
\renewcommand\chapterheadstartvskip{\vspace*{0\baselineskip}}
\RequirePackage[calcwidth]{titlesec}
\titleformat{\section}[hang]{\bfseries}
{\Large\thesection}{12pt}{\Large}[{\titlerule[0.9pt]}]
%--------------------------------------------------%
%\usepackage{draftwatermark}
%\SetWatermarkText{www.math24.gr}
%\SetWatermarkLightness{0.9}
%\SetWatermarkScale{0.6}
%--------------------------------------------------%


\usepackage{xcolor}
\usepackage{amsthm}
\usepackage{framed}
\usepackage{parskip}

\colorlet{shadecolor}{bluesite!20}

\newtheorem*{orismos}{Ορισμός}

\newenvironment{notation}
  {\begin{shaded}\begin{theorem}}
  {\end{theorem}\end{shaded}}




% Document begins
\begin{document}
\pagestyle{fancy}
\fancyhead{}
\fancyfoot{}
\renewcommand{\headrulewidth}{0pt}
\renewcommand{\footrulewidth}{0pt}


\fancyhead[LO,LE]{
 %\textblockcolor{backsite}
 %\begin{textblock}{5}(-2,-0.55)
  %\rule{0cm}{1cm}
 %\end{textblock}
 \textblockcolor{bluesite}
 \begin{textblock}{5}(-1.5,-0.55)
  \rule{0cm}{1cm}
 \end{textblock}
 %\textblockcolor{bluesite}
 %\begin{textblock}{14}(5.5,-0.55)
 % \rule{0cm}{1cm}
 %\end{textblock}
 \begin{textblock}{0}(-1,-0.25)
 \color{cwhite} \begin{Large}www.math24.gr\end{Large}
 \end{textblock}
\begin{textblock}{0}(16,-0.4)
 \color{cwhite} \includegraphics[height=1.5cm]{math24_logo.png}
 \end{textblock}
\textblockcolor{white}
\begin{textblock}{17}(-1,28)
 \color{backsite} \begin{small}Επιμέλεια: Κασωτάκη Ειρήνη (ikasotaki@gmail.com) - Σχολικό έτος 2018 -  2019  \end{small}
 \end{textblock}
}
 \begin{shaded}
\begin{center}
\huge \textbf{Άλγεβρα Α' Λυκείου}\\
\end{center} 
\begin{center}
\textbf{Επαναληπτικές Ασκήσεις στις Ανισώσεις α' βαθμού } \hfill \textbf{}
\end{center}
%\subsection*{}
\end{shaded}
%\vspace{2em
%\begin{shaded}
%\begin{center}
%\huge \textbf{Επαναληπτικές Ασκήσεις για το Κεφάλαιο 1}\\
%Πρόσθεση ρητών αριθμών
%\end{center} 
%\textbf{Μαθηματικά Α' Γυμνασίου} \hfill \textbf{Ημερομηνία Παράδοσης : \hspace{2em} }
%\subsection*{Ονοματεπώνυμο :\hfill  \hspace{5em}}
%\end{shaded}
\vspace{2em}
\begin{itemize}
\item Ανισώσεις α' βαθμού
\item Ανισώσεις α' βαθμού με απόλυτες τιμές
\item Κλασματικές ανισώσεις α' βαθμού
\item Συναλήθευση ανισώσεων α' βαθμού
\end{itemize}

\section*{Άσκηση 1  \hfill \small{}}
Να λύσετε τις παρακάτω ανισώσεις:
\begin{enumerate}[1)]
 \item $|x+5| < 7$
 \item $|x-3| < 4$
 \item $|2x+5|< 10$
 \item $|3x-8|< 17$
 \item $|4x-3|< 13$
\end{enumerate}


\section*{Άσκηση 2  \hfill \small{}}
Να λύσετε τις παρακάτω ανισώσεις:
\begin{enumerate}[1)]
 \item $|x+5| > 4$
 \item $|x-4| > 8$
 \item $|2x-3|> 5$
 \item $|3x-5|> 8 $
 \item $|4x+1|> 5$
\end{enumerate}


\section*{Άσκηση 3  \hfill \small{}}
Να λύσετε τις παρακάτω ανισώσεις:
\begin{enumerate}[1)]
 \item $2(x+5)-3(x-1)>6(x-1)-2$
 \item $2(2x-3)-5>2(5-3x)-1$
 \item $4(x+3)-2x+2>2(x-5)$
 \item $2(3-5x)-3(x-3)<3x-4$
 \item $5(3-x)+3(x-2)<-2(x+5)$
\end{enumerate}
 


\section*{Άσκηση 4  \hfill \small{}}
Να λύσετε τις παρακάτω ανισώσεις:
\begin{enumerate}[1)]
\item $\dfrac{3x-1}{4}+\dfrac{x-5}{2}>\dfrac{3x-1}{2}$
\item $\dfrac{x+1}{5}-\dfrac{x-2}{2}<\dfrac{2x+1}{5}$
\item $\dfrac{x-5}{3}-\dfrac{2x-3}{2}>x-1$
\item $\dfrac{2x+1}{4}-\dfrac{x-2}{3}>2$
\item $\dfrac{x-4}{3}+2<\dfrac{3x+1}{6}$
\end{enumerate}

\section*{Άσκηση 5  \hfill \small{}}
Να βρείτε τις τις τιμές του $x$ για τις οποίες συναληθεύουν οι παρακάτω ανισώσεις: \\
$3x+4(x-2)>3$ και $\dfrac{x}{3}+3>\dfrac{x}{5}$

\section*{Άσκηση 6  \hfill \small{}}
Να βρείτε τις τις τιμές του $x$ για τις οποίες συναληθεύουν οι παρακάτω ανισώσεις: \\
$2x+2(2x+1)>8$ και $\dfrac{x-1}{3}+4<7$

\section*{Άσκηση 7  \hfill \small{}}
Να εξετάσετε εάν συναληθεύουν οι παρακάτω ανισώσεις: \\
$2x+1>4$ και $\dfrac{2x}{3}-2(2x+1)>1$

\section*{Άσκηση 8  \hfill \small{}}
Να εξετάσετε εάν συναληθεύουν οι παρακάτω ανισώσεις: \\
$3(3x-2)-4(x+5)>x+6$ και $\dfrac{2x-5}{2}+1>\dfrac{5x+1}{2}$

\end{document}
