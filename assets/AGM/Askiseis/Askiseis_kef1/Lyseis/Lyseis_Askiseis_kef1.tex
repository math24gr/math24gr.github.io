
\documentclass[a4paper,10pt]{report}

%\usepackage[landscape]{geometry}
\usepackage[cm-default]{fontspec}
\setromanfont{FreeSerif}
\setsansfont{FreeSans}
\setmonofont{FreeMono}
\usepackage[utf8x]{inputenc}
\usepackage{fontspec}
\usepackage{xunicode}
\usepackage{xltxtra}
%\usepackage{xgreek}
\usepackage{amsmath}
\usepackage{unicode-math}
\usepackage{ulem}
\usepackage{color}
\usepackage{verbatim}
\usepackage{nopageno}
\usepackage{graphicx}
\usepackage{textpos}
\setlength{\TPHorizModule}{1cm}
\setlength{\TPVertModule}{1cm}
%\usepackage[colorgrid,texcoord]{eso-pic}
\usepackage[outline]{contour}
\usepackage{wrapfig}
\usepackage{url}
\usepackage{color}
\usepackage{tikz}
\usepackage{fancybox,fancyhdr}
\usepackage{subfigure}
\usepackage{pstricks}
\usepackage{epsfig}
\usepackage{multicol}
\usepackage{listings}
\usepackage{enumerate}
\usepackage{hyperref}
\hypersetup{
  bookmarks=true,
  bookmarksopen=true,
  pdfborder=false,
  pdfpagemode=UseNone,
  raiselinks=true,
  pdfhighlight={/P},
  colorlinks,
  citecolor=black,
  filecolor=black,
  linkcolor=black,
  urlcolor=black
}


\usepackage{fontspec}
\usepackage{xunicode}
\usepackage{xltxtra}


% Margins
%\setlength{\textwidth}{24cm}

%\setlength{\voffset}{0in}
%\setlength{\textheight}{6.8in}
% Colors
\definecolor{cbrown}{rgb}{0.49,0.24,0.07}
\definecolor{cmpez}{rgb}{0.92,0.88,0.79}
\definecolor{cmpez2}{rgb}{0.62,0.33,0.34}
\definecolor{cwhite}{rgb}{1,1,1}
\definecolor{cblack}{rgb}{0,0,0}
\definecolor{cred}{rgb}{0.9,0.15,0.15}
\definecolor{clblue}{rgb}{0.57,0.85,0.97}
\definecolor{corange}{rgb}{0.86,0.49,0.18}
\definecolor{clorange}{rgb}{0.95,0.84,0.61}
\definecolor{cyellow}{rgb}{1,0.95,0.16}
\definecolor{cgreen}{rgb}{0.69,0.8,0.31}
\definecolor{clbrown}{rgb}{0.78,0.67,0.43}
\definecolor{cmagenta}{rgb}{0.79,0.05,0.54}
\definecolor{cgray}{rgb}{0.76,0.73,0.63}
\definecolor{bluesite}{rgb}{0.05,0.43,0.67}
\definecolor{redsite}{rgb}{0.90,0.15,0.15}
\definecolor{backsite}{rgb}{0.07,0.07,0.07}
% ----------- from Costas -----------------------------

%---------------   Main Fonts    -----------------%   

% \setmainfont[Mapping=tex-text]{Calibri}
% \setmainfont[Mapping=tex-text]{Times New Roman}
%\setmainfont[Mapping=tex-text]{Droid Serif}
%\setmainfont[Mapping=tex-text]{Cambria}
%\setmainfont[Mapping=tex-text]{cm-unicode}
% \setmainfont[Mapping=tex-text]{Gentium}
% \setmainfont[Mapping=tex-text]{GFS Didot}
% \setmainfont[Mapping=tex-text]{Comic Sans MS}
 \setmainfont[Mapping=tex-text]{Ubuntu}
% \setmainfont[Mapping=tex-text]{Myriad Pro}
% \setmainfont[Mapping=tex-text]{CMU Concrete}
%\setmainfont[Mapping=tex-text]{DejaVu Sans}
%\setmainfont[Mapping=tex-text]{KerkisSans}
%\setmainfont[Mapping=tex-text]{KerkisCaligraphic}
% \setmainfont[Mapping=tex-text]{Segoe Print}
% \setmainfont[Mapping=tex-text]{Gabriola}

%------------------------------------------------%

% ------------ Mathematics Fonts ----------------%
\setmathfont{Asana-Math.ttf}
% -----------------------------------------------%
% Misc
\author{Κασωτάκη Ε. - Σμαραγδάκης Κ.}
\title{www.math24.gr}
% ---------------- Formation --------------------%


\setlength\topmargin{-0.7cm}
\addtolength\topmargin{-\headheight}
\addtolength\topmargin{-\headsep}
\setlength\textheight{26cm}
\setlength\oddsidemargin{-0.54cm}
\setlength\evensidemargin{-0.54cm}
%\setlength\marginparwidth{1.5in}
\setlength\textwidth{17cm}

\RequirePackage[avantgarde]{quotchap}
\renewcommand\chapterheadstartvskip{\vspace*{0\baselineskip}}
\RequirePackage[calcwidth]{titlesec}
\titleformat{\section}[hang]{\bfseries}
{\Large\thesection}{12pt}{\Large}[{\titlerule[0.9pt]}]
%--------------------------------------------------%
\usepackage{draftwatermark}
\SetWatermarkText{www.math24.gr}
\SetWatermarkLightness{0.9}
\SetWatermarkScale{0.6}
%--------------------------------------------------%


\usepackage{xcolor}
\usepackage{amsthm}
\usepackage{framed}
\usepackage{parskip}

\colorlet{shadecolor}{bluesite!20}

\newtheorem*{orismos}{Ορισμός}

\newenvironment{notation}
  {\begin{shaded}\begin{theorem}}
  {\end{theorem}\end{shaded}}




% Document begins
\begin{document}
\pagestyle{fancy}
\fancyhead{}
\fancyfoot{}
\renewcommand{\headrulewidth}{0pt}
\renewcommand{\footrulewidth}{0pt}


\fancyhead[LO,LE]{
 %\textblockcolor{backsite}
 %\begin{textblock}{5}(-2,-0.55)
  %\rule{0cm}{1cm}
 %\end{textblock}
 \textblockcolor{bluesite}
 \begin{textblock}{5}(-1.5,-0.55)
  \rule{0cm}{1cm}
 \end{textblock}
 %\textblockcolor{bluesite}
 %\begin{textblock}{14}(5.5,-0.55)
 % \rule{0cm}{1cm}
 %\end{textblock}
 \begin{textblock}{0}(-1,-0.25)
 \color{cwhite} \begin{Large}www.math24.gr\end{Large}
 \end{textblock}
\begin{textblock}{0}(16,-0.4)
 \color{cwhite} \includegraphics[height=1.5cm]{math24_logo.png}
 \end{textblock}
\textblockcolor{white}
\begin{textblock}{17}(-1,28)
 \color{backsite} \begin{small}Copyright \textcopyright 2014, Κασωτάκη Ε.(ikasotaki@gmail.com) - Σμαραγδάκης Κ.(kesmarag@gmail.com)\end{small}
 \end{textblock}
}
 \begin{shaded}
\begin{center}
\huge \textbf{Μαθηματικά Α' Γυμνασίου}\\
\end{center} 
\begin{center}
\textbf{Επαναληπτικές Ασκήσεις για το Κεφάλαιο 1 "Οι φυσικοί αριθμοί"} \hfill \textbf{}
\end{center}
%\subsection*{}
\end{shaded}
%\vspace{2em
%\begin{shaded}
%\begin{center}
%\huge \textbf{Επαναληπτικές Ασκήσεις για το Κεφάλαιο 1}\\
%Πρόσθεση ρητών αριθμών
%\end{center} 
%\textbf{Μαθηματικά Α' Γυμνασίου} \hfill \textbf{Ημερομηνία Παράδοσης : \hspace{2em} }
%\subsection*{Ονοματεπώνυμο :\hfill  \hspace{5em}}
%\end{shaded}
\vspace{2em}
\begin{itemize}
\item Πρόσθεση, αφαίρεση και πολλαπλασιασμός φυσικών αριθμών
\item Δυνάμεις φυσικών αριθμών
\item Ευκλείδεια διαίρεση - Διαιρετότητα
\item Χαρακτήρες διαιρετότητας - Μ.Κ.Δ - Ε.Κ.Π - Ανάλυση αριθμού σε γινόμενο πρώτων παραγόντων
\end{itemize}


\section*{Άσκηση 1  \hfill \small{}}
Να συμπληρώσετε τα παρακάτω κενά:
\begin{enumerate}[1)]
 \item Η ιδιότητα $α+β=β+α$ λέγεται \color{bluesite} αντιμεταθετική \color{cblack}
 \item Το 1 όταν \color{bluesite} πολλαπλασιαστεί  \color{cblack} με έναν φυσικό αριθμό δεν τον μεταβάλλει.
 \item Για να πολλαπλασιάσουμε έναν φυσικό αριθμό επί 1000 γράφουμε στο τέλος του αριθμού \color{bluesite} τρία \color{cblack}
       μηδενικά.
 \item Οι δυνάμεις του 1, δηλαδή $1^{ν}$, είναι όλες ίσες με \color{bluesite} 1 (ένα) \color{cblack}
 \item Στους φυσικούς αριθμούς η τέλεια διαίρεση είναι πράξη αντίστροφη του \color{bluesite} πολλαπλασιασμού \color{cblack}
 \item Ο διαιρέτης μιας διαίρεσης δεν μπορεί να είναι \color{bluesite} μηδέν \color{cblack}
 \item Σε μια διαίρεση όταν ο διαιρετέος είναι 0 τότε το πηλίκο είναι (ισούται με) \color{bluesite} 0 (μηδέν) \color{cblack}
 \item Κάθε φυσικός αριθμός \color{bluesite} διαιρεί \color{cblack} τα πολλαπλάσιά του.
 \item Ένας αριθμός που έχει διαιρέτες μόνο τον εαυτό του και το 1 λέγεται \color{bluesite} πρώτος \color{cblack}
 \item Ένας φυσικός αριθμός \color{bluesite} διαιρείται \color{cblack} με το 5 αν λήγει σε 0 ή 5.
\end{enumerate}

\section*{Άσκηση 2  \hfill \small{}}
Να χαρακτηρίσετε καθεμία από τις παρακάτω προτάσεις με τη λέξη "\textbf{Σωστό}``, αν είναι σωστή ή  
"\textbf{Λάθος}`` , αν είναι λανθασμένη:
\begin{enumerate}[1)]
 \item Η διαφορά δύο περιττών αριθμών είναι πάντα περιττός αριθμός \color{bluesite} Λάθος \color{cblack}
 \item Η πράξη $50-(13-3)$ δίνει το ίδιο αποτέλεσμα με την πράξη $50-13-3$ \color{bluesite} Λάθος \color{cblack}
 \item Για να πολλαπλασιάσουμε ένα φυσικό αριθμό με το 10000 γράφουμε στο τέλος του αριθμού 4 μηδενικά \color{bluesite} Σωστό \color{cblack}
 \item Η πράξη $17\cdot(3+10)$ δίνει το ίδιο αποτέλεσμα με την πράξη $17\cdot3+17\cdot10$ \color{bluesite} Σωστό \color{cblack}
 \item Το $2^{4}$ ισούται με $8$ \color{bluesite} Λάθος \color{cblack}
 \item $2+2+2+2=4\cdot2$ \color{bluesite} Σωστό \color{cblack}
 \item $2^{3}=3^{2}$ \color{bluesite} Λάθος \color{cblack}
 \item $4\cdot10^{2}+3\cdot10^{1}+4$ είναι το ανάπτυγμα του αριθμού $434$ σε δυνάμεις του $10$ \color{bluesite} Σωστό  \color{cblack}
 \item Ο διαιρέτης μιας διαίρεση δεν μπορεί να είναι 0 \color{bluesite} Σωστό \color{cblack}
 \item Ο διαιρετέος μιας διαίρεση δεν μπορεί να είναι 0 \color{bluesite} Λάθος \color{cblack}
 \item Η σχέση $22=4\cdot5+2$ είναι μια ευκλείδεια διαίρεση \color{bluesite} Σωστό \color{cblack}
 \item Ο αριθμός $3127$ διαιρείται με το $5$ \color{bluesite} Σωστό  \color{cblack}
 \item Ο αριθμός $9α+3$ διαιρείται με το $3$ \color{bluesite} Σωστό \color{cblack}
 \item Ο αριθμός $20$ αναλύεται σε γινόμενο πρώτων παραγόντων ως $2^{2}\cdot 5$ \color{bluesite} Σωστό \color{cblack}
 \item Ο αριθμός $200$ αναλύεται σε γινόμενο πρώτων παραγόντων ως $2\cdot 10^{2}$ \color{bluesite} Λάθος \color{cblack}
 \item Το ΕΚΠ των 3 και 6 είναι το 18 \color{bluesite} Λάθος  \color{cblack}
 \item Ο ΜΚΔ των 12 και 24 είναι το 6 \color{bluesite} Λάθος \color{cblack}
 \item Το ΕΚΠ των $2^{2}\cdot 3^{4}$ και $2^{3}\cdot 3^{3}$ είναι $2^{3}\cdot 3^{4}$ \color{bluesite} Σωστό \color{cblack}
 \item O MKΔ των $2^{3}\cdot 3^{5}$ και $2^{2}\cdot 3^{7}$ είναι $2^{2}\cdot 3^{5}$ \color{bluesite} Σωστό  \color{cblack}
 \item Οι αριθμοί 270 και 135 έχουν μέγιστο κοινό διαιρέτη τον αριθμό 5 \color{bluesite} Λάθος \color{cblack}
\end{enumerate}


\section*{Άσκηση 3  \hfill \small{}}
Να αντιστοιχίσετε κάθε στοιχείο της αριστερής στήλης με ένα στοιχείο της δεξιάς στήλης
\begin{itemize}
\begin{multicols}{2}
 \item ΕΚΠ(2,6)
 \item ΕΚΠ(8,32)
 \item ΕΚΠ(7,14)
 \item ΜΚΔ(100,1000)
 \item ΜΚΔ(5,32)
 \item ΜΚΔ(9,24)
 \item 32
 \item 100
 \item 3
 \item 6
 \item 14
 \item 1
\end{multicols}
\end{itemize}
\underline{\textbf{Λύση}}\\
 ΕΚΠ(2,6)=6 \\
 ΕΚΠ(8,32)=32 \\
 ΕΚΠ(7,14)=14 \\
 ΜΚΔ(100,1000)=100 \\
 ΜΚΔ(5,32)=1\\
 ΜΚΔ(9,24)=3

\section*{Άσκηση 4  \hfill \small{}}
Να βρείτε το αποτέλεσμα για καθεμία από τις παρακάτω αριθμητικές παραστάσεις:
\begin{enumerate}[1)]
 \item $3\cdot5\cdot2^{2}+5^{2}\cdot3+10$
 \item $3\cdot(5\cdot2^{2}+5^{2}\cdot3)+10$
 \item $3\cdot(5\cdot2^{2}+5^{2}\cdot3+10)$
 \item $3\cdot5\cdot(2^{2}+5^{2}\cdot3)+10$
 \item $3\cdot5\cdot2^{2}+5^{2}\cdot(3+10)$
\end{enumerate}
\underline{\textbf{Λύση}}\\ 
\begin{enumerate}[1)]
 \item $3\cdot5\cdot2^{2}+5^{2}\cdot3+10 = 3\cdot 5 \cdot4 +25\cdot 3+10 = 60+75+10 =145$
 \item $3\cdot(5\cdot2^{2}+5^{2}\cdot3)+10 = 3(5\cdot 4 +25\cdot 3)+10 = 3(20+75)+10 =3\cdot 95 +10 =285+10=295$
 \item $3\cdot(5\cdot2^{2}+5^{2}\cdot3+10)=3(5\cdot 4+25\cdot3 +10)=3(20+75+10)=3\cdot 105=315$
 \item $3\cdot5\cdot(2^{2}+5^{2}\cdot3)+10 = 3\cdot 5(4+25\cdot3)+10=3\cdot5(4+75)+10=3\cdot5\cdot79+10=15\cdot79+10=1185+10=1195$
 \item $3\cdot5\cdot2^{2}+5^{2}\cdot(3+10)=3\cdot 5\cdot2^{2}+5^{2}\cdot13=3\cdot5\cdot4+25\cdot13=60+325=385$
\end{enumerate}


\section*{Άσκηση 5  \hfill \small{}}
Να γράψετε τους παρακάτω αριθμούς σε αναπτυγμένη μορφή με χρήση των δυνάμεων του 10
\begin{enumerate}[1)]
 \item 420
 \item 567
\end{enumerate}
\underline{\textbf{Λύση}}\\
\begin{enumerate}[1)]
 \item $420=4\cdot 10^{2}+2\cdot 10^{1}$
 \item $567=5\cdot10^{2}+6\cdot10^{1}+7$
\end{enumerate}

\section*{Άσκηση 6  \hfill \small{}}
Να βρείτε αν οι παρακάτω αριθμοί διαιρούνται με $2,5,9$
\begin{enumerate}[1)]
 \item 3301
 \item 4075
 \item 8010
 \item 1219
 \item 90
\end{enumerate}
\underline{\textbf{Λύση}}\\ 
\begin{enumerate}[1)]
 \item 3301: δεν διαιρείται με κανέναν από τους δοσμένους αριθμούς
 \item 4075: διαιρείται με το 5 γιατί λήγει σε 5
 \item 8010: διαιρείται με το 2 και το 5 γιατί λήγει σε 0 και \\
 διαιρείται με το 9 γιατί το άθροισμα των ψηφίψν του (που ισούται με 9) διαιρείται με το 9
 \item 1219: δεν διαιρείται με κανέναν από τους δοσμένους αριθμούς
 \item 90: διαιρείται με το 2 και το 5 γιατί λήγει σε 0 και \\
 διαιρείται με το 9 γιατί το άθροισμα των ψηφίψν του (που ισούται με 9) διαιρείται με το 9
\end{enumerate}

\section*{Άσκηση 7  \hfill \small{}}
Να βρείτε το ΕΚΠ των παρακάτω αριθμών: 
\begin{enumerate}[1)]
 \item 4,10
 \item 3,5
 \item 6,13
 \item 12,16
 \item 13,26
\end{enumerate}
\underline{\textbf{Λύση}}\\
\begin{enumerate}[1)]
 \item ΕΚΠ(4,10)=20
 \item ΕΚΠ(3,5)=15
 \item ΕΚΠ(6,13)=78
 \item ΕΚΠ(12,16)=48
 \item ΕΚΠ(13,26)=26
\end{enumerate}


\section*{Άσκηση 8  \hfill \small{}}
Να βρείτε το ΜΚΔ των παρακάτω αριθμών: 
\begin{enumerate}[1)]
 \item 12,36
 \item 17,33
 \item 45,75
 \item 24,28
 \item 7,21
\end{enumerate}
\underline{\textbf{Λύση}}\\
\begin{enumerate}[1)]
 \item ΜΚΔ(12,36)=12
 \item ΜΚΔ(17,33)=1
 \item ΜΚΔ(45,75)=15
 \item ΜΚΔ(24,28)=4
 \item ΜΚΔ(7,21)=7
\end{enumerate}


\section*{Άσκηση 9  \hfill \small{}}
Να αναλυθούν οι αριθμοί 320 και 1320 σε γινόμενα πρώτων παραγόντων. Με την βοήθεια της ανάλυσης αυτής να βρεθούν 
ο ΜΚΔ και το ΕΚΠ αυτών των αριθμών.\\
\underline{\textbf{Λύση}}\\
Κάνοντας πράξεις βρίσκουμε ότι \\
$320=2^{6}\cdot5$\\
$1320=2^{3}\cdot5\cdot33$\\
Άρα $ΜΚΔ(320,1320)=2^{3}\cdot5 =8\cdot5=40$\\
και $ΕΚΠ(320,1320)=2^{6}\cdot5\cdot33=10560$

\section*{Άσκηση 10  \hfill \small{}}
Να αναλυθούν οι αριθμοί 256 και 960 σε γινόμενα πρώτων παραγόντων. Με την βοήθεια της ανάλυσης αυτής να βρεθούν 
ο ΜΚΔ και το ΕΚΠ αυτών των αριθμών. \\
\underline{\textbf{Λύση}}\\
Κάνοντας πράξεις βρίσκουμε ότι \\
$256=2^{8}$\\
$960=2^{6}\cdot5\cdot3$\\
Άρα $ΜΚΔ(256,960)=2^{6}=64$\\
και $ΕΚΠ(256,960)=2^{8}\cdot5\cdot3=3840$


\end{document}
