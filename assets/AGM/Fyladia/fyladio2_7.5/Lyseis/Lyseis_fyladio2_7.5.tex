
\documentclass[a4paper,10pt]{report}

%\usepackage[landscape]{geometry}
\usepackage[cm-default]{fontspec}
\setromanfont{FreeSerif}
\setsansfont{FreeSans}
\setmonofont{FreeMono}
\usepackage[utf8x]{inputenc}
\usepackage{fontspec}
\usepackage{xunicode}
\usepackage{xltxtra}
\usepackage{xgreek}
\usepackage{amsmath}
\usepackage{unicode-math}
\usepackage{ulem}
\usepackage{color}
\usepackage{verbatim}
\usepackage{nopageno}
\usepackage{graphicx}
\usepackage{textpos}
\setlength{\TPHorizModule}{1cm}
\setlength{\TPVertModule}{1cm}
%\usepackage[colorgrid,texcoord]{eso-pic}
\usepackage[outline]{contour}
\usepackage{wrapfig}
\usepackage{url}
\usepackage{color}
\usepackage{tikz}
\usepackage{fancybox,fancyhdr}
\usepackage{subfigure}
\usepackage{pstricks}
\usepackage{epsfig}
\usepackage{multicol}
\usepackage{listings}
\usepackage{enumerate}
\usepackage{hyperref}
\hypersetup{
  bookmarks=true,
  bookmarksopen=true,
  pdfborder=false,
  pdfpagemode=UseNone,
  raiselinks=true,
  pdfhighlight={/P},
  colorlinks,
  citecolor=black,
  filecolor=black,
  linkcolor=black,
  urlcolor=black
}


\usepackage{fontspec}
\usepackage{xunicode}
\usepackage{xltxtra}


% Margins
%\setlength{\textwidth}{24cm}

%\setlength{\voffset}{0in}
%\setlength{\textheight}{6.8in}
% Colors
\definecolor{cbrown}{rgb}{0.49,0.24,0.07}
\definecolor{cmpez}{rgb}{0.92,0.88,0.79}
\definecolor{cmpez2}{rgb}{0.62,0.33,0.34}
\definecolor{cwhite}{rgb}{1,1,1}
\definecolor{cblack}{rgb}{0,0,0}
\definecolor{cred}{rgb}{0.9,0.15,0.15}
\definecolor{clblue}{rgb}{0.57,0.85,0.97}
\definecolor{corange}{rgb}{0.86,0.49,0.18}
\definecolor{clorange}{rgb}{0.95,0.84,0.61}
\definecolor{cyellow}{rgb}{1,0.95,0.16}
\definecolor{cgreen}{rgb}{0.69,0.8,0.31}
\definecolor{clbrown}{rgb}{0.78,0.67,0.43}
\definecolor{cmagenta}{rgb}{0.79,0.05,0.54}
\definecolor{cgray}{rgb}{0.76,0.73,0.63}
\definecolor{bluesite}{rgb}{0.05,0.43,0.67}
\definecolor{redsite}{rgb}{0.90,0.15,0.15}
\definecolor{backsite}{rgb}{0.07,0.07,0.07}
% ----------- from Costas -----------------------------

%---------------   Main Fonts    -----------------%   

% \setmainfont[Mapping=tex-text]{Calibri}
% \setmainfont[Mapping=tex-text]{Times New Roman}
%\setmainfont[Mapping=tex-text]{Droid Serif}
%\setmainfont[Mapping=tex-text]{Cambria}
%\setmainfont[Mapping=tex-text]{cm-unicode}
% \setmainfont[Mapping=tex-text]{Gentium}
% \setmainfont[Mapping=tex-text]{GFS Didot}
% \setmainfont[Mapping=tex-text]{Comic Sans MS}
 \setmainfont[Mapping=tex-text]{Ubuntu}
% \setmainfont[Mapping=tex-text]{Myriad Pro}
% \setmainfont[Mapping=tex-text]{CMU Concrete}
%\setmainfont[Mapping=tex-text]{DejaVu Sans}
%\setmainfont[Mapping=tex-text]{KerkisSans}
%\setmainfont[Mapping=tex-text]{KerkisCaligraphic}
% \setmainfont[Mapping=tex-text]{Segoe Print}
% \setmainfont[Mapping=tex-text]{Gabriola}

%------------------------------------------------%

% ------------ Mathematics Fonts ----------------%
\setmathfont{Asana-Math.ttf}
% -----------------------------------------------%
% Misc
\author{Κασωτάκη Ε. - Σμαραγδάκης Κ.}
\title{www.math24.gr}
% ---------------- Formation --------------------%


\setlength\topmargin{-0.7cm}
\addtolength\topmargin{-\headheight}
\addtolength\topmargin{-\headsep}
\setlength\textheight{26cm}
\setlength\oddsidemargin{-0.54cm}
\setlength\evensidemargin{-0.54cm}
%\setlength\marginparwidth{1.5in}
\setlength\textwidth{17cm}

\RequirePackage[avantgarde]{quotchap}
\renewcommand\chapterheadstartvskip{\vspace*{0\baselineskip}}
\RequirePackage[calcwidth]{titlesec}
\titleformat{\section}[hang]{\bfseries}
{\Large\thesection}{12pt}{\Large}[{\titlerule[0.9pt]}]
%--------------------------------------------------%
\usepackage{draftwatermark}
\SetWatermarkText{www.math24.gr}
\SetWatermarkLightness{0.9}
\SetWatermarkScale{0.6}
%--------------------------------------------------%


\usepackage{xcolor}
\usepackage{amsthm}
\usepackage{framed}
\usepackage{parskip}

\colorlet{shadecolor}{bluesite!20}

\newtheorem*{orismos}{Ορισμός}

\newenvironment{notation}
  {\begin{shaded}\begin{theorem}}
  {\end{theorem}\end{shaded}}




% Document begins
\begin{document}
\pagestyle{fancy}
\fancyhead{}
\fancyfoot{}
\renewcommand{\headrulewidth}{0pt}
\renewcommand{\footrulewidth}{0pt}


\fancyhead[LO,LE]{
 %\textblockcolor{backsite}
 %\begin{textblock}{5}(-2,-0.55)
  %\rule{0cm}{1cm}
 %\end{textblock}
 \textblockcolor{bluesite}
 \begin{textblock}{5}(-1.5,-0.55)
  \rule{0cm}{1cm}
 \end{textblock}
 %\textblockcolor{bluesite}
 %\begin{textblock}{14}(5.5,-0.55)
 % \rule{0cm}{1cm}
 %\end{textblock}
 \begin{textblock}{0}(-1,-0.25)
 \color{cwhite} \begin{Large}www.math24.gr\end{Large}
 \end{textblock}
\begin{textblock}{0}(16,-0.4)
 \color{cwhite} \includegraphics[height=1.5cm]{math24_logo.png}
 \end{textblock}
\textblockcolor{white}
\begin{textblock}{17}(-1,28)
 \color{backsite} \begin{small}Copyright \textcopyright 2013, Κασωτάκη Ε.(ikasotaki@gmail.com) - Σμαραγδάκης Κ.(kesmarag@gmail.com)\end{small}
 \end{textblock}
}
 
%\begin{shaded}
%\begin{center}
%\huge \textbf{Φυλλάδιο Ασκήσεων}\\
%Πρόσθεση ρητών αριθμών
%\end{center} 
%\textbf{Μαθηματικά Α' Γυμνασίου} \hfill \textbf{Ημερομηνία Παράδοσης : \hspace{2em} }
%\subsection*{Ονοματεπώνυμο :\hfill  \hspace{5em}}
%\end{shaded}
%\vspace{2em}
%\begin{itemize}
% \item Πολλαπλασιασμός 2 ομόσημων ρητών αριθμών
% \item Πολλαπλασιασμός 2 ετερόσημων ρητών αριθμών
% \item Γινόμενο πολλών παραγόντων
% \item Αντίστροφοι αριθμοί
%\end{itemize}
%\section*{Θεωρία - Πολλαπλασιασμός 2 ομόσημων ρητών αριθμών\hfill \small{}}
%\begin{itemize}
% \item \textbf{Ομόσημοι} λέγονται οι αριθμοί που έχουν το ίδιο πρόσημο. 
% \item \underline{ Κανόνας για τον πολλαπλασιασμό 2 ομόσημων ρητών αριθμών:} \\
%       Για να \textbf{πολλαπλασιάσουμε 2 ομόσημους} ρητούς αριθμούς, 
%       \textbf{πολλαπλασιάζουμε} τις απόλυτες τιμές τους και στο γινόμενο βάζουμε το πρόσημο "\textbf{+}"\\
%       Δηλαδή $+\cdot +=+$ και $-\cdot -=+$\\
 %      \textbf{π.χ} $(+8.22)\cdot(+100)=+822$\\
  %     \textbf{π.χ} $3\cdot71=213$\\
   %    \textbf{π.χ} $3.72\cdot 10=37.2$ \\
    %   \textbf{π.χ} $(-15)\cdot(-2)=+30$ \\
     %  \textbf{π.χ} $(-10.02)\cdot(-100)=1002$ \\
      % \textbf{π.χ} $(-3.27)\cdot(-2)=6.54$ \\
%\end{itemize}


\section*{Άσκηση 1  - Λύση \hfill \small{}}
%Να υπολογίσετε τα παρακάτω γινόμενα :
\begin{multicols}{2}
\begin{enumerate}[1)]
 \item $3.25\cdot 4=13$
 \item $10\cdot 3.77=37.7$
 \item $0.032\cdot100=3.2$
 \item $65\cdot3=195$
 \item $16\cdot4=64$
 \item $(-3)\cdot(-31)=+93$
 \item $(-10)\cdot(-8.01)=+80.1$
 \item $(-100)\cdot(-0.07)=+7$
 \item $(-\dfrac{1}{2})\cdot(-\dfrac{3}{2})=\cdots=+\dfrac{3}{4}$
 \item $(-\dfrac{1}{3})\cdot(-\dfrac{4}{5})=\cdots=+\frac{4}{15}$
 \item $4\cdot\dfrac{3}{16}=\cdots=\dfrac{3}{4}$
 \item $(-2)\cdot(-8.5)=+17$
 \item $(-7)\cdot(-82)=+574$
 \item $37\cdot23=851$
 \item $10\cdot1=10$
 \item $(-9223)\cdot(-1)=9223$
 \item $(-32)\cdot(-\dfrac{1}{16})=2$
 \item $(-4.4)\cdot(-100)=440$
 \item $(+36.3)\cdot(+2)=72.6$
 \item $\dfrac{1}{28}\cdot(\dfrac{14}{3})=\cdots=\frac{14}{84}=\dfrac{1}{6}$
\end{enumerate}
\end{multicols}



%\section*{Θεωρία - Πολλαπλασιασμός 2 ετερόσημων ρητών αριθμών\hfill \small{}}
%\begin{itemize}
% \item \textbf{Ετερόσημοι} λέγονται οι αριθμοί που έχουν διαφορετικό πρόσημο. 
% \item \underline{ Κανόνας για τον πολλαπλασιασμό 2 ετερόσημων ρητών αριθμών:} \\
%       Για να \textbf{πολλαπλασιάσουμε 2 ετερόσημους} ρητούς αριθμούς, 
%       \textbf{πολλαπλασιάζουμε} τις απόλυτες τιμές τους και στο γινόμενο βάζουμε το πρόσημο "\textbf{-}"\\
%       Δηλαδή $+\cdot -=-$ και $-\cdot +=-$\\
%       \textbf{π.χ} $(+1.1)\cdot(-32)=-35.2$\\
%       \textbf{π.χ} $(+7)\cdot(-6)=-42$\\
%       \textbf{π.χ} $(+10.23)\cdot(-100)=-1023$ \\
%       \textbf{π.χ} $(-8)\cdot(+11)=-88$ \\
%       \textbf{π.χ} $(-31.71)\cdot(+10)=-317.1$ \\
%       \textbf{π.χ} $(-36)\cdot(+23)=-828$ \\
% \end{itemize}

\section*{Άσκηση 2 - Λύση \hfill \small{}}
%Να υπολογίσετε τα παρακάτω γινόμενα :
\begin{multicols}{2}
\begin{enumerate}[1)]
 \item $(+27)\cdot(-3)=-81$
 \item $(+\dfrac{1}{8})\cdot(-\dfrac{3}{4})=\cdots=-\dfrac{3}{32}$
 \item $(+82)\cdot(-13)=-1066$
 \item $3(-89)=-267$
 \item $16.23\cdot(-1000)=-16230$
 \item $(-81)\cdot(+13)=-1053$
 \item $(-24.32)\cdot(+10)=-243.2$
 \item $(-\dfrac{3}{5})\cdot(\dfrac{2}{7})=\cdots=-\frac{6}{35}$
 \item $(-53)\cdot36=-1908$
 \item $(-108)\cdot71=-7668$
 \item $61\cdot(-7)=-427$
 \item $(-8)\cdot52=-416$
 \item $13.1\cdot(-1.3)=-17.03$
 \item $(-1.7)\cdot(+7.1)=-12.07$
 \item $81\cdot(-36)=-2916$
 \item $(-6.1)\cdot(3.7)=-22.57$
 \item $\dfrac{7}{3}\cdot(-\dfrac{4}{5})=\cdots=-\dfrac{28}{15}$
 \item $(-69)\cdot(4)=-276$
 \item $327\cdot(-5)=-1635$
 \item $(-13.01)\cdot10=-130.1$
\end{enumerate}
\end{multicols}






%\section*{Θεωρία - Γινόμενο πολλών παραγόντων \hfill \small{}}

% Για να υπολογίσουμε ένα γινόμενο \textbf{πολλών παραγόντων}, που κανένας δεν είναι μηδέν, \textbf{πολλαπλασιάζουμε} 
%       τις απόλυτες τιμές τους και στο γινόμενο βάζουμε 
%\begin{itemize}
% \item το πρόσημο \textbf{"+"} αν το πλήθος των αρνητικών παραγόντων είναι άρτιο\\
%       \textbf{π.χ} $(-3)(-3)(-1)(-2)=+18$\\
%       \textbf{π.χ} $(-7)\cdot(+2)\cdot(+\dfrac{1}{2})\cdot(-3)=+21$      
% \item το πρόσημο \textbf{"-"} αν το πλήθος των αρνητικών παραγόντων είναι περιττό\\
%       \textbf{π.χ} $(-3)(-2)(+10)(-2)=-120$\\
%       \textbf{π.χ} $(-3.2)\cdot(-2)\cdot(+10)=-64$  
%\end{itemize}
%\underline{Παρατήρηση:} Το γινόμενο πολλών παραγόντων που τουλάχιστον ένας παράγοντας είναι $0$ ισούται με μηδέν\\
%       \textbf{π.χ} $(-3)\cdot(+2087)\cdot0=0$\\
%       \textbf{π.χ} $(-3.27)\cdot0\cdot(-32)\cdot0=0$ 

%\newpage
\section*{Άσκηση 3 - Λύση\hfill \small{}}
%Να υπολογίσετε τα παρακάτω γινόμενα :
\begin{multicols}{2}
\begin{enumerate}[1)]
 \item $2.28\cdot(-100)\cdot \dfrac{1}{2}=\cdots =-114$
 \item $(-3)\cdot(-\dfrac{1}{12})\cdot(-4)\cdot(-23)=\cdots=+23$
 \item $81\cdot2\cdot(-\dfrac{1}{2})\cdot0=0$
 \item $(-1)(-2)(-1)(-2)(-1)=\cdots=-4$
 \item $(-\dfrac{1}{3})\cdot(3)\cdot(-\dfrac{1}{4})\cdot(-4)=\cdots=-1$
 \item $3.78\cdot(-2)\cdot(-5.1)\cdot0=0$
 \item $(-2)(-2)(-2)=\cdots=-8$
 \item $(-3)(+3)(-3)=\cdots=+27$
 \item $(+2)(-2)(-2)(-\dfrac{1}{2})=\cdots =-4$
 \item $100\cdot(-0.01)\cdot(38)=\cdots =-38$
\end{enumerate}
\end{multicols}

%\section*{Θεωρία - Αντίστροφοι αριθμοί \hfill \small{}}
%Οι ρητοί αριθμοί $α$ και $β$ λέγονται \textbf{αντίστροφοι}, όταν είναι διάφοροι του μηδενός και το γινόμενό τους 
%είναι ίσο με τη μονάδα, δηλαδή $α\cdot β=1$
%\begin{itemize}
% \item \textbf{π.χ} οι $\dfrac{1}{2},2$ είναι αντίστροφοι αφού 
%$\dfrac{1}{2}\cdot 2=\dfrac{1}{2}\cdot \dfrac{2}{1}=\dfrac{1\cdot2}{2\cdot1}=\dfrac{2}{2}=1$
% \item \textbf{π.χ} οι $-0.5,-2$ είναι αντίστροφοι αφού $(-0.5)\cdot(-2)=+1$
%\end{itemize}

\newpage
\section*{Άσκηση 4 - Λύση\hfill \small{}}
%Να ελέγξετε αν οι παρακάτω αριθμοί είναι αντίστροφοι
\begin{enumerate}[1)]
 \item οι $\dfrac{1}{4},4$ είναι αντίστροφοι γιατί $\dfrac{1}{4}\cdot4=1$
 \item οι $-\dfrac{3}{5},-\dfrac{5}{3}$ είναι αντίστροφοι γιατί $(-\dfrac{3}{5})\cdot(-\dfrac{5}{3})=1$ 
 \item οι $\dfrac{2}{22},11$ είναι αντίστροφοι γιατί $\dfrac{2}{22}\cdot11=\dfrac{22}{22}=1$
 \item οι $\dfrac{8}{3},\dfrac{3}{8}$ είναι αντίστροφοι γιατί $(\dfrac{8}{3})\cdot(\dfrac{3}{8})=1$ 
 \item οι $-\dfrac{3}{2},-\dfrac{4}{6}$ είναι αντίστροφοι γιατί $(-\dfrac{3}{2})\cdot(-\dfrac{4}{6})=\cdots=1$
 \item οι $-3,-\dfrac{1}{2}$ δεν είναι αντίστροφοι γιατί $(-3)\cdot(-\dfrac{1}{2})=\dfrac{3}{2}\neq1$
 \item οι $0.10,10$ είναι αντίστροφοι γιατί $0,10\cdot10=1$ 
 \item οι $-\dfrac{1}{2},2$ δεν είναι αντίστροφοι γιατί $(-\dfrac{1}{2})\cdot2=-1\neq1$
 \item οι $15,\dfrac{3}{5}$ δεν είναι αντίστροφοι γιατί $15\cdot\dfrac{3}{5}=9\neq1$
 \item οι $\dfrac{1}{65},-65$ δεν είναι αντίστροφοι γιατί $\dfrac{1}{65}\cdot(-65)=-1\neq1$
\end{enumerate}





\end{document}
