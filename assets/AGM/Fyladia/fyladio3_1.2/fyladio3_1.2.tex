
\documentclass[a4paper,10pt]{report}

%\usepackage[landscape]{geometry}
\usepackage[cm-default]{fontspec}
\setromanfont{FreeSerif}
\setsansfont{FreeSans}
\setmonofont{FreeMono}
\usepackage[utf8x]{inputenc}
\usepackage{fontspec}
\usepackage{xunicode}
\usepackage{xltxtra}
\usepackage{xgreek}
\usepackage{amsmath}
\usepackage{unicode-math}
\usepackage{ulem}
\usepackage{color}
\usepackage{verbatim}
\usepackage{nopageno}
\usepackage{graphicx}
\usepackage{textpos}
\setlength{\TPHorizModule}{1cm}
\setlength{\TPVertModule}{1cm}
%\usepackage[colorgrid,texcoord]{eso-pic}
\usepackage[outline]{contour}
\usepackage{wrapfig}
\usepackage{url}
\usepackage{color}
\usepackage{tikz}
\usepackage{fancybox,fancyhdr}
\usepackage{subfigure}
\usepackage{pstricks}
\usepackage{epsfig}
\usepackage{multicol}
\usepackage{listings}
\usepackage{enumerate}
\usepackage{hyperref}
\hypersetup{
  bookmarks=true,
  bookmarksopen=true,
  pdfborder=false,
  pdfpagemode=UseNone,
  raiselinks=true,
  pdfhighlight={/P},
  colorlinks,
  citecolor=black,
  filecolor=black,
  linkcolor=black,
  urlcolor=black
}


\usepackage{fontspec}
\usepackage{xunicode}
\usepackage{xltxtra}


% Margins
%\setlength{\textwidth}{24cm}

%\setlength{\voffset}{0in}
%\setlength{\textheight}{6.8in}
% Colors
\definecolor{cbrown}{rgb}{0.49,0.24,0.07}
\definecolor{cmpez}{rgb}{0.92,0.88,0.79}
\definecolor{cmpez2}{rgb}{0.62,0.33,0.34}
\definecolor{cwhite}{rgb}{1,1,1}
\definecolor{cblack}{rgb}{0,0,0}
\definecolor{cred}{rgb}{0.9,0.15,0.15}
\definecolor{clblue}{rgb}{0.57,0.85,0.97}
\definecolor{corange}{rgb}{0.86,0.49,0.18}
\definecolor{clorange}{rgb}{0.95,0.84,0.61}
\definecolor{cyellow}{rgb}{1,0.95,0.16}
\definecolor{cgreen}{rgb}{0.69,0.8,0.31}
\definecolor{clbrown}{rgb}{0.78,0.67,0.43}
\definecolor{cmagenta}{rgb}{0.79,0.05,0.54}
\definecolor{cgray}{rgb}{0.76,0.73,0.63}
\definecolor{bluesite}{rgb}{0.05,0.43,0.67}
\definecolor{redsite}{rgb}{0.90,0.15,0.15}
\definecolor{backsite}{rgb}{0.07,0.07,0.07}
% ----------- from Costas -----------------------------

%---------------   Main Fonts    -----------------%   

% \setmainfont[Mapping=tex-text]{Calibri}
% \setmainfont[Mapping=tex-text]{Times New Roman}
%\setmainfont[Mapping=tex-text]{Droid Serif}
%\setmainfont[Mapping=tex-text]{Cambria}
%\setmainfont[Mapping=tex-text]{cm-unicode}
% \setmainfont[Mapping=tex-text]{Gentium}
% \setmainfont[Mapping=tex-text]{GFS Didot}
% \setmainfont[Mapping=tex-text]{Comic Sans MS}
 \setmainfont[Mapping=tex-text]{Ubuntu}
% \setmainfont[Mapping=tex-text]{Myriad Pro}
% \setmainfont[Mapping=tex-text]{CMU Concrete}
%\setmainfont[Mapping=tex-text]{DejaVu Sans}
%\setmainfont[Mapping=tex-text]{KerkisSans}
%\setmainfont[Mapping=tex-text]{KerkisCaligraphic}
% \setmainfont[Mapping=tex-text]{Segoe Print}
% \setmainfont[Mapping=tex-text]{Gabriola}

%------------------------------------------------%

% ------------ Mathematics Fonts ----------------%
\setmathfont{Asana-Math.ttf}
% -----------------------------------------------%
% Misc
\author{Κασωτάκη Ε. - Σμαραγδάκης Κ.}
\title{www.math24.gr}
% ---------------- Formation --------------------%


\setlength\topmargin{-0.7cm}
\addtolength\topmargin{-\headheight}
\addtolength\topmargin{-\headsep}
\setlength\textheight{26cm}
\setlength\oddsidemargin{-0.54cm}
\setlength\evensidemargin{-0.54cm}
%\setlength\marginparwidth{1.5in}
\setlength\textwidth{17cm}

\RequirePackage[avantgarde]{quotchap}
\renewcommand\chapterheadstartvskip{\vspace*{0\baselineskip}}
\RequirePackage[calcwidth]{titlesec}
\titleformat{\section}[hang]{\bfseries}
{\Large\thesection}{12pt}{\Large}[{\titlerule[0.9pt]}]
%--------------------------------------------------%
\usepackage{draftwatermark}
\SetWatermarkText{www.math24.gr}
\SetWatermarkLightness{0.9}
\SetWatermarkScale{0.6}
%--------------------------------------------------%


\usepackage{xcolor}
\usepackage{amsthm}
\usepackage{framed}
\usepackage{parskip}

\colorlet{shadecolor}{bluesite!20}

\newtheorem*{orismos}{Ορισμός}

\newenvironment{notation}
  {\begin{shaded}\begin{theorem}}
  {\end{theorem}\end{shaded}}




% Document begins
\begin{document}
\pagestyle{fancy}
\fancyhead{}
\fancyfoot{}
\renewcommand{\headrulewidth}{0pt}
\renewcommand{\footrulewidth}{0pt}


\fancyhead[LO,LE]{
 %\textblockcolor{backsite}
 %\begin{textblock}{5}(-2,-0.55)
  %\rule{0cm}{1cm}
 %\end{textblock}
 \textblockcolor{bluesite}
 \begin{textblock}{5}(-1.5,-0.55)
  \rule{0cm}{1cm}
 \end{textblock}
 %\textblockcolor{bluesite}
 %\begin{textblock}{14}(5.5,-0.55)
 % \rule{0cm}{1cm}
 %\end{textblock}
 \begin{textblock}{0}(-1,-0.25)
 \color{cwhite} \begin{Large}www.math24.gr\end{Large}
 \end{textblock}
\begin{textblock}{0}(16,-0.4)
 \color{cwhite} \includegraphics[height=1.5cm]{math24_logo.png}
 \end{textblock}
\textblockcolor{white}
\begin{textblock}{17}(-1,28)
 \color{backsite} \begin{small}Copyright \textcopyright 2012, Κασωτάκη Ε.(ikasotaki@gmail.com) - Σμαραγδάκης Κ.(kesmarag@gmail.com)\end{small}
 \end{textblock}
}
 
\begin{shaded}
\begin{center}
\huge \textbf{Φυλλάδιο Ασκήσεων}\\
%Πρόσθεση ρητών αριθμών
\end{center} 
\textbf{Μαθηματικά Α' Γυμνασίου} \hfill \textbf{Ημερομηνία Παράδοσης : \hspace{2em} }
\subsection*{Ονοματεπώνυμο :\hfill  \hspace{5em}}
\end{shaded}
\vspace{2em}
\begin{itemize}
 \item Επιμεριστική ιδιότητα
 \item Ασκήσεις με πράξεις πρόσθεσης, αφαίρεσης και πολλαπλασιασμού
 \item Προβλήματα με πράξεις πρόσθεσης, αφαίρεσης και πολλαπλασιασμού
\end{itemize}





\section*{Άσκηση 1  \hfill \small{} 20 μονάδες}
Να αντιστοιχίσετε κάθε στοιχείο της αριστερής στήλης με ένα στοιχείο της δεξιάς στήλης
\begin{itemize}
\begin{multicols}{2}
 \item $1+10+2+5$
 \item $1\cdot10\cdot2\cdot5$
 \item $1+10+2\cdot5$
 \item $1\cdot10+2+5$
 \item $1\cdot10+2\cdot5$
 \item $21$
 \item $20$
 \item $18$
 \item $17$
 \item $100$
\end{multicols}
\end{itemize}


\section*{Θεωρία - Επιμεριστική Ιδιότητα  \hfill \small{}}
\begin{itemize}
 \item Επιμεριστική ιδιότητα του πολλαπλασιασμού ως προς την πρόσθεση: $α \cdot (β+γ)=α\cdot β+α\cdot γ$ \\
 \textbf{π.χ } $2\cdot(3+5)=2\cdot3+2\cdot5$ (γιατί $2\cdot (3+5)=2\cdot 8=16$ και $2\cdot3+2\cdot5=6+10=16$) \\ 
 \textbf{π.χ }Αντίστροφο της επιμεριστικής: $3\cdot2+3\cdot8=3\cdot(2+8)$  
(γιατί $3\cdot 2+3\cdot8=6+24=30$ και $3\cdot(2+8)=3\cdot10=30$) \\ 
 \item Επιμεριστική ιδιότητα του πολλαπλασιασμού ως προς την αφαίρεση: $α \cdot (β-γ)=α\cdot β-α\cdot γ$ \\
 \textbf{π.χ } $3\cdot(5-2)=3\cdot5-3\cdot2$ (γιατί $3\cdot (5-2)=3\cdot 3=9$ και $3\cdot5-3\cdot2=15-6=9$) \\ 
 \textbf{π.χ }Αντίστροφο της επιμεριστικής: $3\cdot12- 3\cdot2=3\cdot(12-2)$  
(γιατί $3\cdot 12 -3\cdot2=36-6=30$ και $3\cdot(12-2)=3\cdot10=30$) \\
\end{itemize}

\section*{Άσκηση 2  \hfill \small{}20  μονάδες}
Να αντιστοιχίσετε κάθε στοιχείο της αριστερής στήλης με ένα στοιχείο της δεξιάς στήλης
\begin{itemize}
\begin{multicols}{2}
 \item $7\cdot(18-13)$
 \item $7\cdot(18+13)$
 \item $3\cdot7+3\cdot8$
 \item $3\cdot(18+17)$
 \item $3\cdot18-3\cdot7$
 \item $7\cdot18-7\cdot13$
 \item $3\cdot(7+8)$
 \item $7\cdot18+7\cdot13$
 \item $18\cdot3+17\cdot3$
 \item $3\cdot(18-7)$
\end{multicols}
\end{itemize}

\section*{Άσκηση 3  \hfill \small{}20 μονάδες}
Να υπολογίσετε τα παρακάτω γινόμενα χρησιμοποιώντας την επιμεριστική ιδιότητα: 
\begin{enumerate} [1)]
 \item $6\cdot 16$
 \item $5\cdot 13$
 \item $20\cdot 12$
 \item $5\cdot 99$
 \item $7\cdot 98$
\end{enumerate}



\section*{Άσκηση 4  \hfill \small{}20 μονάδες}
Οι γονείς του Κώστα του έδωσαν 70\texteuro για να αγοράσει δώρο για τα γενέθλιά του. Ο Κώστας πήγε σε ένα 
κατάστημα υπολογιστών και βρήκε ένα πληκτρολόγιο που κόστιζε 16\texteuro, ένα ποντίκι που κόστιζε 11\texteuro 
και ηχεία που κόστιζαν 35\texteuro. Τον φτάνουν τα χρήματα για να τα αγοράσει όλα;



\section*{Άσκηση 5  \hfill \small{}20  μονάδες}
Η Ειρήνη γεννήθηκε το 1982 και είναι 27 χρόνια μικρότερη από τη μητέρα της.
\begin{enumerate}[i)]
 \item Ποια χρονολογία γεννήθηκε η μητέρα της Ειρήνης;
 \item Πόσων χρονών είναι σήμερα η Ειρήνη;
 \item Πόσων χρονών είναι σήμερα η μητέρα της Ειρήνης;
\end{enumerate}





 









%\let\thefootnote\relax\footnotetext{Επιμέλεια : Κασωτάκη Ειρήνη (ikasotaki@gmail.com),    Σμαραγδάκης Κώστας (kesmarag@gmail.com)} 


\end{document}
