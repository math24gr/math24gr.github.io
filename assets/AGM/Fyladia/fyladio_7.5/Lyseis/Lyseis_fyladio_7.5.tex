
\documentclass[a4paper,10pt]{report}

%\usepackage[landscape]{geometry}
\usepackage[cm-default]{fontspec}
\setromanfont{FreeSerif}
\setsansfont{FreeSans}
\setmonofont{FreeMono}
\usepackage[utf8x]{inputenc}
\usepackage{fontspec}
\usepackage{xunicode}
\usepackage{xltxtra}
\usepackage{xgreek}
\usepackage{amsmath}
\usepackage{unicode-math}
\usepackage{ulem}
\usepackage{color}
\usepackage{verbatim}
\usepackage{nopageno}
\usepackage{graphicx}
\usepackage{textpos}
\setlength{\TPHorizModule}{1cm}
\setlength{\TPVertModule}{1cm}
%\usepackage[colorgrid,texcoord]{eso-pic}
\usepackage[outline]{contour}
\usepackage{wrapfig}
\usepackage{url}
\usepackage{color}
\usepackage{tikz}
\usepackage{fancybox,fancyhdr}
\usepackage{subfigure}
\usepackage{pstricks}
\usepackage{epsfig}
\usepackage{multicol}
\usepackage{listings}
\usepackage{enumerate}
\usepackage{hyperref}
\hypersetup{
  bookmarks=true,
  bookmarksopen=true,
  pdfborder=false,
  pdfpagemode=UseNone,
  raiselinks=true,
  pdfhighlight={/P},
  colorlinks,
  citecolor=black,
  filecolor=black,
  linkcolor=black,
  urlcolor=black
}


\usepackage{fontspec}
\usepackage{xunicode}
\usepackage{xltxtra}


% Margins
%\setlength{\textwidth}{24cm}

%\setlength{\voffset}{0in}
%\setlength{\textheight}{6.8in}
% Colors
\definecolor{cbrown}{rgb}{0.49,0.24,0.07}
\definecolor{cmpez}{rgb}{0.92,0.88,0.79}
\definecolor{cmpez2}{rgb}{0.62,0.33,0.34}
\definecolor{cwhite}{rgb}{1,1,1}
\definecolor{cblack}{rgb}{0,0,0}
\definecolor{cred}{rgb}{0.9,0.15,0.15}
\definecolor{clblue}{rgb}{0.57,0.85,0.97}
\definecolor{corange}{rgb}{0.86,0.49,0.18}
\definecolor{clorange}{rgb}{0.95,0.84,0.61}
\definecolor{cyellow}{rgb}{1,0.95,0.16}
\definecolor{cgreen}{rgb}{0.69,0.8,0.31}
\definecolor{clbrown}{rgb}{0.78,0.67,0.43}
\definecolor{cmagenta}{rgb}{0.79,0.05,0.54}
\definecolor{cgray}{rgb}{0.76,0.73,0.63}
\definecolor{bluesite}{rgb}{0.05,0.43,0.67}
\definecolor{redsite}{rgb}{0.90,0.15,0.15}
\definecolor{backsite}{rgb}{0.07,0.07,0.07}
% ----------- from Costas -----------------------------

%---------------   Main Fonts    -----------------%   

% \setmainfont[Mapping=tex-text]{Calibri}
% \setmainfont[Mapping=tex-text]{Times New Roman}
%\setmainfont[Mapping=tex-text]{Droid Serif}
%\setmainfont[Mapping=tex-text]{Cambria}
%\setmainfont[Mapping=tex-text]{cm-unicode}
% \setmainfont[Mapping=tex-text]{Gentium}
% \setmainfont[Mapping=tex-text]{GFS Didot}
% \setmainfont[Mapping=tex-text]{Comic Sans MS}
 \setmainfont[Mapping=tex-text]{Ubuntu}
% \setmainfont[Mapping=tex-text]{Myriad Pro}
% \setmainfont[Mapping=tex-text]{CMU Concrete}
%\setmainfont[Mapping=tex-text]{DejaVu Sans}
%\setmainfont[Mapping=tex-text]{KerkisSans}
%\setmainfont[Mapping=tex-text]{KerkisCaligraphic}
% \setmainfont[Mapping=tex-text]{Segoe Print}
% \setmainfont[Mapping=tex-text]{Gabriola}

%------------------------------------------------%

% ------------ Mathematics Fonts ----------------%
\setmathfont{Asana-Math.ttf}
% -----------------------------------------------%
% Misc
\author{Κασωτάκη Ε. - Σμαραγδάκης Κ.}
\title{www.math24.gr}
% ---------------- Formation --------------------%


\setlength\topmargin{-0.7cm}
\addtolength\topmargin{-\headheight}
\addtolength\topmargin{-\headsep}
\setlength\textheight{26cm}
\setlength\oddsidemargin{-0.54cm}
\setlength\evensidemargin{-0.54cm}
%\setlength\marginparwidth{1.5in}
\setlength\textwidth{17cm}

\RequirePackage[avantgarde]{quotchap}
\renewcommand\chapterheadstartvskip{\vspace*{0\baselineskip}}
\RequirePackage[calcwidth]{titlesec}
\titleformat{\section}[hang]{\bfseries}
{\Large\thesection}{12pt}{\Large}[{\titlerule[0.9pt]}]
%--------------------------------------------------%
\usepackage{draftwatermark}
\SetWatermarkText{www.math24.gr}
\SetWatermarkLightness{0.9}
\SetWatermarkScale{0.6}
%--------------------------------------------------%


\usepackage{xcolor}
\usepackage{amsthm}
\usepackage{framed}
\usepackage{parskip}

\colorlet{shadecolor}{bluesite!20}

\newtheorem*{orismos}{Ορισμός}

\newenvironment{notation}
  {\begin{shaded}\begin{theorem}}
  {\end{theorem}\end{shaded}}




% Document begins
\begin{document}
\pagestyle{fancy}
\fancyhead{}
\fancyfoot{}
\renewcommand{\headrulewidth}{0pt}
\renewcommand{\footrulewidth}{0pt}


\fancyhead[LO,LE]{
 %\textblockcolor{backsite}
 %\begin{textblock}{5}(-2,-0.55)
  %\rule{0cm}{1cm}
 %\end{textblock}
 \textblockcolor{bluesite}
 \begin{textblock}{5}(-1.5,-0.55)
  \rule{0cm}{1cm}
 \end{textblock}
 %\textblockcolor{bluesite}
 %\begin{textblock}{14}(5.5,-0.55)
 % \rule{0cm}{1cm}
 %\end{textblock}
 \begin{textblock}{0}(-1,-0.25)
 \color{cwhite} \begin{Large}www.math24.gr\end{Large}
 \end{textblock}
\begin{textblock}{0}(16,-0.4)
 \color{cwhite} \includegraphics[height=1.5cm]{math24_logo.png}
 \end{textblock}
\textblockcolor{white}
\begin{textblock}{17}(-1,28)
 \color{backsite} \begin{small}Copyright \textcopyright 2011, Κασωτάκη Ε.(ikasotaki@gmail.com) - Σμαραγδάκης Κ.(kesmarag@gmail.com)\end{small}
 \end{textblock}
}
 
%\begin{shaded}
%\begin{center}
%\huge \textbf{Φυλλάδιο Ασκήσεων}\\
%Πρόσθεση ρητών αριθμών
%\end{center} 
%\textbf{Μαθηματικά Α' Γυμνασίου} \hfill \textbf{Ημερομηνία Παράδοσης : \hspace{2em} }
%\subsection*{Ονοματεπώνυμο :\hfill  \hspace{5em}}
%\end{shaded}
%\vspace{2em}
%\begin{itemize}
% \item Πολλαπλασιασμός 2 ομόσημων  αριθμών
% \item Πολλαπλασιασμός 2 ετερόσημων αριθμών
% \item Γινόμενο πολλών παραγόντων
%\end{itemize}
%\section*{Θεωρία - Πολλαπλασιασμός 2 ομόσημων αριθμών\hfill \small{}}
%\begin{itemize}
% \item \textbf{Ομόσημοι} λέγονται οι αριθμοί που έχουν το ίδιο πρόσημο. 
% \item \underline{ Κανόνας για τον πολλαπλασιασμό 2 ομόσημων  αριθμών:} \\
%       Για να \textbf{πολλαπλασιάσουμε 2 ομόσημους}  αριθμούς, 
%       \textbf{πολλαπλασιάζουμε} τις απόλυτες τιμές τους και στο γινόμενο βάζουμε το πρόσημο "\textbf{+}"\\
%      Δηλαδή $+\cdot +=+$ και $-\cdot -=+$\\
%       \textbf{π.χ} $(+8)\cdot(+100)=+800$\\
%      \textbf{π.χ} $3\cdot71=213$\\
%      \textbf{π.χ} $37\cdot 10=37$ \\
%      \textbf{π.χ} $(-15)\cdot(-2)=+30$ \\
%      \textbf{π.χ} $(-101)\cdot(-100)=10100$ \\
%      \textbf{π.χ} $(-3)\cdot(-2)=6$ \\
%\end{itemize}


\section*{Άσκηση 1 - Λύση\hfill \small{}}
%Να υπολογίσετε τα παρακάτω γινόμενα :
%\begin{multicols}{2}
\begin{enumerate}[1)]
 \item $32\cdot 4=128$
 \item $10\cdot 377=3770$
 \item $32\cdot100=3200$
 \item $65\cdot3=195$
 \item $16\cdot4=64$
 \item $(-3)\cdot(-31)=93$
 \item $(-10)\cdot(-801)=8010$
 \item $(-100)\cdot(-7)=700$
 \item $(-5)\cdot(-13)=65$
 \item $(-25)\cdot(-4)=100$
 \item $4\cdot16=64$
 \item $(-2)\cdot(-85)=170$
 \item $(-7)\cdot(-82)=574$
 \item $37\cdot23=851$
 \item $10\cdot1=10$
 \item $(-9223)\cdot(-1)=9223$
 \item $(-32)\cdot(-2)=64$
 \item $(-44)\cdot(-100)=4400$
 \item $(+36)\cdot(+3)=108$
 \item $90\cdot3=270$
\end{enumerate}
%\end{multicols}

%\newpage

%\section*{Θεωρία - Πολλαπλασιασμός 2 ετερόσημων αριθμών\hfill \small{}}
%\begin{itemize}
% \item \textbf{Ετερόσημοι} λέγονται οι αριθμοί που έχουν διαφορετικό πρόσημο. 
% \item \underline{ Κανόνας για τον πολλαπλασιασμό 2 ετερόσημων  αριθμών:} \\
%       Για να \textbf{πολλαπλασιάσουμε 2 ετερόσημους}  αριθμούς, 
%       \textbf{πολλαπλασιάζουμε} τις απόλυτες τιμές τους και στο γινόμενο βάζουμε το πρόσημο "\textbf{-}"\\
%       Δηλαδή $+\cdot -=-$ και $-\cdot +=-$\\
%       \textbf{π.χ} $(+1)\cdot(-32)=-32$\\
%       \textbf{π.χ} $(+7)\cdot(-6)=-42$\\
%       \textbf{π.χ} $(+1023)\cdot(-100)=-102300$ \\
%       \textbf{π.χ} $(-8)\cdot(+11)=-88$ \\
%       \textbf{π.χ} $(-317)\cdot(+10)=-3170$ \\
%       \textbf{π.χ} $(-36)\cdot(+23)=-828$ \\
%\end{itemize}



\section*{Άσκηση 2 - Λύση\hfill \small{}}
%Να υπολογίσετε τα παρακάτω γινόμενα :
%\begin{multicols}{2}
\begin{enumerate}[1)]
 \item $(+27)\cdot(-3)=-81$
 \item $(+81)\cdot(-3)=-243$
 \item $(+82)\cdot(-13)=-1066$
 \item $3(-89)=-267$
 \item $163\cdot(-1000)=-163000$
 \item $(-81)\cdot(+13)=-1053$
 \item $(-242)\cdot(+10)=-2420$
 \item $(-5)\cdot(23)=-115$
 \item $(-53)\cdot36=-1908$
 \item $(-108)\cdot71=-7668$
 \item $61\cdot(-7)=-427$
 \item $(-8)\cdot52=-416$
 \item $131\cdot(-2)=-262$
 \item $(-17)\cdot(+7)=-119$
 \item $81\cdot(-36)=-2916$
 \item $(-61)\cdot(3)=-183$
 \item $45\cdot(-4)=-180$
 \item $(-69)\cdot(4)=-276$
 \item $327\cdot(-5)=-1635$
 \item $(-139)\cdot10=-1390$
\end{enumerate}
%\end{multicols}






%\section*{Θεωρία - Γινόμενο πολλών παραγόντων \hfill \small{}}

 %Για να υπολογίσουμε ένα γινόμενο \textbf{πολλών παραγόντων}, που κανένας δεν είναι μηδέν, \textbf{πολλαπλασιάζουμε} 
 %      τις απόλυτες τιμές τους και στο γινόμενο βάζουμε 
%\begin{itemize}
% \item το πρόσημο \textbf{"+"} αν το πλήθος των αρνητικών παραγόντων είναι άρτιο\\
%       \textbf{π.χ} $(-3)(-3)(-1)(-2)=+18$\\
%       \textbf{π.χ} $(-7)\cdot(+2)\cdot(+3)\cdot(-3)=+126$      
% \item το πρόσημο \textbf{"-"} αν το πλήθος των αρνητικών παραγόντων είναι περιττό\\
%       \textbf{π.χ} $(-3)(-2)(+10)(-2)=-120$\\
%       \textbf{π.χ} $(-32)\cdot(-2)\cdot(-10)=-640$  
%\end{itemize}
%\underline{Παρατήρηση:} Το γινόμενο πολλών παραγόντων που τουλάχιστον ένας παράγοντας είναι $0$ ισούται με μηδέν\\
%       \textbf{π.χ} $(-3)\cdot(+2087)\cdot0=0$\\
%       \textbf{π.χ} $(-327)\cdot0\cdot(-32)\cdot0=0$ 

%\newpage
\section*{Άσκηση 3 - Λύση\hfill \small{}}
%Να υπολογίσετε τα παρακάτω γινόμενα :
%\begin{multicols}{2}
\begin{enumerate}[1)]
 \item $228\cdot(-100)\cdot 1=-22800$
 \item $(-3)\cdot(-1)\cdot(-4)\cdot(-23)=+276$
 \item $81\cdot2\cdot(-45)\cdot0=0$
 \item $(-1)(-2)(-1)(-2)(-1)=-4$
 \item $(-6)\cdot(3)\cdot(-5)\cdot(-4)=-360$
 \item $378\cdot(-2)\cdot(-5)\cdot0=0$
 \item $(-2)(-2)(-2)=-8$
 \item $(-3)(+3)(-3)+27$
 \item $(+2)(-2)(-2)(-2)=-16$
 \item $100\cdot(-2)\cdot(38)=-7600$
\end{enumerate}
%\end{multicols}






\end{document}
