
\documentclass[a4paper,10pt]{report}

%\usepackage[landscape]{geometry}
\usepackage[cm-default]{fontspec}
\setromanfont{FreeSerif}
\setsansfont{FreeSans}
\setmonofont{FreeMono}
\usepackage[utf8x]{inputenc}
\usepackage{fontspec}
\usepackage{xunicode}
\usepackage{xltxtra}
\usepackage{xgreek}
\usepackage{amsmath}
\usepackage{unicode-math}
\usepackage{ulem}
\usepackage{color}
\usepackage{verbatim}
\usepackage{nopageno}
\usepackage{graphicx}
\usepackage{textpos}
\setlength{\TPHorizModule}{1cm}
\setlength{\TPVertModule}{1cm}
%\usepackage[colorgrid,texcoord]{eso-pic}
\usepackage[outline]{contour}
\usepackage{wrapfig}
\usepackage{url}
\usepackage{color}
\usepackage{tikz}
\usepackage{fancybox,fancyhdr}
\usepackage{subfigure}
\usepackage{pstricks}
\usepackage{epsfig}
\usepackage{multicol}
\usepackage{listings}
\usepackage{enumerate}
\usepackage{hyperref}
\hypersetup{
  bookmarks=true,
  bookmarksopen=true,
  pdfborder=false,
  pdfpagemode=UseNone,
  raiselinks=true,
  pdfhighlight={/P},
  colorlinks,
  citecolor=black,
  filecolor=black,
  linkcolor=black,
  urlcolor=black
}


\usepackage{fontspec}
\usepackage{xunicode}
\usepackage{xltxtra}


% Margins
%\setlength{\textwidth}{24cm}

%\setlength{\voffset}{0in}
%\setlength{\textheight}{6.8in}
% Colors
\definecolor{cbrown}{rgb}{0.49,0.24,0.07}
\definecolor{cmpez}{rgb}{0.92,0.88,0.79}
\definecolor{cmpez2}{rgb}{0.62,0.33,0.34}
\definecolor{cwhite}{rgb}{1,1,1}
\definecolor{cblack}{rgb}{0,0,0}
\definecolor{cred}{rgb}{0.9,0.15,0.15}
\definecolor{clblue}{rgb}{0.57,0.85,0.97}
\definecolor{corange}{rgb}{0.86,0.49,0.18}
\definecolor{clorange}{rgb}{0.95,0.84,0.61}
\definecolor{cyellow}{rgb}{1,0.95,0.16}
\definecolor{cgreen}{rgb}{0.69,0.8,0.31}
\definecolor{clbrown}{rgb}{0.78,0.67,0.43}
\definecolor{cmagenta}{rgb}{0.79,0.05,0.54}
\definecolor{cgray}{rgb}{0.76,0.73,0.63}
\definecolor{bluesite}{rgb}{0.05,0.43,0.67}
\definecolor{redsite}{rgb}{0.90,0.15,0.15}
\definecolor{backsite}{rgb}{0.07,0.07,0.07}
% ----------- from Costas -----------------------------

%---------------   Main Fonts    -----------------%   

% \setmainfont[Mapping=tex-text]{Calibri}
% \setmainfont[Mapping=tex-text]{Times New Roman}
%\setmainfont[Mapping=tex-text]{Droid Serif}
%\setmainfont[Mapping=tex-text]{Cambria}
%\setmainfont[Mapping=tex-text]{cm-unicode}
% \setmainfont[Mapping=tex-text]{Gentium}
% \setmainfont[Mapping=tex-text]{GFS Didot}
% \setmainfont[Mapping=tex-text]{Comic Sans MS}
 \setmainfont[Mapping=tex-text]{Ubuntu}
% \setmainfont[Mapping=tex-text]{Myriad Pro}
% \setmainfont[Mapping=tex-text]{CMU Concrete}
%\setmainfont[Mapping=tex-text]{DejaVu Sans}
%\setmainfont[Mapping=tex-text]{KerkisSans}
%\setmainfont[Mapping=tex-text]{KerkisCaligraphic}
% \setmainfont[Mapping=tex-text]{Segoe Print}
% \setmainfont[Mapping=tex-text]{Gabriola}

%------------------------------------------------%

% ------------ Mathematics Fonts ----------------%
%\setmathfont{Asana-Math.ttf}
% -----------------------------------------------%
% Misc
\author{Κασωτάκη Ε. - Σμαραγδάκης Κ.}
\title{www.math24.gr}
% ---------------- Formation --------------------%


\setlength\topmargin{-0.7cm}
\addtolength\topmargin{-\headheight}
\addtolength\topmargin{-\headsep}
\setlength\textheight{26cm}
\setlength\oddsidemargin{-0.54cm}
\setlength\evensidemargin{-0.54cm}
%\setlength\marginparwidth{1.5in}
\setlength\textwidth{17cm}

\RequirePackage[avantgarde]{quotchap}
\renewcommand\chapterheadstartvskip{\vspace*{0\baselineskip}}
\RequirePackage[calcwidth]{titlesec}
\titleformat{\section}[hang]{\bfseries}
{\Large\thesection}{12pt}{\Large}[{\titlerule[0.9pt]}]
%--------------------------------------------------%
%\usepackage{draftwatermark}
%\SetWatermarkText{www.math24.gr}
%\SetWatermarkLightness{0.9}
%\SetWatermarkScale{0.6}
%--------------------------------------------------%


\usepackage{xcolor}
\usepackage{amsthm}
\usepackage{framed}
\usepackage{parskip}

\colorlet{shadecolor}{bluesite!20}

\newtheorem*{orismos}{Ορισμός}

\newenvironment{notation}
  {\begin{shaded}\begin{theorem}}
  {\end{theorem}\end{shaded}}




% Document begins
\begin{document}
\pagestyle{fancy}
\fancyhead{}
\fancyfoot{}
\renewcommand{\headrulewidth}{0pt}
\renewcommand{\footrulewidth}{0pt}


\fancyhead[LO,LE]{
 %\textblockcolor{backsite}
 %\begin{textblock}{5}(-2,-0.55)
  %\rule{0cm}{1cm}
 %\end{textblock}
 \textblockcolor{bluesite}
 \begin{textblock}{5}(-1.5,-0.55)
  \rule{0cm}{1cm}
 \end{textblock}
 %\textblockcolor{bluesite}
 %\begin{textblock}{14}(5.5,-0.55)
 % \rule{0cm}{1cm}
 %\end{textblock}
 \begin{textblock}{0}(-1,-0.25)
 \color{cwhite} \begin{Large}www.math24.gr\end{Large}
 \end{textblock}
\begin{textblock}{0}(16,-0.4)
 \color{cwhite} \includegraphics[height=1.5cm]{math24_logo.png}
 \end{textblock}
\textblockcolor{white}
\begin{textblock}{17}(-1,28)
 \color{backsite} \begin{small}Επιμέλεια: Κασωτάκη Ειρήνη (ikasotaki@gmail.com) - Σχολικό έτος 2019 - 2020\end{small}
 \end{textblock}
}
 
\begin{shaded}
\begin{center}
\huge \textbf{Φυλλάδιο Ασκήσεων}\\
%Πρόσθεση ρητών αριθμών
\end{center} 
\textbf{Μαθηματικά Α' Γυμνασίου} \hfill \textbf{Ημερομηνία Παράδοσης : \hspace{2em} }
\subsection*{Ονοματεπώνυμο :\hfill  \hspace{5em}}
\end{shaded}
\vspace{2em}
\begin{itemize}
 \item Μετατροπή λεκτικών εκφράσεων σε μαθηματικές εκφράσεις
 \item Μαθηματικές εκφράσεις διατυπωμένες με απλούστερο τρόπο
\end{itemize}

\section*{Θεωρία - Μετατροπή λεκτικών εκφράσεων σε μαθηματικές εκφράσεις\hfill \small{}}
\begin{center}
 \begin{tabular}{|l|c|}\hline 
\textbf{Λεκτική πρόταση} \quad        &    \textbf{Μαθηματική πρόταση}       \\
\hline 
Ο επόμενος ενός φυσικού αριθμού                      & $n+1$        \\
\hline 
Ο προηγούμενος ενός φυσικού αριθμού                  & $n-1$ \\           
\hline
Ένας αριθμός αυξάνεται κατά $7$                      & $x+7$        \\
\hline
Ένας αριθμός μειώνεται κατά $3$                      & $x-3$\\
\hline
Το τετραπλάσιο ενός αριθμού                          &  $4x$     \\
\hline 
Το διπλάσιο ενός αριθμού αυξημένο κατά 5             &  $2x+5$     \\
\hline 
Το τριπλάσιο ενός αριθμού ελαττωμένο κατά 1          &  $3x-1$     \\
\hline 
Το άθροισμα δύο αριθμών                              &  $x+y$     \\
\hline 
Το διπλάσιο ενός αριθμού αυξημένο κατά 5             &  $2x+5$     \\
\hline 
Η διαφορά δύο αριθμών                                &  $x-y$     \\
\hline 
Τα πολλαπλάσια του  5                                &  $5α$     \\
\hline 
Αν σ' ένα αριθμό προσθέσουμε $2$, μας δίνει $17$     &  $x+2=17$     \\
\hline 
Αν από ένα αριθμό αφαιρέσουμε $3$, μας δίνει $22$    &  $x-3=22$     \\
\hline 
Ένας άρτιος φυσικός αριθμός                          &  $2κ$     \\
\hline 
Ένας περιττός φυσικός αριθμός                        &  $2κ+1$     \\
\hline 
\end{tabular}
\end{center}


\section*{Άσκηση 1  \hfill \small{20 μονάδες}}
Να μετατρέψετε τις παρακάτω λεκτικές προτάσεις σε μαθηματικές εκφράσεις:
\begin{enumerate}[1)]
 \item Ένας αριθμός αυξάνεται κατά $23$.
 \item Ένας αριθμός μειώνεται κατά $13$.
 \item Το πενταπλάσιο ενός αριθμού.
 \item Το τριπλάσιο ενός αριθμού αυξημένο κατά $32$.
 \item Το τετραπλάσιο ενός αριθμού ελαττωμένο κατά $27$.
 \item Το άθροισμα δύο αριθμών είναι ίσο με $37$.
 \item Η διαφορά δύο αριθμών είναι μεγαλύτερη από $4$.
 \item Αν σ' ένα αριθμό προσθέσουμε το $7$, μας δίνει $41$.
 \item Το άθροισμα δύο αριθμών αυξημένο κατά $8$.
 \item Η διαφορά δύο αριθμών είναι ίση με $50$.
\end{enumerate}




\section*{Θεωρία - Μαθηματικές εκφράσεις διατυπωμένες με απλούστερο τρόπο \hfill \small{}}
\begin{center}
 \begin{tabular}{|l|c|}\hline 
\textbf{Μαθηματική έκφραση} \quad        &    \textbf{Απλούστερη μαθηματική έκφραση}       \\
\hline 
$α+α+α$                                  &  $3\cdot α$\\ 
\hline
$2\cdot α+2\cdot α$                      &  $5\cdot α$        \\                        
\hline
$α+α+β+β+β$                              & $2\cdot α+3\cdot β$       \\
\hline
$4\cdot x +2\cdot x+3\cdot x$            &  $9\cdot x$     \\
\hline 
$5\cdot β-3\cdot β$                      & $2\cdot β$       \\
\hline
$2\cdot x+3\cdot x+4\cdot y+2\cdot y$    & $5\cdot x+6\cdot 6$       \\
\hline
\end{tabular}
\end{center}




\section*{Άσκηση 2  \hfill \small{80 μονάδες}}
Να γράψετε με απλούστερο τρόπο τις παρακάτω μαθηματικές εκφράσεις:
\begin{enumerate}[1)]
 \item $β+β+β+β$
 \item $2\cdot x+5\cdot x$
 \item $2\cdot α+ 3\cdot α+2\cdot β+4\cdot β$
 \item $α+α+α+β+β+β$
 \item $2\cdot x+2\cdot x+3\cdot y +2\cdot y$
 \item $4\cdot x -2\cdot x  $
 \item $5\cdot x+3\cdot x -4\cdot x$
 \item $5\cdot α+2\cdot β-3\cdot α+4\cdot β$
 \item $5\cdot x-2\cdot x+6\cdot x -3\cdot x$
 \item $13\cdot α-2\cdot β+10\cdot β+3\cdot α$
 \item $88\cdot α + 12\cdot α+β+2\cdot β+γ+γ$
 \item $100\cdot α-15\cdot α+20\cdot β +30\cdot β -10\cdot β$
 \item $19\cdot α+3\cdot α+α+10\cdot β-2\cdot β -3\cdot β$
 \item $32\cdot α-16\cdot α-16\cdot β +32\cdot β$
 \item $x+x+4\cdot x+5\cdot x+9\cdot y+3\cdot y$
 \item $20\cdot x+10\cdot x-5\cdot x+30\cdot y+70\cdot y-90\cdot y$
 \item $2\cdot x-2\cdot y-2\cdot ω+5\cdot x+5\cdot y+5\cdot ω$
 \item $2\cdot x+3\cdot x+4\cdot x-2\cdot y-2\cdot x+12\cdot y$
 \item $4\cdot x-2\cdot y+5\cdot x-3\cdot y-3\cdot x+10\cdot y$
 \item $23\cdot x+5\cdot x-3\cdot x+2\cdot y$
\end{enumerate}













\end{document}
