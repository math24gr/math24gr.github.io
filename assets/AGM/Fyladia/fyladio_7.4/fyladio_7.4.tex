
\documentclass[a4paper,10pt]{report}

%\usepackage[landscape]{geometry}
\usepackage[cm-default]{fontspec}
\setromanfont{FreeSerif}
\setsansfont{FreeSans}
\setmonofont{FreeMono}
\usepackage[utf8x]{inputenc}
\usepackage{fontspec}
\usepackage{xunicode}
\usepackage{xltxtra}
\usepackage{xgreek}
\usepackage{amsmath}
\usepackage{unicode-math}
\usepackage{ulem}
\usepackage{color}
\usepackage{verbatim}
\usepackage{nopageno}
\usepackage{graphicx}
\usepackage{textpos}
\setlength{\TPHorizModule}{1cm}
\setlength{\TPVertModule}{1cm}
%\usepackage[colorgrid,texcoord]{eso-pic}
\usepackage[outline]{contour}
\usepackage{wrapfig}
\usepackage{url}
\usepackage{color}
\usepackage{tikz}
\usepackage{fancybox,fancyhdr}
\usepackage{subfigure}
\usepackage{pstricks}
\usepackage{epsfig}
\usepackage{multicol}
\usepackage{listings}
\usepackage{enumerate}
\usepackage{hyperref}
\hypersetup{
  bookmarks=true,
  bookmarksopen=true,
  pdfborder=false,
  pdfpagemode=UseNone,
  raiselinks=true,
  pdfhighlight={/P},
  colorlinks,
  citecolor=black,
  filecolor=black,
  linkcolor=black,
  urlcolor=black
}


\usepackage{fontspec}
\usepackage{xunicode}
\usepackage{xltxtra}


% Margins
%\setlength{\textwidth}{24cm}

%\setlength{\voffset}{0in}
%\setlength{\textheight}{6.8in}
% Colors
\definecolor{cbrown}{rgb}{0.49,0.24,0.07}
\definecolor{cmpez}{rgb}{0.92,0.88,0.79}
\definecolor{cmpez2}{rgb}{0.62,0.33,0.34}
\definecolor{cwhite}{rgb}{1,1,1}
\definecolor{cblack}{rgb}{0,0,0}
\definecolor{cred}{rgb}{0.9,0.15,0.15}
\definecolor{clblue}{rgb}{0.57,0.85,0.97}
\definecolor{corange}{rgb}{0.86,0.49,0.18}
\definecolor{clorange}{rgb}{0.95,0.84,0.61}
\definecolor{cyellow}{rgb}{1,0.95,0.16}
\definecolor{cgreen}{rgb}{0.69,0.8,0.31}
\definecolor{clbrown}{rgb}{0.78,0.67,0.43}
\definecolor{cmagenta}{rgb}{0.79,0.05,0.54}
\definecolor{cgray}{rgb}{0.76,0.73,0.63}
\definecolor{bluesite}{rgb}{0.05,0.43,0.67}
\definecolor{redsite}{rgb}{0.90,0.15,0.15}
\definecolor{backsite}{rgb}{0.07,0.07,0.07}
% ----------- from Costas -----------------------------

%---------------   Main Fonts    -----------------%   

% \setmainfont[Mapping=tex-text]{Calibri}
% \setmainfont[Mapping=tex-text]{Times New Roman}
%\setmainfont[Mapping=tex-text]{Droid Serif}
%\setmainfont[Mapping=tex-text]{Cambria}
%\setmainfont[Mapping=tex-text]{cm-unicode}
% \setmainfont[Mapping=tex-text]{Gentium}
% \setmainfont[Mapping=tex-text]{GFS Didot}
% \setmainfont[Mapping=tex-text]{Comic Sans MS}
 \setmainfont[Mapping=tex-text]{Ubuntu}
% \setmainfont[Mapping=tex-text]{Myriad Pro}
% \setmainfont[Mapping=tex-text]{CMU Concrete}
%\setmainfont[Mapping=tex-text]{DejaVu Sans}
%\setmainfont[Mapping=tex-text]{KerkisSans}
%\setmainfont[Mapping=tex-text]{KerkisCaligraphic}
% \setmainfont[Mapping=tex-text]{Segoe Print}
% \setmainfont[Mapping=tex-text]{Gabriola}

%------------------------------------------------%

% ------------ Mathematics Fonts ----------------%
\setmathfont{Asana-Math.ttf}
% -----------------------------------------------%
% Misc
\author{Κασωτάκη Ε. - Σμαραγδάκης Κ.}
\title{www.math24.gr}
% ---------------- Formation --------------------%


\setlength\topmargin{-0.7cm}
\addtolength\topmargin{-\headheight}
\addtolength\topmargin{-\headsep}
\setlength\textheight{26cm}
\setlength\oddsidemargin{-0.54cm}
\setlength\evensidemargin{-0.54cm}
%\setlength\marginparwidth{1.5in}
\setlength\textwidth{17cm}

\RequirePackage[avantgarde]{quotchap}
\renewcommand\chapterheadstartvskip{\vspace*{0\baselineskip}}
\RequirePackage[calcwidth]{titlesec}
\titleformat{\section}[hang]{\bfseries}
{\Large\thesection}{12pt}{\Large}[{\titlerule[0.9pt]}]
%--------------------------------------------------%
\usepackage{draftwatermark}
\SetWatermarkText{www.math24.gr}
\SetWatermarkLightness{0.9}
\SetWatermarkScale{0.6}
%--------------------------------------------------%


\usepackage{xcolor}
\usepackage{amsthm}
\usepackage{framed}
\usepackage{parskip}

\colorlet{shadecolor}{bluesite!20}

\newtheorem*{orismos}{Ορισμός}

\newenvironment{notation}
  {\begin{shaded}\begin{theorem}}
  {\end{theorem}\end{shaded}}




% Document begins
\begin{document}
\pagestyle{fancy}
\fancyhead{}
\fancyfoot{}
\renewcommand{\headrulewidth}{0pt}
\renewcommand{\footrulewidth}{0pt}


\fancyhead[LO,LE]{
 %\textblockcolor{backsite}
 %\begin{textblock}{5}(-2,-0.55)
  %\rule{0cm}{1cm}
 %\end{textblock}
 \textblockcolor{bluesite}
 \begin{textblock}{5}(-1.5,-0.55)
  \rule{0cm}{1cm}
 \end{textblock}
 %\textblockcolor{bluesite}
 %\begin{textblock}{14}(5.5,-0.55)
 % \rule{0cm}{1cm}
 %\end{textblock}
 \begin{textblock}{0}(-1,-0.25)
 \color{cwhite} \begin{Large}www.math24.gr\end{Large}
 \end{textblock}
\begin{textblock}{0}(16,-0.4)
 \color{cwhite} \includegraphics[height=1.5cm]{math24_logo.png}
 \end{textblock}
\textblockcolor{white}
\begin{textblock}{17}(-1,28)
 \color{backsite} \begin{small}Copyright \textcopyright 2011, Κασωτάκη Ε.(ikasotaki@gmail.com) - Σμαραγδάκης Κ.(kesmarag@gmail.com)\end{small}
 \end{textblock}
}
 
\begin{shaded}
\begin{center}
\huge \textbf{Φυλλάδιο Ασκήσεων}\\
%Πρόσθεση ρητών αριθμών
\end{center} 
\textbf{Μαθηματικά Α' Γυμνασίου} \hfill \textbf{Ημερομηνία Παράδοσης : \hspace{2em} }
\subsection*{Ονοματεπώνυμο :\hfill  \hspace{5em}}
\end{shaded}
\vspace{2em}
\begin{itemize}
 \item Απαλοιφή παρενθέσεων
 \item Μεταβολή ενός μεγέθους
 \item Αφαίρεση αριθμών
\end{itemize}
%-------------------------------------------------------------------------
\section*{Θεωρία - Απαλοιφή παρενθέσεων\hfill \small{}}
\begin{itemize}
 \item Όταν μία παρένθεση έχει μπροστά της το πρόσημο "\textbf{+}" (ή δεν έχει πρόσημο), μπορούμε να την 
       απαλείψουμε μαζί με το "+" (αν έχει) και να γράψουμε τους όρους που περιέχει με τα πρόσημά τους.\\
       \textbf{π.χ} $+(5+7-3-2+1)=5+7-3-2+1$ \\
       \textbf{π.χ} $(7-10-4-11+13)=7-10-4-11+13$ \\
 \item Όταν μία παρένθεση έχει μπροστά της το πρόσημο "\textbf{-}" , μπορούμε να την 
       απαλείψουμε μαζί με το "-" και να γράψουμε τους όρους που περιέχει με αντίθετα πρόσημά τους.\\
       \textbf{π.χ} $-(5+7-3-2+1)=-5-7+3+2-1$ \\
       \textbf{π.χ} $-(7-10-4-11+13)=-7+10+4+11-13$ \\      
\end{itemize}


\section*{Άσκηση 1  \hfill \small{30 μονάδες}}
Να επιλέξετε τη σωστή απάντηση:
\begin{enumerate}[1)]
 \item $+(3+5-2-1)$    %to 1o
\begin{multicols}{4}
\begin{enumerate}[i)]
 \item $3+5-2-1$
 \item $3-5+2+1$
 \item $-3-5+2+1$
 \item $-3+5+2+1$
\end{enumerate}
\end{multicols}
 \item $-(17-13+15)$    %to 2o
\begin{multicols}{4}
\begin{enumerate}[i)]
 \item $-17-13-15$
 \item $-17+13-15$
 \item $-17-13+15$
 \item $17-13+15$
\end{enumerate}
\end{multicols}
 \item $-(5+2)+(4+1)$    %to 3o
\begin{multicols}{4}
\begin{enumerate}[i)]
 \item $-5-2-4-1$
 \item $-5-2-4+1$
 \item $5+2+4+1$
 \item $-5-2+4+1$
\end{enumerate}
\end{multicols}
 \item $(5+1)-(17-13)$    %to 4o
\begin{multicols}{4}
\begin{enumerate}[i)]
 \item $5+1+17+13$
 \item $-5-1-17-13$
 \item $5+1-17+13$
 \item $5+1-17-13$
\end{enumerate}
\end{multicols}
 \item $-(5+1)-(-7-3)$    %to 5o
\begin{multicols}{4}
\begin{enumerate}[i)]
 \item $-5-1+7+3$
 \item $-5-1-7-3$
 \item $-5+1-7-3$
 \item $5+1+7+3$
\end{enumerate}
\end{multicols}
\end{enumerate}

\section*{Θεωρία - Μεταβολή ενός μεγέθους\hfill \small{}}
Για να βρούμε πόσο μεταβλήθηκε ένα μέγεθος, αφαιρούμε από την τελική τιμή του μεγέθους την αρχική τιμή.\\
Δηλαδή Μεταβολή=Τελική τιμή - Αρχική τιμή
\begin{itemize}
 \item \textbf{Αν} η μεταβολή είναι θετικός αριθμός  \textbf{τότε} το μέγεθος αυξήθηκε.
 \item \textbf{Αν} η μεταβολή είναι αρνητικός αριθμός  \textbf{τότε} το μέγεθος μειώθηκε.
 \item \textbf{Αν} η μεταβολή ισούται με μηδέν  \textbf{τότε} το μέγεθος δε μεταβλήθηκε.\\ \\
       \textbf{Παράδειγμα}\\ Ο Κώστας στην αρχή του καλοκαιριού είχε ύψος 150cm και στο τέλος του 
                    καλοκαιριού 152cm. Πόσο μεταβλήθηκε το ύψος του;\\
       Μεταβολή του ύψους=Τελική τιμή - Αρχική τιμή\\
       Μεταβολή του ύψους=152cm - 150cm\\
       Μεταβολή του ύψους=2cm\\
       Άρα το ύψος του Κώστα αυξήθηκε κατά 2cm.\\ \\
%-----------------------------------------------------------------------------
        \textbf{Παράδειγμα}\\ Ένα χειμωνιάτικο πρωί η θερμοκρασία σε μία πόλη ήταν -1\textdegree C και το  
                              μεσημέρι ήταν 5\textdegree C. Πόσο μεταβλήθηκε η θερμοκρασία;\\
       Μεταβολή της θερμοκρασίας=Τελική τιμή - Αρχική τιμή\\
       Μεταβολή της θερμοκρασίας=5\textdegree C - (-1\textdegree C)\\
       Μεταβολή της θερμοκρασίας=5\textdegree C +1\textdegree C\\
       Μεταβολή της θερμοκρασίας=+6\textdegree C\\
       Άρα η θερμοκρασία αυξήθηκε κατά 6\textdegree C. \\ \\
%------------------------------------------------------------------------------------
\textbf{Παράδειγμα}\\ Η Ειρήνη έφυγε από το σπίτι με 10\texteuro και πήγε στο περίπτερο να αγοράσει παγωτά και 
                    επέστρεψε στο σπίτι με 4\texteuro. Πόσο μεταβλήθηκαν τα χρήματά της;\\
       Μεταβολή χρημάτων=Τελική τιμή - Αρχική τιμή\\
       Μεταβολή χρημάτων=4\texteuro - 10\texteuro\\
       Μεταβολή χρημάτων=-6\texteuro\\
       Άρα τα χρήματα της Ειρήνης μειώθηκαν κατά 6\texteuro.
\end{itemize}

\section*{Άσκηση 2  \hfill \small{30 μονάδες}}
Η Ρηνιώ την πρώτη μέρα του κρυολογήματός της είχε (πυρετό) θερμοκρασία 39\textdegree C και τη δεύτερη μέρα είχε 37\textdegree C. 
Πόσο μεταβλήθηκε η θερμοκρασία της;



\section*{Άσκηση 3 \hfill \small{40 μονάδες}}
Να συμπληρώσετε τον παρακάτω πίνακα
\begin{center}
 \begin{tabular}{|c|c|c|c|}\hline 
   \textbf{α}    & \textbf{β} &         \textbf{α+β}  & \textbf{α-β}            \\
\hline 
   +3& +10 &     &           \\
\hline 
  +5 & -17 &     &           \\
\hline 
+3&        & +20 &           \\
\hline 
-10&   &0  &    \\
\hline
\end{tabular}
\end{center}






\end{document}
