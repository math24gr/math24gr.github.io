
\documentclass[a4paper,10pt]{report}

%\usepackage[landscape]{geometry}
\usepackage[cm-default]{fontspec}
\setromanfont{FreeSerif}
\setsansfont{FreeSans}
\setmonofont{FreeMono}
\usepackage[utf8x]{inputenc}
\usepackage{fontspec}
\usepackage{xunicode}
\usepackage{xltxtra}
\usepackage{xgreek}
\usepackage{amsmath}
\usepackage{unicode-math}
\usepackage{ulem}
\usepackage{color}
\usepackage{verbatim}
\usepackage{nopageno}
\usepackage{graphicx}
\usepackage{textpos}
\setlength{\TPHorizModule}{1cm}
\setlength{\TPVertModule}{1cm}
%\usepackage[colorgrid,texcoord]{eso-pic}
\usepackage[outline]{contour}
\usepackage{wrapfig}
\usepackage{url}
\usepackage{color}
\usepackage{tikz}
\usepackage{fancybox,fancyhdr}
\usepackage{subfigure}
\usepackage{pstricks}
\usepackage{epsfig}
\usepackage{multicol}
\usepackage{listings}
\usepackage{enumerate}
\usepackage{hyperref}
\hypersetup{
  bookmarks=true,
  bookmarksopen=true,
  pdfborder=false,
  pdfpagemode=UseNone,
  raiselinks=true,
  pdfhighlight={/P},
  colorlinks,
  citecolor=black,
  filecolor=black,
  linkcolor=black,
  urlcolor=black
}


\usepackage{fontspec}
\usepackage{xunicode}
\usepackage{xltxtra}


% Margins
%\setlength{\textwidth}{24cm}

%\setlength{\voffset}{0in}
%\setlength{\textheight}{6.8in}
% Colors
\definecolor{cbrown}{rgb}{0.49,0.24,0.07}
\definecolor{cmpez}{rgb}{0.92,0.88,0.79}
\definecolor{cmpez2}{rgb}{0.62,0.33,0.34}
\definecolor{cwhite}{rgb}{1,1,1}
\definecolor{cblack}{rgb}{0,0,0}
\definecolor{cred}{rgb}{0.9,0.15,0.15}
\definecolor{clblue}{rgb}{0.57,0.85,0.97}
\definecolor{corange}{rgb}{0.86,0.49,0.18}
\definecolor{clorange}{rgb}{0.95,0.84,0.61}
\definecolor{cyellow}{rgb}{1,0.95,0.16}
\definecolor{cgreen}{rgb}{0.69,0.8,0.31}
\definecolor{clbrown}{rgb}{0.78,0.67,0.43}
\definecolor{cmagenta}{rgb}{0.79,0.05,0.54}
\definecolor{cgray}{rgb}{0.76,0.73,0.63}
\definecolor{bluesite}{rgb}{0.05,0.43,0.67}
\definecolor{redsite}{rgb}{0.90,0.15,0.15}
\definecolor{backsite}{rgb}{0.07,0.07,0.07}
% ----------- from Costas -----------------------------

%---------------   Main Fonts    -----------------%   

% \setmainfont[Mapping=tex-text]{Calibri}
% \setmainfont[Mapping=tex-text]{Times New Roman}
%\setmainfont[Mapping=tex-text]{Droid Serif}
%\setmainfont[Mapping=tex-text]{Cambria}
%\setmainfont[Mapping=tex-text]{cm-unicode}
% \setmainfont[Mapping=tex-text]{Gentium}
% \setmainfont[Mapping=tex-text]{GFS Didot}
% \setmainfont[Mapping=tex-text]{Comic Sans MS}
 \setmainfont[Mapping=tex-text]{Ubuntu}
% \setmainfont[Mapping=tex-text]{Myriad Pro}
% \setmainfont[Mapping=tex-text]{CMU Concrete}
%\setmainfont[Mapping=tex-text]{DejaVu Sans}
%\setmainfont[Mapping=tex-text]{KerkisSans}
%\setmainfont[Mapping=tex-text]{KerkisCaligraphic}
% \setmainfont[Mapping=tex-text]{Segoe Print}
% \setmainfont[Mapping=tex-text]{Gabriola}

%------------------------------------------------%

% ------------ Mathematics Fonts ----------------%
\setmathfont{Asana-Math.ttf}
% -----------------------------------------------%
% Misc
\author{Κασωτάκη Ε. - Σμαραγδάκης Κ.}
\title{www.math24.gr}
% ---------------- Formation --------------------%


\setlength\topmargin{-0.7cm}
\addtolength\topmargin{-\headheight}
\addtolength\topmargin{-\headsep}
\setlength\textheight{26cm}
\setlength\oddsidemargin{-0.54cm}
\setlength\evensidemargin{-0.54cm}
%\setlength\marginparwidth{1.5in}
\setlength\textwidth{17cm}

\RequirePackage[avantgarde]{quotchap}
\renewcommand\chapterheadstartvskip{\vspace*{0\baselineskip}}
\RequirePackage[calcwidth]{titlesec}
\titleformat{\section}[hang]{\bfseries}
{\Large\thesection}{12pt}{\Large}[{\titlerule[0.9pt]}]
%--------------------------------------------------%
\usepackage{draftwatermark}
\SetWatermarkText{www.math24.gr}
\SetWatermarkLightness{0.9}
\SetWatermarkScale{0.6}
%--------------------------------------------------%


\usepackage{xcolor}
\usepackage{amsthm}
\usepackage{framed}
\usepackage{parskip}

\colorlet{shadecolor}{bluesite!20}

\newtheorem*{orismos}{Ορισμός}

\newenvironment{notation}
  {\begin{shaded}\begin{theorem}}
  {\end{theorem}\end{shaded}}




% Document begins
\begin{document}
\pagestyle{fancy}
\fancyhead{}
\fancyfoot{}
\renewcommand{\headrulewidth}{0pt}
\renewcommand{\footrulewidth}{0pt}


\fancyhead[LO,LE]{
 %\textblockcolor{backsite}
 %\begin{textblock}{5}(-2,-0.55)
  %\rule{0cm}{1cm}
 %\end{textblock}
 \textblockcolor{bluesite}
 \begin{textblock}{5}(-1.5,-0.55)
  \rule{0cm}{1cm}
 \end{textblock}
 %\textblockcolor{bluesite}
 %\begin{textblock}{14}(5.5,-0.55)
 % \rule{0cm}{1cm}
 %\end{textblock}
 \begin{textblock}{0}(-1,-0.25)
 \color{cwhite} \begin{Large}www.math24.gr\end{Large}
 \end{textblock}
\begin{textblock}{0}(16,-0.4)
 \color{cwhite} \includegraphics[height=1.5cm]{math24_logo.png}
 \end{textblock}
\textblockcolor{white}
\begin{textblock}{17}(-1,28)
 \color{backsite} \begin{small}Copyright \textcopyright 2012, Κασωτάκη Ε.(ikasotaki@gmail.com) - Σμαραγδάκης Κ.(kesmarag@gmail.com)\end{small}
 \end{textblock}
}
 
\begin{shaded}
\begin{center}
\huge \textbf{Φυλλάδιο Ασκήσεων}\\
%Πρόσθεση ρητών αριθμών
\end{center} 
\textbf{Μαθηματικά Α' Γυμνασίου} \hfill \textbf{Ημερομηνία Παράδοσης : \hspace{2em} }
\subsection*{Ονοματεπώνυμο :\hfill  \hspace{5em}}
\end{shaded}
\vspace{2em}
\begin{itemize}
 \item Κριτήρια διαιρετότητας
 \item Ανάλυση ενός αριθμού σε γινόμενο πρώτων παραγόντων
 \item Εύρεση $ΕΚΠ$ και $ΜΚΔ$ με χρήση της ανάλυσης σε γινόμενο πρώτων παραγόντων
\end{itemize}

\section*{Θεωρία - Κριτήρια Διαιρετότητας \hfill \small{}}
\begin{itemize}
 \item Ένας φυσικός αριθμός διαιρείται με $10,100,1000,\cdots$ αν λήγει σε ένα, δύο, τρία, $\cdots$ μηδενικά 
       αντίστοιχα.
 \item Ένας φυσικός αριθμός διαιρείται με το $2$ αν το τελευταίο ψηφίο είναι $0,2,4,6,\cdots$
 \item Ένας φυσικός αριθμός διαιρείται με το $3$ αν το άθροισμα των ψηφίων του διαιρείται με το 3.
 \item Ένας φυσικός αριθμός διαιρείται με το $4$ αν τα δύο τελευταία ψηφία του σχηματίζουν αριθμό 
       που διαιρείται με το $4$.
 \item Ένας φυσικός αριθμός διαιρείται με το $5$ αν λήγει σε $0$ ή $5$.
 \item Ένας φυσικός αριθμός διαιρείται με το $9$ αν το άθροισμα των ψηφίων του διαιρείται με το $9$.
 \item Ένας φυσικός αριθμός διαιρείται με το $25$ αν τα δύο τελευταία του ψηφία σχηματίζουν αριθμό που διαιρείται
       με το $25$.
\end{itemize}
\textbf{Παράδειγμα}\\
το $450$ διαιρείται με το:
\begin{itemize}
 \item $2$ γιατί λήγει σε $0$
 \item $3$ γιατί το άθροισμα των ψηφίων του είναι $4+5+0=9$ και διαιρείται με το $3$
 \item $5$ γιατί λήγει σε $0$
 \item $9$ γιατί το άθροισμα των ψηφίων του είναι $4+5+0=9$ και διαιρείται με το $9$
 \item $10$ γιατί λήγει σε $0$
 \item $25$ γιατί τα δύο τελευταία ψηφία σχηματίζουν τον αριθμό $50$ που διαιρείται με το $25$
\end{itemize}





\section*{Άσκηση 1  \hfill \small{25 μονάδες}}
Να συμπληρώσετε τον παρακάτω πίνακα χρησιμοποιώντας "\textbf{ΝΑΙ}" ή \textbf{'ΌΧΙ}".
\begin{center}
 \begin{tabular}{|c|c|c|c|c|c|}
\hline                   
        & \textbf{Διαιρείται με το 2}    & \textbf{Διαιρείται με το 3} & \textbf{Διαιρείται με το 4} &  \textbf{Διαιρείται με το 5}&\textbf{Διαιρείται με το 10}   \\
\hline 
132     &      &      &      &     &              \\
\hline
245     &      &      &      &     &     \\
\hline
460     &      &      &      &     &   \\
\hline
510     &      &     &       &     &  \\
\hline
729     &      &     &       &     &  \\
\hline
\end{tabular}
\end{center}





\section*{Άσκηση 2  \hfill \small{25 μονάδες}}
Να βρείτε αν οι παρακάτω αριθμοί διαιρούνται με το 2, το 3 και το 5 και να αιτιολογήσετε την απάντησή σας.
\begin{enumerate}[i)]
 \item $1072$
 \item $2810$
 \item $1785$
 \item $3335$
 \item $9990$
\end{enumerate}

\section*{Θεωρία  \hfill \small{}}
\textbf{Εύρεση ΕΚΠ και ΜΚΔ με χρήση της ανάλυσης σε γινόμενο πρώτων παραγόντων}
\begin{itemize}
 \item Αναλύουμε τους αριθμούς σε γινόμενα πρώτων παραγόντων
 \item Για να βρούμε το ΕΚΠ των αριθμών παίρνουμε τους κοινούς και τους μη κοινούς παράγοντες 
       με το μεγαλύτερο εκθέτη
 \item Για να βρούμε το ΜΚΔ παίρνουμε τους κοινούς παράγοντες με το μικρότερο εκθέτη
\end{itemize}
\textbf{Παράδειγμα}\\
Εύρεση $ΕΚΠ$ και $ΜΚΔ$ με χρήση της ανάλυσης σε γινόμενο παραγόντων για τους αριθμούς $15$ και $27$
\begin{itemize}
 \item $15=3\cdot 5$ και $27=3^{3}$
 \item $ΕΚΠ(15,27)=3^{3}\cdot 5=27\cdot 5=135$
 \item $ΜΚΔ(15,27)=3$
\end{itemize}




\section*{Άσκηση 3  \hfill \small{}25  μονάδες}
Να αναλύσετε τους παρακάτω αριθμούς σε γινόμενο πρώτων παραγόντων:
\begin{enumerate}[i)]
 \item $315$
 \item $336$
 \item $4080$
 \item $4725$
 \item $11880$
\end{enumerate}





\section*{Άσκηση 4  \hfill \small{} 25 μονάδες}
Να αναλύσετε τους αριθμούς $2520$ και $3500$ σε γινόμενο πρώτων παραγόντων. 
Με τη χρήση της ανάλυσης αυτής να βρείτε το $ΕΚΠ$ και το $ΜΚΔ$ τους.




%\let\thefootnote\relax\footnotetext{Επιμέλεια : Κασωτάκη Ειρήνη (ikasotaki@gmail.com),    Σμαραγδάκης Κώστας (kesmarag@gmail.com)} 


\end{document}
