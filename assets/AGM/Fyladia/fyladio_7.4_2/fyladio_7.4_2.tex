
\documentclass[a4paper,10pt]{report}

%\usepackage[landscape]{geometry}
\usepackage[cm-default]{fontspec}
\setromanfont{FreeSerif}
\setsansfont{FreeSans}
\setmonofont{FreeMono}
\usepackage[utf8x]{inputenc}
\usepackage{fontspec}
\usepackage{xunicode}
\usepackage{xltxtra}
\usepackage{xgreek}
\usepackage{amsmath}
\usepackage{unicode-math}
\usepackage{ulem}
\usepackage{color}
\usepackage{verbatim}
\usepackage{nopageno}
\usepackage{graphicx}
\usepackage{textpos}
\setlength{\TPHorizModule}{1cm}
\setlength{\TPVertModule}{1cm}
%\usepackage[colorgrid,texcoord]{eso-pic}
\usepackage[outline]{contour}
\usepackage{wrapfig}
\usepackage{url}
\usepackage{color}
\usepackage{tikz}
\usepackage{fancybox,fancyhdr}
\usepackage{subfigure}
\usepackage{pstricks}
\usepackage{epsfig}
\usepackage{multicol}
\usepackage{listings}
\usepackage{enumerate}
\usepackage{hyperref}
\hypersetup{
  bookmarks=true,
  bookmarksopen=true,
  pdfborder=false,
  pdfpagemode=UseNone,
  raiselinks=true,
  pdfhighlight={/P},
  colorlinks,
  citecolor=black,
  filecolor=black,
  linkcolor=black,
  urlcolor=black
}


\usepackage{fontspec}
\usepackage{xunicode}
\usepackage{xltxtra}


% Margins
%\setlength{\textwidth}{24cm}

%\setlength{\voffset}{0in}
%\setlength{\textheight}{6.8in}
% Colors
\definecolor{cbrown}{rgb}{0.49,0.24,0.07}
\definecolor{cmpez}{rgb}{0.92,0.88,0.79}
\definecolor{cmpez2}{rgb}{0.62,0.33,0.34}
\definecolor{cwhite}{rgb}{1,1,1}
\definecolor{cblack}{rgb}{0,0,0}
\definecolor{cred}{rgb}{0.9,0.15,0.15}
\definecolor{clblue}{rgb}{0.57,0.85,0.97}
\definecolor{corange}{rgb}{0.86,0.49,0.18}
\definecolor{clorange}{rgb}{0.95,0.84,0.61}
\definecolor{cyellow}{rgb}{1,0.95,0.16}
\definecolor{cgreen}{rgb}{0.69,0.8,0.31}
\definecolor{clbrown}{rgb}{0.78,0.67,0.43}
\definecolor{cmagenta}{rgb}{0.79,0.05,0.54}
\definecolor{cgray}{rgb}{0.76,0.73,0.63}
\definecolor{bluesite}{rgb}{0.05,0.43,0.67}
\definecolor{redsite}{rgb}{0.90,0.15,0.15}
\definecolor{backsite}{rgb}{0.07,0.07,0.07}
% ----------- from Costas -----------------------------

%---------------   Main Fonts    -----------------%   

% \setmainfont[Mapping=tex-text]{Calibri}
% \setmainfont[Mapping=tex-text]{Times New Roman}
%\setmainfont[Mapping=tex-text]{Droid Serif}
%\setmainfont[Mapping=tex-text]{Cambria}
%\setmainfont[Mapping=tex-text]{cm-unicode}
% \setmainfont[Mapping=tex-text]{Gentium}
% \setmainfont[Mapping=tex-text]{GFS Didot}
% \setmainfont[Mapping=tex-text]{Comic Sans MS}
 \setmainfont[Mapping=tex-text]{Ubuntu}
% \setmainfont[Mapping=tex-text]{Myriad Pro}
% \setmainfont[Mapping=tex-text]{CMU Concrete}
%\setmainfont[Mapping=tex-text]{DejaVu Sans}
%\setmainfont[Mapping=tex-text]{KerkisSans}
%\setmainfont[Mapping=tex-text]{KerkisCaligraphic}
% \setmainfont[Mapping=tex-text]{Segoe Print}
% \setmainfont[Mapping=tex-text]{Gabriola}

%------------------------------------------------%

% ------------ Mathematics Fonts ----------------%
%\setmathfont{Asana-Math.ttf}
% -----------------------------------------------%
% Misc
\author{Κασωτάκη Ε. - Σμαραγδάκης Κ.}
\title{www.math24.gr}
% ---------------- Formation --------------------%


\setlength\topmargin{-0.7cm}
\addtolength\topmargin{-\headheight}
\addtolength\topmargin{-\headsep}
\setlength\textheight{26cm}
\setlength\oddsidemargin{-0.54cm}
\setlength\evensidemargin{-0.54cm}
%\setlength\marginparwidth{1.5in}
\setlength\textwidth{17cm}

\RequirePackage[avantgarde]{quotchap}
\renewcommand\chapterheadstartvskip{\vspace*{0\baselineskip}}
\RequirePackage[calcwidth]{titlesec}
\titleformat{\section}[hang]{\bfseries}
{\Large\thesection}{12pt}{\Large}[{\titlerule[0.9pt]}]
%--------------------------------------------------%
%\usepackage{draftwatermark}
%\SetWatermarkText{www.math24.gr}
%\SetWatermarkLightness{0.9}
%\SetWatermarkScale{0.6}
%--------------------------------------------------%


\usepackage{xcolor}
\usepackage{amsthm}
\usepackage{framed}
\usepackage{parskip}

\colorlet{shadecolor}{bluesite!20}

\newtheorem*{orismos}{Ορισμός}

\newenvironment{notation}
  {\begin{shaded}\begin{theorem}}
  {\end{theorem}\end{shaded}}




% Document begins
\begin{document}
\pagestyle{fancy}
\fancyhead{}
\fancyfoot{}
\renewcommand{\headrulewidth}{0pt}
\renewcommand{\footrulewidth}{0pt}


\fancyhead[LO,LE]{
 %\textblockcolor{backsite}
 %\begin{textblock}{5}(-2,-0.55)
  %\rule{0cm}{1cm}
 %\end{textblock}
 \textblockcolor{bluesite}
 \begin{textblock}{5}(-1.5,-0.55)
  \rule{0cm}{1cm}
 \end{textblock}
 %\textblockcolor{bluesite}
 %\begin{textblock}{14}(5.5,-0.55)
 % \rule{0cm}{1cm}
 %\end{textblock}
 \begin{textblock}{0}(-1,-0.25)
 \color{cwhite} \begin{Large}www.math24.gr\end{Large}
 \end{textblock}
\begin{textblock}{0}(16,-0.4)
 \color{cwhite} \includegraphics[height=1.5cm]{math24_logo.png}
 \end{textblock}
\textblockcolor{white}
\begin{textblock}{17}(-1,28)
 \color{backsite} \begin{small}Επιμέλεια: Κασωτάκη Ειρήνη (ikasotaki@gmail.com) - Σχολικό έτος 2019 - 2020\end{small}
 \end{textblock}
}
 
\begin{shaded}
\begin{center}
\huge \textbf{Φυλλάδιο Ασκήσεων}\\
%Πρόσθεση ρητών αριθμών
\end{center} 
\textbf{Μαθηματικά Α' Γυμνασίου} \hfill \textbf{Ημερομηνία Παράδοσης : \hspace{2em} }
\subsection*{Ονοματεπώνυμο :\hfill  \hspace{5em}}
\end{shaded}
\vspace{2em}
\begin{itemize}
 \item Αφαίρεση ρητών
 \item Απαλοιφή παρενθέσεων
\end{itemize}
%-------------------------------------------------------------------------
\section*{Αφαίρεση Ρητών - Απαλοιφή παρενθέσεων\hfill \small{}}
Αφαίρεση ρητών:
\begin{itemize}
\item Για να αφαιρέσουμε από έναν αριθμό $α$ έναν αριθμό $β$, προσθέτουμε στον $α$ τον αντίθετο 
του $β$.\\
$α-β=α+(-β)$\\
Δηλαδή, στους ρητούς η αφαίρεση μετατρέπεται σε πρόσθεση.
\end{itemize}
Απαλοιφή παρενθέσεων:
\begin{itemize}
 \item Όταν μία παρένθεση έχει μπροστά της το πρόσημο "\textbf{+}" (ή δεν έχει πρόσημο), μπορούμε να την 
       απαλείψουμε μαζί με το "+" (αν έχει) και να γράψουμε τους όρους που περιέχει με τα πρόσημά τους.\
 \item Όταν μία παρένθεση έχει μπροστά της το πρόσημο "\textbf{-}" , μπορούμε να την 
       απαλείψουμε μαζί με το "-" και να γράψουμε τους όρους που περιέχει με αντίθετα πρόσημά τους.\\     
\end{itemize}

\textbf{Παράδειγμα 1} Υπολογισμός της διαφοράς $(+3)-(-5)$

\textbf{α' τρόπος μετστρέποντας την αφαίρεση σε πρόσθεση:}\\
$(+3)-(-5)=(+3)+(+5)=+8$\\
\textbf{β' τρόπος απαλοίφοντας τις παρενθέσεις:}\\
$(+3)-(-5)= +3+5=+8$\\

\textbf{Παράδειγμα 2} Υπολογισμός της παράστασης $Α=(+3)-(+2)+(-2)+(+4)-(-1)$

\textbf{α' τρόπος μετστρέποντας την αφαίρεση σε πρόσθεση:}\\
$Α=(+3)-(+2)+(-2)+(+4)-(-1)=(+3)+(-2)+(-2)+(+4)+(+1)=(-7)+(+5)=-2$\\
\textbf{β' τρόπος απαλοίφοντας τις παρενθέσεις:}\\
$Α=(+3)-(+2)+(-2)+(+4)-(-1)=-3-2-2+4+1=-7+5=-2$


\section*{Άσκηση 1  \hfill \small{}}%παρόμοια με άσκηση 2 από το βιβλίο
Να υπολογίσετε τις παρακάτω διαφορές:
\begin{enumerate}[i)]
\item $(+4)-(-5)$
\item $(-3)-(-4)$
\item $(-2)-(+7)$
\item $(+3)-(+5)$
\item $(-7)-(+4)$
\end{enumerate}


\section*{Άσκηση 2  \hfill \small{}}%παρόμοια με άσκηση 2 από το βιβλίο
Να υπολογίσετε τις παρακάτω διαφορές:
\begin{enumerate}[i)]
\item $(-\dfrac{3}{4})-(\dfrac{1}{4})$
\item $(-\dfrac{2}{3})-(+\dfrac{4}{3})$
\item $(-\dfrac{2}{3})-(-\dfrac{1}{6})$
\item $(+\dfrac{2}{3})-(+\dfrac{5}{4})$
\item $(+\dfrac{1}{2})-(+\dfrac{4}{3})$
\end{enumerate}


\section*{Άσκηση 3  \hfill \small{}}%παρόμοια με άσκηση 2 από το βιβλίο
Να υπολογίσετε τις παρακάτω διαφορές:
\begin{enumerate}[i)]
\item $(+5,25)-(-1,13)$
\item $(-7,55)-(-2.23)$
\item $(+15,35)-(+5,35)$
\item $(-13,27)-(-3,27)$
\item $(-4)-(-5.3)$
\end{enumerate}


\section*{Άσκηση 4  \hfill \small{}}%παρόμοια με άσκηση 2 από το βιβλίο
Να κάνετε τις παρακάτω πράξεις:
\begin{enumerate}[i)]
\item $4-(-8)$
\item $-7-(+7)$
\item $-2-(-1)$
\item $5-(-17)$
\item $10-(+2)$
\end{enumerate}


\section*{Άσκηση 5  \hfill \small{}}%παρόμοια με άσκηση 2 από το βιβλίο
Να κάνετε τις παρακάτω πράξεις:
\begin{enumerate}[i)]
\item $-\dfrac{4}{7}-(-\dfrac{3}{7})$
\item $-\dfrac{4}{7}-(+\dfrac{3}{7})$
\item $-\dfrac{2}{5}-(-\dfrac{1}{3})$
\item $-\dfrac{2}{15}-(+\dfrac{3}{5})$
\item $\dfrac{3}{8}-(-\dfrac{1}{4})$
\end{enumerate}


\section*{Άσκηση 6  \hfill \small{}}%παρόμοια με άσκηση 2 από το βιβλίο
Να κάνετε τις παρακάτω πράξεις:
\begin{enumerate}[i)]
\item $17,83-(+1,22)$
\item $14,21-(-3,24)$
\item $-5,2-(-3,1)$
\item $-4,7-(+3,3)$
\item $5-(-2,75)$
\end{enumerate}


\section*{Άσκηση 7  \hfill \small{}}%παρόμοια με άσκηση 4 από το βιβλίο
Να κάνετε τις παρακάτω πράξεις:
\begin{enumerate}[i)]
\item $(+4)-(+3)+(+7)$
\item $(-4)+(-5)-(-10)$
\item $(+5)+(+3)-(+4)$
\item $(+3)-(-4)-(-10)$
\item $(-10)-(-20)+(-30)$
\end{enumerate}

\section*{Άσκηση 8  \hfill \small{}}%παρόμοια με άσκηση 4 από το βιβλίο
Να κάνετε τις παρακάτω πράξεις:
\begin{enumerate}[i)]
\item $(+5)+(+3)+(-4)-(-5)-(+3)$
\item $(+2)-(-5)-(+7)-(-3)+(-5)$
\item $(+5)-(+3)-(-4)-(+4)+(+5)$
\item $(-4)-(-5)-(+3)+(-2)+(+5)$
\item $(-2)+(+4)+(-5)-(-4)-(+2)$
\end{enumerate}

\section*{Άσκηση 9  \hfill \small{}}%παρόμοια με άσκηση 4 από το βιβλίο
Να κάνετε τις παρακάτω πράξεις:
\begin{enumerate}[i)]
\item $(-\dfrac{2}{5})-(-\dfrac{1}{5})+(+\dfrac{4}{5})$
\item $(+\dfrac{3}{7})-(+\dfrac{2}{7})+(-\dfrac{4}{7})$
\item $(-\dfrac{1}{2})-(+\dfrac{3}{2})-(-\dfrac{1}{2})$
\item $(+\dfrac{2}{3})-(-\dfrac{1}{2})+(+\dfrac{4}{3})$
\item $(1\dfrac{2}{3})-(-\dfrac{3}{2})+(+\dfrac{5}{6})$
\end{enumerate}

\section*{Άσκηση 10  \hfill \small{}}%παρόμοια με άσκηση 4 από το βιβλίο
Να κάνετε τις παρακάτω πράξεις:
\begin{enumerate}[i)]
\item $(+5,75)-(-1,11)+(-2,3)$
\item $(+5,5)-(+2,5)+(-4,5)$
\item $(-3,8)-(-2,5)-(+1,5)$
\item $(+4)-(-1,5)-(+3,5)$
\item $(+7,25)-(+2,5)-(-20.5)$
\end{enumerate}

\end{document}
