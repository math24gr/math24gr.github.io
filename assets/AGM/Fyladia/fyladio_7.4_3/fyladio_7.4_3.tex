
\documentclass[a4paper,10pt]{report}

%\usepackage[landscape]{geometry}
\usepackage[cm-default]{fontspec}
\setromanfont{FreeSerif}
\setsansfont{FreeSans}
\setmonofont{FreeMono}
\usepackage[utf8x]{inputenc}
\usepackage{fontspec}
\usepackage{xunicode}
\usepackage{xltxtra}
\usepackage{xgreek}
\usepackage{amsmath}
\usepackage{unicode-math}
\usepackage{ulem}
\usepackage{color}
\usepackage{verbatim}
\usepackage{nopageno}
\usepackage{graphicx}
\usepackage{textpos}
\setlength{\TPHorizModule}{1cm}
\setlength{\TPVertModule}{1cm}
%\usepackage[colorgrid,texcoord]{eso-pic}
\usepackage[outline]{contour}
\usepackage{wrapfig}
\usepackage{url}
\usepackage{color}
\usepackage{tikz}
\usepackage{fancybox,fancyhdr}
\usepackage{subfigure}
\usepackage{pstricks}
\usepackage{epsfig}
\usepackage{multicol}
\usepackage{listings}
\usepackage{enumerate}
\usepackage{hyperref}
\hypersetup{
  bookmarks=true,
  bookmarksopen=true,
  pdfborder=false,
  pdfpagemode=UseNone,
  raiselinks=true,
  pdfhighlight={/P},
  colorlinks,
  citecolor=black,
  filecolor=black,
  linkcolor=black,
  urlcolor=black
}


\usepackage{fontspec}
\usepackage{xunicode}
\usepackage{xltxtra}


% Margins
%\setlength{\textwidth}{24cm}

%\setlength{\voffset}{0in}
%\setlength{\textheight}{6.8in}
% Colors
\definecolor{cbrown}{rgb}{0.49,0.24,0.07}
\definecolor{cmpez}{rgb}{0.92,0.88,0.79}
\definecolor{cmpez2}{rgb}{0.62,0.33,0.34}
\definecolor{cwhite}{rgb}{1,1,1}
\definecolor{cblack}{rgb}{0,0,0}
\definecolor{cred}{rgb}{0.9,0.15,0.15}
\definecolor{clblue}{rgb}{0.57,0.85,0.97}
\definecolor{corange}{rgb}{0.86,0.49,0.18}
\definecolor{clorange}{rgb}{0.95,0.84,0.61}
\definecolor{cyellow}{rgb}{1,0.95,0.16}
\definecolor{cgreen}{rgb}{0.69,0.8,0.31}
\definecolor{clbrown}{rgb}{0.78,0.67,0.43}
\definecolor{cmagenta}{rgb}{0.79,0.05,0.54}
\definecolor{cgray}{rgb}{0.76,0.73,0.63}
\definecolor{bluesite}{rgb}{0.05,0.43,0.67}
\definecolor{redsite}{rgb}{0.90,0.15,0.15}
\definecolor{backsite}{rgb}{0.07,0.07,0.07}
% ----------- from Costas -----------------------------

%---------------   Main Fonts    -----------------%   

% \setmainfont[Mapping=tex-text]{Calibri}
% \setmainfont[Mapping=tex-text]{Times New Roman}
%\setmainfont[Mapping=tex-text]{Droid Serif}
%\setmainfont[Mapping=tex-text]{Cambria}
%\setmainfont[Mapping=tex-text]{cm-unicode}
% \setmainfont[Mapping=tex-text]{Gentium}
% \setmainfont[Mapping=tex-text]{GFS Didot}
% \setmainfont[Mapping=tex-text]{Comic Sans MS}
 \setmainfont[Mapping=tex-text]{Ubuntu}
% \setmainfont[Mapping=tex-text]{Myriad Pro}
% \setmainfont[Mapping=tex-text]{CMU Concrete}
%\setmainfont[Mapping=tex-text]{DejaVu Sans}
%\setmainfont[Mapping=tex-text]{KerkisSans}
%\setmainfont[Mapping=tex-text]{KerkisCaligraphic}
% \setmainfont[Mapping=tex-text]{Segoe Print}
% \setmainfont[Mapping=tex-text]{Gabriola}

%------------------------------------------------%

% ------------ Mathematics Fonts ----------------%
%\setmathfont{Asana-Math.ttf}
% -----------------------------------------------%
% Misc
\author{Κασωτάκη Ε. - Σμαραγδάκης Κ.}
\title{www.math24.gr}
% ---------------- Formation --------------------%


\setlength\topmargin{-0.7cm}
\addtolength\topmargin{-\headheight}
\addtolength\topmargin{-\headsep}
\setlength\textheight{26cm}
\setlength\oddsidemargin{-0.54cm}
\setlength\evensidemargin{-0.54cm}
%\setlength\marginparwidth{1.5in}
\setlength\textwidth{17cm}

\RequirePackage[avantgarde]{quotchap}
\renewcommand\chapterheadstartvskip{\vspace*{0\baselineskip}}
\RequirePackage[calcwidth]{titlesec}
\titleformat{\section}[hang]{\bfseries}
{\Large\thesection}{12pt}{\Large}[{\titlerule[0.9pt]}]
%--------------------------------------------------%
%\usepackage{draftwatermark}
%\SetWatermarkText{www.math24.gr}
%\SetWatermarkLightness{0.9}
%\SetWatermarkScale{0.6}
%--------------------------------------------------%


\usepackage{xcolor}
\usepackage{amsthm}
\usepackage{framed}
\usepackage{parskip}

\colorlet{shadecolor}{bluesite!20}

\newtheorem*{orismos}{Ορισμός}

\newenvironment{notation}
  {\begin{shaded}\begin{theorem}}
  {\end{theorem}\end{shaded}}




% Document begins
\begin{document}
\pagestyle{fancy}
\fancyhead{}
\fancyfoot{}
\renewcommand{\headrulewidth}{0pt}
\renewcommand{\footrulewidth}{0pt}


\fancyhead[LO,LE]{
 %\textblockcolor{backsite}
 %\begin{textblock}{5}(-2,-0.55)
  %\rule{0cm}{1cm}
 %\end{textblock}
 \textblockcolor{bluesite}
 \begin{textblock}{5}(-1.5,-0.55)
  \rule{0cm}{1cm}
 \end{textblock}
 %\textblockcolor{bluesite}
 %\begin{textblock}{14}(5.5,-0.55)
 % \rule{0cm}{1cm}
 %\end{textblock}
 \begin{textblock}{0}(-1,-0.25)
 \color{cwhite} \begin{Large}www.math24.gr\end{Large}
 \end{textblock}
\begin{textblock}{0}(16,-0.4)
 \color{cwhite} \includegraphics[height=1.5cm]{math24_logo.png}
 \end{textblock}
\textblockcolor{white}
\begin{textblock}{17}(-1,28)
 \color{backsite} \begin{small}Επιμέλεια: Κασωτάκη Ειρήνη (ikasotaki@gmail.com) - Σχολικό έτος 2019 - 2020\end{small}
 \end{textblock}
}
 
\begin{shaded}
\begin{center}
\huge \textbf{Φυλλάδιο Ασκήσεων}\\
%Πρόσθεση ρητών αριθμών
\end{center} 
\textbf{Μαθηματικά Α' Γυμνασίου} \hfill \textbf{Ημερομηνία Παράδοσης : \hspace{2em} }
\subsection*{Ονοματεπώνυμο :\hfill  \hspace{5em}}
\end{shaded}
\vspace{2em}
\begin{itemize}
 \item Πρόσθεση και αφαίρεση ρητών
 \item Απαλοιφή παρενθέσεων
 \item Υπολογισμός αριθμητικών παραστάσεων
\end{itemize}
%-------------------------------------------------------------------------
\section*{Υπολογισμός τιμής αριθμητικής παράστασης\hfill \small{}}

\textbf{Παράδειγμα 1} \\
Υπολογισμός της τιμής της αριθμητικής παράστασης  $+3-3-2+5-7+6$

%https://www.tablesgenerator.com/#
\begin{table}[h]
\begin{tabular}{ll|l}
$+4-3-2-7+6$ & $= +4+5+6-3-2-7$ & χωρίσαμε τους θετικούς από τους αρνητικούς                       \\
             &                  &                                                                  \\
             & $= +15-12$       & προσθέσαμε χωριστά τους θετικούς και γράψαμε το αποτέλεσμα $+15$ \\
             &                  & και χωριστά τους αρνητικούς και γράψαμε το αποτέλεσμα $-12$      \\
             &                  &                                                                  \\
             & $= +3$           & προσθέσαμε τους 2 ετερόσημους αριθμούς                           \\
             &                  & και γράψαμε το αποτέλεσμα $+3$                                  
\end{tabular}
\end{table}


\textbf{Παράδειγμα 2} \\
Υπολογισμός της τιμής της αριθμητικής παράστασης  $-3+6-2-(+1-3+4-2-5)$ \\


\textbf{α' τρόπος κάνοντας πρώτα τις πράξεις που είναι μέσα στις παρενθέσεις:} 

\begin{table}[h]
\begin{tabular}{l|l}
$-3+6-2-(+1-3+4-2-5)$   & θα κάνουμε πρώτα τις πράξεις μέσα στην παρένθεση                  \\
                        &                                                                   \\
$= -3+6-2-(+1+4-3-2-5)$ & μέσα στην παρένθεση: χωρίσαμε τους θετικούς από τους αρνητικούς   \\
                        &                                                                   \\
$= -3+6-2-(+5-10)$      & μέσα στην παρένθεση: προσθέσαμε χωριστά τους θετικούς             \\
                        & και γράψαμε το αποτέλεσμα $+5$                                    \\
                        & και χωριστά τους αρνητικούς και γράψαμε το αποτέλεσμα $-10$       \\
                        &                                                                   \\
$= -3+6-2-(-5)$        & μέσα στην παρένθεση: προσθέσαμε τους 2 ετερόσημους αριθμούς       \\
                        & και γράψαμε το αποτέλεσμα $-5$                                   \\
                        &                                                                   \\
$= -3+6-2+5$             & κάναμε απαλοιφή παρένθεσης: επειδή η παρένθεση είχε μπροστά       \\
                        & της το πρόσημο "-" βγάλαμε την παρένθεση και γράψαμε              \\
                        & τον αριθμό που βρισκόταν μέσα στην παρένθεση με αντίθετο πρόσημο  \\
                        &                                                                   \\
$= -3-2+6+5$             & χωρίσαμε τους αρνητικούς από τους θετικούς                        \\
                        &                                                                   \\
$= -5+11$                 & προσθέσαμε χωριστά τους αρνητικούς και γράψαμε το αποτέλεσμα $-5$ \\
                        & και χωριστά τους θετικούς και γράψαμε το αποτέλεσμα $+11$         \\
                        &                                                                   \\
$= +6$                   & προσθέσαμε τους 2 ετερόσημους αριθμούς                            \\
                        & και γράψαμε το αποτέλεσμα $+6$                                  
\end{tabular}
\end{table}

\newpage

\textbf{β' τρόπος απαλοίφοντας τις παρενθέσεις:}\\

\begin{table}[h]
\begin{tabular}{l|l}
$-3+6-2-(+1-3+4-2-5)$ & πρώτα θα κάνουμε απαλοιφή της παρένθεσης                           \\
                      &                                                                    \\
$= -3+6-2-1+3-4+2+5$  & κάναμε απαλοιφή παρένθεσης: επειδή η παρένθεση είχε μπροστά της    \\
                      & το πρόσημο "-" βγάλαμε την παρένθεση και γράψαμε τους αριθμούς     \\
                      & που βρισκόταν μέσα στην παρένθεση με αντίθετα πρόσημα              \\
                      &                                                                    \\
$= -3-2-1-4+6+3+2+5$    & χωρίσαμε τους αρνητικούς από τους θετικούς                         \\
                      &                                                                    \\
$= -10+16$              & προσθέσαμε χωριστά τους αρνητικούς και γράψαμε το αποτέλεσμα $-10$ \\
                      & και χωριστά τους θετικούς και γράψαμε το αποτέλεσμα $+16$          \\
                      &                                                                    \\
$= +6$                  & προσθέσαμε τους 2 ετερόσημους αριθμούς                             \\
                      & και γράψαμε το αποτέλεσμα $+6$                                    
\end{tabular}
\end{table}


\textbf{Παράδειγμα 3} \\
Υπολογισμός της τιμής της αριθμητικής παράστασης  $3+4\cdot 5-2\cdot 3-1+2\cdot(-4)$

%https://www.tablesgenerator.com/#
\begin{table}[h]
\begin{tabular}{l|l}
$3+4\cdot5-2\cdot3-1+2\cdot(-4)$ & πρώτα θα υπολογίσουμε τα γινόμενα                                  \\
                                 &                                                                    \\
$= 3+20-6-1-8$                   & κάναμε απαλοιφή παρένθεσης: επειδή η παρένθεση είχε μπροστά της    \\
                                 &                                                                    \\
$-6-1-8+3+20$                    & χωρίσαμε τους αρνητικούς από τους θετικούς                         \\
                                 &                                                                    \\
$-15+23$                         & προσθέσαμε χωριστά τους αρνητικούς και γράψαμε το αποτέλεσμα $-15$ \\
                                 & και χωριστά τους θετικούς και γράψαμε το αποτέλεσμα $+23$          \\
                                 &                                                                    \\
$+8$                             & προσθέσαμε τους 2 ετερόσημους αριθμούς                             \\
                                 & και γράψαμε το αποτέλεσμα $+8$                                    
\end{tabular}
\end{table}

\section*{Άσκηση 1  \hfill \small{}}
Να υπολογίσετε τις τιμές των παρακάτω αριθμητικών παραστάσεων:
\begin{enumerate}[i)]
\item $+3+2-4+7-6-1$
\item $-3-4+1+5-8-2$
\item $-10+40+20-40-10+30$
\item $+15-25-20+30+10$
\item $+17-12+11+13-15-8$
\end{enumerate}


\section*{Άσκηση 2  \hfill \small{}}
Να υπολογίσετε τις τιμές των παρακάτω αριθμητικών παραστάσεων:
\begin{enumerate}[i)]
\item $+1-2+3-(-3+4-1+2)$
\item $+2+4-1-3+(4+1-2-5+1)$
\item $+2-1+(1+4-2-3-5)+3$
\item $-4-2-(-3-2+1+8)-2$
\item $-3+(+2+1-4-5+3)-2+7+1$
\end{enumerate}

\section*{Άσκηση 3  \hfill \small{}}
Να υπολογίσετε τις τιμές των παρακάτω αριθμητικών παραστάσεων:
\begin{enumerate}[i)]
\item $3+5\cdot 5-3\cdot 3-6+2\cdot(-1)$
\item $3\cdot 5+5-2\cdot 3-1+3\cdot(-4)$
\item $4+2\cdot 5-5\cdot 1+5-2\cdot(2)$
\item $4\cdot 1-2\cdot (-3)-1+2\cdot(-4)+3$
\item $6-1\cdot 6+2\cdot 1+15-4\cdot(-4)-2$
\end{enumerate}

\section*{Άσκηση 4  \hfill \small{}}
Να υπολογίσετε τις τιμές των παρακάτω αριθμητικών παραστάσεων:
\begin{enumerate}[i)]
\item $−5+3−(−2\cdot5−10)+(2+7−5−6) $
\item $+3+(−4\cdot4+10)-(2+7\cdot2−1−6) $
\item $(−4\cdot2+1)-(2+3\cdot2−6)+4 $
\item $+3\cdot2-(4\cdot5-10)-3+4-(2+1\cdot5−6) $
\item $-2\cdot2-3-(2\cdot5-10)-3\cdot1+5-(2\cdot9−6) $
\end{enumerate}

\newpage

\section*{Άσκηση 2  \hfill \small{}}
Ένα φόρεμα με αρχική τιμή 80€ έχει έκπτωση 40\%. Ποια είναι η τελική τιμή πώλησής του;

\section*{Άσκηση 3  \hfill \small{}}
Ένα φόρεμα με αρχική τιμή 70€ έχει έκπτωση 20\%. Ποια είναι η τελική τιμή πώλησής του;

\end{document}
