
\documentclass[a4paper,10pt]{report}

%\usepackage[landscape]{geometry}
\usepackage[cm-default]{fontspec}
\setromanfont{FreeSerif}
\setsansfont{FreeSans}
\setmonofont{FreeMono}
\usepackage[utf8x]{inputenc}
\usepackage{fontspec}
\usepackage{xunicode}
\usepackage{xltxtra}
\usepackage{xgreek}
\usepackage{amsmath}
\usepackage{unicode-math}
\usepackage{ulem}
\usepackage{color}
\usepackage{verbatim}
\usepackage{nopageno}
\usepackage{graphicx}
\usepackage{textpos}
\setlength{\TPHorizModule}{1cm}
\setlength{\TPVertModule}{1cm}
%\usepackage[colorgrid,texcoord]{eso-pic}
\usepackage[outline]{contour}
\usepackage{wrapfig}
\usepackage{url}
\usepackage{color}
\usepackage{tikz}
\usepackage{fancybox,fancyhdr}
\usepackage{subfigure}
\usepackage{pstricks}
\usepackage{epsfig}
\usepackage{multicol}
\usepackage{listings}
\usepackage{enumerate}
\usepackage{hyperref}
\hypersetup{
  bookmarks=true,
  bookmarksopen=true,
  pdfborder=false,
  pdfpagemode=UseNone,
  raiselinks=true,
  pdfhighlight={/P},
  colorlinks,
  citecolor=black,
  filecolor=black,
  linkcolor=black,
  urlcolor=black
}


\usepackage{fontspec}
\usepackage{xunicode}
\usepackage{xltxtra}


% Margins
%\setlength{\textwidth}{24cm}

%\setlength{\voffset}{0in}
%\setlength{\textheight}{6.8in}
% Colors
\definecolor{cbrown}{rgb}{0.49,0.24,0.07}
\definecolor{cmpez}{rgb}{0.92,0.88,0.79}
\definecolor{cmpez2}{rgb}{0.62,0.33,0.34}
\definecolor{cwhite}{rgb}{1,1,1}
\definecolor{cblack}{rgb}{0,0,0}
\definecolor{cred}{rgb}{0.9,0.15,0.15}
\definecolor{clblue}{rgb}{0.57,0.85,0.97}
\definecolor{corange}{rgb}{0.86,0.49,0.18}
\definecolor{clorange}{rgb}{0.95,0.84,0.61}
\definecolor{cyellow}{rgb}{1,0.95,0.16}
\definecolor{cgreen}{rgb}{0.69,0.8,0.31}
\definecolor{clbrown}{rgb}{0.78,0.67,0.43}
\definecolor{cmagenta}{rgb}{0.79,0.05,0.54}
\definecolor{cgray}{rgb}{0.76,0.73,0.63}
\definecolor{bluesite}{rgb}{0.05,0.43,0.67}
\definecolor{redsite}{rgb}{0.90,0.15,0.15}
\definecolor{backsite}{rgb}{0.07,0.07,0.07}
% ----------- from Costas -----------------------------

%---------------   Main Fonts    -----------------%   

% \setmainfont[Mapping=tex-text]{Calibri}
% \setmainfont[Mapping=tex-text]{Times New Roman}
%\setmainfont[Mapping=tex-text]{Droid Serif}
%\setmainfont[Mapping=tex-text]{Cambria}
%\setmainfont[Mapping=tex-text]{cm-unicode}
% \setmainfont[Mapping=tex-text]{Gentium}
% \setmainfont[Mapping=tex-text]{GFS Didot}
% \setmainfont[Mapping=tex-text]{Comic Sans MS}
 \setmainfont[Mapping=tex-text]{Ubuntu}
% \setmainfont[Mapping=tex-text]{Myriad Pro}
% \setmainfont[Mapping=tex-text]{CMU Concrete}
%\setmainfont[Mapping=tex-text]{DejaVu Sans}
%\setmainfont[Mapping=tex-text]{KerkisSans}
%\setmainfont[Mapping=tex-text]{KerkisCaligraphic}
% \setmainfont[Mapping=tex-text]{Segoe Print}
% \setmainfont[Mapping=tex-text]{Gabriola}

%------------------------------------------------%

% ------------ Mathematics Fonts ----------------%
\setmathfont{Asana-Math.ttf}
% -----------------------------------------------%
% Misc
\author{Κασωτάκη Ε. - Σμαραγδάκης Κ.}
\title{www.math24.gr}
% ---------------- Formation --------------------%


\setlength\topmargin{-0.7cm}
\addtolength\topmargin{-\headheight}
\addtolength\topmargin{-\headsep}
\setlength\textheight{26cm}
\setlength\oddsidemargin{-0.54cm}
\setlength\evensidemargin{-0.54cm}
%\setlength\marginparwidth{1.5in}
\setlength\textwidth{17cm}

\RequirePackage[avantgarde]{quotchap}
\renewcommand\chapterheadstartvskip{\vspace*{0\baselineskip}}
\RequirePackage[calcwidth]{titlesec}
\titleformat{\section}[hang]{\bfseries}
{\Large\thesection}{12pt}{\Large}[{\titlerule[0.9pt]}]
%--------------------------------------------------%
\usepackage{draftwatermark}
\SetWatermarkText{www.math24.gr}
\SetWatermarkLightness{0.9}
\SetWatermarkScale{0.6}
%--------------------------------------------------%


\usepackage{xcolor}
\usepackage{amsthm}
\usepackage{framed}
\usepackage{parskip}

\colorlet{shadecolor}{bluesite!20}

\newtheorem*{orismos}{Ορισμός}

\newenvironment{notation}
  {\begin{shaded}\begin{theorem}}
  {\end{theorem}\end{shaded}}




% Document begins
\begin{document}
\pagestyle{fancy}
\fancyhead{}
\fancyfoot{}
\renewcommand{\headrulewidth}{0pt}
\renewcommand{\footrulewidth}{0pt}


\fancyhead[LO,LE]{
 %\textblockcolor{backsite}
 %\begin{textblock}{5}(-2,-0.55)
  %\rule{0cm}{1cm}
 %\end{textblock}
 \textblockcolor{bluesite}
 \begin{textblock}{5}(-1.5,-0.55)
  \rule{0cm}{1cm}
 \end{textblock}
 %\textblockcolor{bluesite}
 %\begin{textblock}{14}(5.5,-0.55)
 % \rule{0cm}{1cm}
 %\end{textblock}
 \begin{textblock}{0}(-1,-0.25)
 \color{cwhite} \begin{Large}www.math24.gr\end{Large}
 \end{textblock}
\begin{textblock}{0}(16,-0.4)
 \color{cwhite} \includegraphics[height=1.5cm]{math24_logo.png}
 \end{textblock}
\textblockcolor{white}
\begin{textblock}{17}(-1,28)
 \color{backsite} \begin{small}Copyright \textcopyright 2011, Κασωτάκη Ε.(ikasotaki@gmail.com) - Σμαραγδάκης Κ.(kesmarag@gmail.com)\end{small}
 \end{textblock}
}
 
\begin{shaded}
\begin{center}
\huge \textbf{Φυλλάδιο Ασκήσεων}\\
%Πρόσθεση ρητών αριθμών
\end{center} 
\textbf{Μαθηματικά Α' Γυμνασίου} \hfill \textbf{Ημερομηνία Παράδοσης : \hspace{2em} }
\subsection*{Ονοματεπώνυμο :\hfill  \hspace{5em}}
\end{shaded}
\vspace{2em}
\begin{itemize}
 \item Ισοδύναμα κλάσματα 
 \item Απλοποίηση κλάσματος - Μετατροπή κλάσματος σε ανάγωγο κλάσμα
 \item Ομώνυμα κλάσματα - Ετερώνυμα κλάσματα - Μετατροπή κλασμάτων σε ομώνυμα
\end{itemize}
\section*{Θεωρία - Ισοδύναμα Κλάσματα\hfill \small{}}
\begin{itemize}
 \item \textbf{Ισοδύναμα κλάσματα:} \\
        Δύο κλάσματα $\dfrac{α}{β}$ και $\dfrac{γ}{δ}$ λέγονται ισοδύναμα 
        όταν εκφράζουν το ίδιο τμήμα ενός μεγέθους ή ίσων μεγεθών.
 \item \textbf{ Κανόνας:} \\
      \textbf{Αν} δύο κλάσματα $\dfrac{α}{β}$ και $\dfrac{γ}{δ}$ είναι ισοδύναμα \textbf{τότε} τα "χιαστί γινόμενα" 
      $α\cdot δ$ και $β\cdot γ$ είναι ίσα.\\
      Δηλαδή: \textbf{Αν} $\dfrac{α}{β}=\dfrac{γ}{δ}$ \textbf{τότε} $α\cdot δ=β\cdot γ$ 
 \item \textbf{Πώς εξετάζουμε αν 2 κλάσματα είναι ισοδύναμα:}\\
       Για να εξετάσουμε αν 2 κλάσματα είναι ισοδύναμα, υπολογίζουμε τα "χιαστί γινόμενα" και 
       \begin{itemize}
        \item 1η περίπτωση: \textbf{αν} είναι ίσα \textbf{τότε} τα κλάσματα είναι ισοδύναμα
        \item 2η περίπτωση: \textbf{αν} δεν είναι ίσα \textbf{τότε} τα κλάσματα δεν είναι ισοδύναμα
       \end{itemize}
       \textbf{Παράδειγμα:} Είναι τα κλάσματα $\dfrac{1}{2}$ και $\dfrac{2}{4}$  ισοδύναμα;\\
       Υπολογίζουμε τα "χιαστί γινόμενα" $1\cdot4=4$ και $2\cdot2=4$. Αφού τα "χιαστί γινόμενα" είναι ίσα, άρα και τα 
      κλάσματα είναι ισοδύναμα.\\
     \textbf{Παράδειγμα:} Είναι τα κλάσματα $\dfrac{1}{3}$ και $\dfrac{3}{5}$  ισοδύναμα;\\
       Υπολογίζουμε τα "χιαστί γινόμενα" $1\cdot5=5$ και $3\cdot3=$. Αφού τα "χιαστί γινόμενα" δεν είναι ίσα, άρα τα 
      κλάσματα δεν είναι ισοδύναμα.

\end{itemize}


\section*{Άσκηση 1  (Ισοδύναμα Κλάσματα)\hfill \small{5 μονάδες}}
Να εξετάσετε αν τα παρακάτω κλάσματα είναι ισοδύναμα:
\begin{enumerate}[i)]
 \item $\dfrac{12}{20}$ και $\dfrac{3}{5}$
 \item $\dfrac{4}{8}$ και $\dfrac{8}{16}$
 \item $\dfrac{7}{3}$ και $\dfrac{21}{8}$
 \item $\dfrac{30}{60}$ και $\dfrac{20}{30}$
 \item $\dfrac{22}{2}$ και $\dfrac{44}{4}$
\end{enumerate}

\section*{Θεωρία - Απλοποίηση κλάσματος - Μετατροπή κλάσματος σε ανάγωγο \hfill \small{}}
\textbf{Κανόνας}: \\
Όταν οι όροι ενός κλάσματος πολλαπλασιαστούν ή διαιρεθούν με τον ίδιο φυσικό αριθμό $\neq0$ 
τότε προκύπτει ισοδύναμο κλάσμα.\\ \\
\textbf{Απλοποίηση Κλάσματος} \\
Απλοποίηση ενός κλάσματος ονομάζεται η διαδικασία κατά την οποία οι όροι ενός κλάσματος διαιρούνται με τον ίδιο φυσικό 
αριθμό $\neq0$ και προκύπτει ένα κλάσμα ισοδύναμο με το αρχικό με μικρότερους όρους.\\ \\
\textbf{Ανάγωγο κλάσμα} \\
Ανάγωγο ονομάζεται το κλάσμα εκείνο που δεν μπορεί να απλοποιηθεί (δηλαδή δεν υπάρχει κοινός διαιρέτης αριθμητή 
και παρονομαστή).\\ \\
\textbf{Μεθοδολογία απλοποίησης κλασμάτων - Μετατροπή κλάσματος σε ανάγωγο κλάσμα} \\
Όταν μας ζητάνε ν' απλοποιήσουμε ένα κλάσμα αυτό σημαίνει ότι πρέπει να το μετατρέψουμε σε ανάγωγο κλάσμα. 
Για να μετατρέψουμε ένα κλάσμα σε ανάγωγο διαιρούμε τους όρους του κλάσματος με το μέγιστο κοινό διαιρέτη τους, 
οπότε το κλάσμα που προκύπτει είναι ανάγωγο.
\begin{itemize}
 \item \textbf{π.χ} Για να απλοποιήσουμε το κλάσμα $\dfrac{10}{20}$ βρίσκουμε ότι $ΜΚΔ(10,20)=10$ οπότε 
$\dfrac{10}{20}=\dfrac{10:10}{20:10}=\dfrac{1}{2}$
\end{itemize}




\section*{Άσκηση 2 (Μετατροπή κλάσματος σε ανάγωγο ) \hfill \small{5 μονάδες}}
Να απλοποιήσετε τα παρακάτω κλάσματα:
\begin{enumerate}[i)]
 \item $\dfrac{13}{26}$
 \item $\dfrac{24}{9}$
 \item $\dfrac{15}{18}$
 \item $\dfrac{14}{21}$
 \item $\dfrac{8}{32}$
\end{enumerate}

\section*{Άσκηση 3 (Ανάγωγα Κλάσματα ) \hfill \small{5 μονάδες}}
Να βρείτε ποια από τα παραπάνω κλάσματα είναι ανάγωγα:
\begin{enumerate}[i)]
 \item  $\dfrac{12}{24}$
 \item  $\dfrac{3}{5}$
 \item  $\dfrac{11}{7}$
 \item  $\dfrac{13}{14}$
 \item  $\dfrac{5}{15}$
\end{enumerate}



\section*{Θεωρία - Ομώνυμα και Ετερώνυμα Κλάσματα\hfill \small{}}
\textbf{Ομώνυμα}: \\
Ομώνυμα ονομάζονται δύο ή περισσότερα κλάσματα τα οποία έχουν τον ίδιο παρονομαστή. \\
\textbf{π.χ} τα κλάσματα $\dfrac{2}{5},\dfrac{4}{5},\dfrac{7}{5}$ είναι ομώνυμα \\ \\
\textbf{Ετερώνυμα}: \\
Ετερώνυμα ονομάζονται δύο ή περισσότερα κλάσματα τα οποία έχουν διαφορετικό παρονομαστή. \\
\textbf{π.χ} τα κλάσματα $\dfrac{1}{3},\dfrac{2}{5},\dfrac{2}{7}$ είναι ετερώνυμα \\ \\
\textbf{Μετατροπή κλασμάτων σε ομώνυμα}: \\
Για να μετατρέψουμε δύο ή περισσότερα κλάσματα σε ομώνυμα εκτελούμε τα παρακάτω βήματα:
\begin{itemize}
 \item \textbf{1ο Βήμα} Ελέγχουμε αν τα ετερώνυμα κλάσματα απλοποιούνται και αν ναι τα μετατρέπουμε σε ανάγωγα.
 \item \textbf{2ο Βήμα} Βρίσκουμε το $ΕΚΠ$ των παρονομαστών των ανάγωγων ετερώνυμων κλασμάτων.
 \item \textbf{3ο Βήμα} Διαιρούμε το $ΕΚΠ$ που βρήκαμε με καθένα από τους παρονομαστές.
 \item \textbf{4ο Βήμα} Πολλαπλασιάζουμε τον αριθμητή και τον παρονομαστή κάθε κλάσματος επί τον αντίστοιχο αριθμό που 
                        βρήκαμε και έτσι τα κλάσματα μετατρέπονται σε ισοδύναμα ομώνυμα.
\end{itemize}
%------------------------------------------------------------------------------------------
\vspace{2em}
\textbf{Παράδειγμα:} \\
Μετατροπή των κλασμάτων $\dfrac{3}{2},\dfrac{5}{3}$ και $\dfrac{14}{12} $ σε ομώνυμα
\begin{itemize}
 \item \textbf{1ο Βήμα} Το κλάσμα $\dfrac{3}{2}$ είναι ανάγωγο αφού $ΜΚΔ(3,2)=1$\\
                        Το κλάσμα $\dfrac{5}{3}$ είναι ανάγωγο αφού $ΜΚΔ(5,3)=1$\\
                        Το κλάσμα $\dfrac{14}{12}$ δεν είναι ανάγωγο αφού $ΜΚΔ(14,12)=2$ οπότε 
                        το μετατρέπουμε σε ανάγωγο\\
                         $\dfrac{14}{12}=\dfrac{14:2}{12:2}=\dfrac{7}{6}$
 \item \textbf{2ο Βήμα} Τώρα τα κλάσματα είναι $\dfrac{3}{2},\dfrac{5}{3}$ και $\dfrac{7}{6} $ \\
                        Βρίσκουμε το $ΕΚΠ$ των παρονομαστών: $ΕΚΠ(2,3,6)=6$\\
 \item \textbf{3ο Βήμα} \\Για το κλάσμα $\dfrac{3}{2}$ έχουμε $6:2=3$\\
                        Για το κλάσμα $\dfrac{5}{3}$ έχουμε $6:3=2$\\
                        Για το κλάσμα $\dfrac{7}{6}$ έχουμε $6:6=1$
 \item \textbf{4ο Βήμα} \\ $\dfrac{3}{2}=\dfrac{3\cdot3}{2\cdot3}=\dfrac{9}{6}$\\
                         $\dfrac{5}{3}=\dfrac{5\cdot2}{3\cdot2}=\dfrac{10}{6}$\\
                         $\dfrac{7}{6}=\dfrac{7\cdot1}{6\cdot1}=\dfrac{7}{6}$\\
\end{itemize}
%------------------------------------------------------------------------------------------


\section*{Άσκηση 4 (Μετατροπή κλασμάτων σε ομώνυμα) \hfill \small{5 μονάδες}}
Να μετατρέψετε τα παρακάτω κλάσματα σε ομώνυμα
\begin{enumerate}[i)]
 \item $\dfrac{4}{6}$ και $\dfrac{5}{4}$
 \item $\dfrac{5}{2}$ και $\dfrac{3}{4}$
 \item $\dfrac{7}{14}$ και $\dfrac{8}{6}$
 \item $\dfrac{1}{5}$ και $\dfrac{4}{20}$
 \item $\dfrac{1}{2}$,$\dfrac{1}{3}$     και $\dfrac{1}{4}$
\end{enumerate}






\end{document}
