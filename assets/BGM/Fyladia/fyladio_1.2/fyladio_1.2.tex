
\documentclass[a4paper,10pt]{report}

%\usepackage[landscape]{geometry}
\usepackage[cm-default]{fontspec}
\setromanfont{FreeSerif}
\setsansfont{FreeSans}
\setmonofont{FreeMono}
\usepackage[utf8x]{inputenc}
\usepackage{fontspec}
\usepackage{xunicode}
\usepackage{xltxtra}
\usepackage{xgreek}
\usepackage{amsmath}
\usepackage{unicode-math}
\usepackage{ulem}
\usepackage{color}
\usepackage{verbatim}
\usepackage{nopageno}
\usepackage{graphicx}
\usepackage{textpos}
\setlength{\TPHorizModule}{1cm}
\setlength{\TPVertModule}{1cm}
%\usepackage[colorgrid,texcoord]{eso-pic}
\usepackage[outline]{contour}
\usepackage{wrapfig}
\usepackage{url}
\usepackage{color}
\usepackage{tikz}
\usepackage{fancybox,fancyhdr}
\usepackage{subfigure}
\usepackage{pstricks}
\usepackage{epsfig}
\usepackage{multicol}
\usepackage{listings}
\usepackage{enumerate}
\usepackage{hyperref}
\hypersetup{
  bookmarks=true,
  bookmarksopen=true,
  pdfborder=false,
  pdfpagemode=UseNone,
  raiselinks=true,
  pdfhighlight={/P},
  colorlinks,
  citecolor=black,
  filecolor=black,
  linkcolor=black,
  urlcolor=black
}


\usepackage{fontspec}
\usepackage{xunicode}
\usepackage{xltxtra}


% Margins
%\setlength{\textwidth}{24cm}

%\setlength{\voffset}{0in}
%\setlength{\textheight}{6.8in}
% Colors
\definecolor{cbrown}{rgb}{0.49,0.24,0.07}
\definecolor{cmpez}{rgb}{0.92,0.88,0.79}
\definecolor{cmpez2}{rgb}{0.62,0.33,0.34}
\definecolor{cwhite}{rgb}{1,1,1}
\definecolor{cblack}{rgb}{0,0,0}
\definecolor{cred}{rgb}{0.9,0.15,0.15}
\definecolor{clblue}{rgb}{0.57,0.85,0.97}
\definecolor{corange}{rgb}{0.86,0.49,0.18}
\definecolor{clorange}{rgb}{0.95,0.84,0.61}
\definecolor{cyellow}{rgb}{1,0.95,0.16}
\definecolor{cgreen}{rgb}{0.69,0.8,0.31}
\definecolor{clbrown}{rgb}{0.78,0.67,0.43}
\definecolor{cmagenta}{rgb}{0.79,0.05,0.54}
\definecolor{cgray}{rgb}{0.76,0.73,0.63}
\definecolor{bluesite}{rgb}{0.05,0.43,0.67}
\definecolor{redsite}{rgb}{0.90,0.15,0.15}
\definecolor{backsite}{rgb}{0.07,0.07,0.07}
% ----------- from Costas -----------------------------

%---------------   Main Fonts    -----------------%   

% \setmainfont[Mapping=tex-text]{Calibri}
% \setmainfont[Mapping=tex-text]{Times New Roman}
%\setmainfont[Mapping=tex-text]{Droid Serif}
%\setmainfont[Mapping=tex-text]{Cambria}
%\setmainfont[Mapping=tex-text]{cm-unicode}
% \setmainfont[Mapping=tex-text]{Gentium}
% \setmainfont[Mapping=tex-text]{GFS Didot}
% \setmainfont[Mapping=tex-text]{Comic Sans MS}
 \setmainfont[Mapping=tex-text]{Ubuntu}
% \setmainfont[Mapping=tex-text]{Myriad Pro}
% \setmainfont[Mapping=tex-text]{CMU Concrete}
%\setmainfont[Mapping=tex-text]{DejaVu Sans}
%\setmainfont[Mapping=tex-text]{KerkisSans}
%\setmainfont[Mapping=tex-text]{KerkisCaligraphic}
% \setmainfont[Mapping=tex-text]{Segoe Print}
% \setmainfont[Mapping=tex-text]{Gabriola}

%------------------------------------------------%

% ------------ Mathematics Fonts ----------------%
%\setmathfont{Asana-Math.ttf}
% -----------------------------------------------%
% Misc
\author{Κασωτάκη Ε. - Σμαραγδάκης Κ.}
\title{www.math24.gr}
% ---------------- Formation --------------------%


\setlength\topmargin{-0.7cm}
\addtolength\topmargin{-\headheight}
\addtolength\topmargin{-\headsep}
\setlength\textheight{26cm}
\setlength\oddsidemargin{-0.54cm}
\setlength\evensidemargin{-0.54cm}
%\setlength\marginparwidth{1.5in}
\setlength\textwidth{17cm}

\RequirePackage[avantgarde]{quotchap}
\renewcommand\chapterheadstartvskip{\vspace*{0\baselineskip}}
\RequirePackage[calcwidth]{titlesec}
\titleformat{\section}[hang]{\bfseries}
{\Large\thesection}{12pt}{\Large}[{\titlerule[0.9pt]}]
%--------------------------------------------------%
%\usepackage{draftwatermark}
%\SetWatermarkText{www.math24.gr}
%\SetWatermarkLightness{0.9}
%\SetWatermarkScale{0.6}
%--------------------------------------------------%


\usepackage{xcolor}
\usepackage{amsthm}
\usepackage{framed}
\usepackage{parskip}

\colorlet{shadecolor}{bluesite!20}

\newtheorem*{orismos}{Ορισμός}

\newenvironment{notation}
  {\begin{shaded}\begin{theorem}}
  {\end{theorem}\end{shaded}}




% Document begins
\begin{document}
\pagestyle{fancy}
\fancyhead{}
\fancyfoot{}
\renewcommand{\headrulewidth}{0pt}
\renewcommand{\footrulewidth}{0pt}


\fancyhead[LO,LE]{
 %\textblockcolor{backsite}
 %\begin{textblock}{5}(-2,-0.55)
  %\rule{0cm}{1cm}
 %\end{textblock}
 \textblockcolor{bluesite}
 \begin{textblock}{5}(-1.5,-0.55)
  \rule{0cm}{1cm}
 \end{textblock}
 %\textblockcolor{bluesite}
 %\begin{textblock}{14}(5.5,-0.55)
 % \rule{0cm}{1cm}
 %\end{textblock}
 \begin{textblock}{0}(-1,-0.25)
 \color{cwhite} \begin{Large}www.math24.gr\end{Large}
 \end{textblock}
\begin{textblock}{0}(16,-0.4)
 \color{cwhite} \includegraphics[height=1.5cm]{math24_logo.png}
 \end{textblock}
\textblockcolor{white}
\begin{textblock}{17}(-1,28)
 \color{backsite} \begin{small}Επιμέλεια: Κασωτάκη Ειρήνη (ikasotaki@gmail.com) - Σχολικό έτος 2019 - 2020\end{small}
 \end{textblock}
}
 
\begin{shaded}
\begin{center}
\huge \textbf{Φυλλάδιο Ασκήσεων}\\
%Πρόσθεση ρητών αριθμών
\end{center} 
\textbf{Μαθηματικά Β' Γυμνασίου} \hfill \textbf{Ημερομηνία Παράδοσης : \hspace{2em} }
\subsection*{Ονοματεπώνυμο : \hfill  \hspace{5em}}
\end{shaded}
\vspace{2em}
\begin{itemize}
 \item Εξίσωση α' βαθμού
 \item Λύση εξίσωσης
 \item Αναγωγή ομοίων όρων
\end{itemize}
\section*{Θεωρία \hfill \small{}}
\textbf{Εξίσωση} ονομάζεται μία ισότητα που περιέχει τον άγνωστο αριθμό $x$\\
\begin{itemize}
 \item \textbf{π.χ} $3x+5=11-x$ \\
       \begin{itemize}
        \item Η παράσταση $3x+5$ ονομάζεται πρώτο μέλος της εξίσωσης
        \item Η παράσταση $11-x$ ονομάζεται δεύτερο μέλος της εξίσωσης
       \end{itemize}
\end{itemize}
%-------------------------------------------------------------------
\textbf{Λύση} μίας εξίσωσης είναι η τιμή της μεταβλητής που την επαληθεύει
\begin{itemize}
 \item \textbf{π.χ} η $x=1$ είναι λύση της εξίσωσης $2x+3=5$ γιατί την επαληθεύει ($2\cdot1+3=5$ ισχύει) 
 \item \textbf{π.χ} η $x=2$ δεν είναι λύση της εξίσωσης $2x+3=5$ γιατί δεν την επαληθεύει ($2\cdot2+3=5$ δεν ισχύει) 
\end{itemize}




\section*{Άσκηση 1  \hfill \small{20 μονάδες}}
Να εξετάσετε αν ο αριθμός που δίνεται είναι λύση της εξίσωσης
\begin{enumerate}[1)]
 \item $2x+4=5$, \quad $x=\dfrac{1}{2}$
 \item $3x+2=10$, \quad $x=2$
 \item $5x+10=3x+12$, \quad $x=1$
 \item $3x+5=6x$, \quad $x=\dfrac{1}{3}$
 \item $7x+5=6x+1$, \quad $x=10$
\end{enumerate}

\section*{Θεωρία \hfill \small{}}
Σε μία εξίσωση μπορούμε
\begin{itemize}
 \item να \textbf{προσθέσουμε} και στα δύο μέλη τον ίδιο αριθμό
 \item να \textbf{αφαιρέσουμε} και από τα δύο μέλη τον ίδιο αριθμό
 \item να \textbf{πολλαπλασιάσουμε} και τα δύο μέλη τον ίδιο αριθμό
 \item να \textbf{διαιρέσουμε} και τα δύο μέλη με τον ίδιο αριθμό $\neq0$
 \item να "\textbf{μεταφέρουμε}" όρους από το ένα μέλος στο άλλο, αλλάζοντας το πρόσημό τους
\end{itemize}
%-------------------------------------------------------------------------------------------
\begin{center}
 \begin{tabular}{|c|l|}\hline 
\textbf{Επίλυση εξίσωσης} \quad        &    \textbf{Περιγραφή λύσης}       \\
$3x+20=x+60$     \quad \quad                 &         \\
\hline 
 $3x+20-x=60$                                 & Μεταφέρουμε το $+x$ στο πρώτο μέλος οπότε γίνεται $-x$ \\ 
                                  & \\                               
\hline
$3x-x=60-20$                                  &  Μεταφέρουμε το $+20$ στο δεύτερο μέλος οπότε γίνεται $-20$        \\
                                  & \\                               
 
\hline
$(3-1)x=60-20$                                  & Κάνουμε αναγωγή ομοίων όρων       \\
                                  & \\
\hline
$2x=40$                                  & Κάνουμε τις πράξεις       \\
                                  & \\
\hline
$\dfrac{2x}{2}=\dfrac{40}{2}$      &  Διαιρούμε με το συντελεστή του αγνώστου και τα δύο μέλη της εξίσωσης      \\
                                  & \\  
\hline
$x=20$                                  &  Απλοποιούμε τα κλάσματα       \\
                                  & \\ 
\hline 
\end{tabular}
\end{center}


\section*{Άσκηση 2  \hfill \small{20 μονάδες}}
Να λύσετε τις παρακάτω εξισώσεις
\begin{enumerate}[1)]
 \item $6x+1=4x+21$
 \item $7x-20=-x-4$
 \item $10x+2=5x+12$
 \item $5x+7=4x+7$
 \item $3x+10=6x-5$
\end{enumerate}
%------------------------------------------------------------------
\vspace{3em}
\begin{center}
 \begin{tabular}{|c|l|}\hline 
\textbf{Επίλυση εξίσωσης} \quad        &    \textbf{Περιγραφή λύσης}       \\
$2(x+1)+3(x+2)=3(1-x)$     \quad \quad                 &         \\
\hline 
$2x+2+3x+3x+6=3-3x$                                  & Κάνουμε τις πράξεις (επιμεριστική ιδιότητα)\\ 
                                  & \\                               
\hline
$2x+3x+3x=3-2-6$                                  &  Χωρίζουμε γνωστούς από αγνώστους        \\
                                  & \\                               
 
\hline
$8x=-5$                                  & Κάνουμε αναγωγή ομοίων όρων       \\
                                  & \\
\hline
$\dfrac{8x}{8}=-\dfrac{5}{8}$                                  &  Διαιρούμε με το συντελεστή του αγνώστου και τα δύο μέλη της εξίσωσης      \\
                                  & \\  
\hline
$x=-\dfrac{5}{8}$                                  &  Απλοποιούμε τα κλάσματα       \\
                                  & \\ 
\hline 
\end{tabular}
\end{center}
%-----------------------------------------------------
\section*{Άσκηση 3  \hfill \small{20 μονάδες}}
Να λύσετε τις παρακάτω εξισώσεις
\begin{enumerate}[1)]
 \item $2(x+1)=3(3-x)$
 \item $5(x-1)+2=6x$
 \item $5(x-2)+3x=7(x+1)$
 \item $2(x-1)+2(x+1)=3(x+7)$
 \item $6(x-2)=8(x-3)$
\end{enumerate}
%-----------------------------------------------------------
\vspace{3em}

\begin{center}
 \begin{tabular}{|c|l|}\hline 
\textbf{Επίλυση εξίσωσης} \quad        &    \textbf{Περιγραφή λύσης}       \\
$\dfrac{x+1}{2}+\dfrac{x+4}{3}=x+2$     \quad \quad                 &         \\
\hline 
 $6(\dfrac{x+1}{2}+\dfrac{x+4}{3})=6(x+2)$                                 & Απαλοιφή παρονομαστών: \\ 
                                  & Πολλαπλασιάζουμε και τα δύο μέλη της εξίσωσης με το ΕΚΠ των παρονομαστών\\   
                                  & \\                            
\hline
 $6\cdot \dfrac{x+1}{2}+6\cdot \dfrac{x+4}{3}=6x+12$   &  Κάνουμε τις πράξεις (επιμεριστική ιδιότητα)       \\
                                  & \\  
\hline
$3(x+1)+2(x+4)=6x+12$                                  &  Απλοποιούμε τα κλάσματα       \\
                                  & \\                          
\hline
$3x+3+2x+8=6x+12$                                  &  Κάνουμε τις πράξεις (επιμεριστική ιδιότητα)       \\
                                  & \\  
                                  & \\
\hline
$3x+2x-6x=12-3-8$                                  &  Χωρίζουμε γνωστούς από αγνώστους      \\
                                  & \\  
\hline
                $-x=1$                  & Κάνουμε αναγωγή ομοίων όρων       \\
                                  & \\
\hline
$\dfrac{-x}{-1}=\dfrac{1}{-1}$    &  Διαιρούμε με το συντελεστή του αγνώστου και τα δύο μέλη της εξίσωσης      \\
                                  & \\  
\hline
$x=-1$                                  &  Απλοποιούμε τα κλάσματα       \\
                                  & \\ 
\hline 
\end{tabular}
\end{center}
%---------------------------------------------------------------
\section*{Άσκηση 4  \hfill \small{20 μονάδες}}
Να λύσετε τις παρακάτω εξισώσεις
\begin{enumerate}[1)]
 \item $\dfrac{x-1}{2}=\dfrac{2x+5}{6}$
 \item $\dfrac{2x}{5}+1=\dfrac{7x+2}{10}$
 \item $\dfrac{x}{2}+\dfrac{x-1}{3}=\dfrac{2x}{6}$
 \item $\dfrac{x-7}{13}+\dfrac{x+10}{13}=x$
 \item $\dfrac{3x+2}{2}+5=\frac{8x+10}{4}$
\end{enumerate}
%-----------------------------------------------------------
\section*{Θεωρία \hfill \small{}}
\begin{itemize}
 \item Όταν μία εξίσωση δεν έχει καμία λύση ονομάζεται \textbf{αδύνατη}
 \item Όταν για μία εξίσωση κάθε αριθμός είναι λύση της, τότε η εξίσωση ονομάζεται \textbf{ταυτότητα}
\end{itemize}
%-------------------------------------------------------------
\vspace{3em}
\begin{center}
 \begin{tabular}{|c|l|}\hline 
\textbf{Επίλυση εξίσωσης} \quad        &    \textbf{Περιγραφή λύσης}       \\
$5(x+1)+x=6(x+2)$     \quad \quad                 &         \\
\hline 
$5x+5+x=6x+12$                                  & Κάνουμε τις πράξεις (επιμεριστική ιδιότητα)\\ 
                                  & \\                               
\hline
$5x+x-6x=12-5$                                  &  Χωρίζουμε γνωστούς από αγνώστους        \\
                                  & \\                               
 
\hline
$0x=-7$                                  & Κάνουμε αναγωγή ομοίων όρων       \\
                                  & \\
\hline
Η εξίσωση είναι αδύνατη                                  &  Τι παρατηρούμε;     \\
                                  & \\  
\hline 
\end{tabular}
\end{center}
%--------------------------------------------------------------------
\section*{Άσκηση 5  \hfill \small{10 μονάδες}}
Να λύσετε τις παρακάτω εξισώσεις
\begin{enumerate}[1)]
 \item $5(x-2)=5x+7$
 \item $3(2x+1)=2(1+3x)$
 \item $2x+5(x-1)=3(x+2)$
 \item $\dfrac{x-1}{2}+2=7+\dfrac{x}{2}$
 \item $\dfrac{x-1}{2}+\dfrac{x-1}{3}=\dfrac{5x-10}{6}$
\end{enumerate}
%------------------------------------------------------------------
\vspace{3em}
\begin{center}
 \begin{tabular}{|c|l|}\hline 
\textbf{Επίλυση εξίσωσης} \quad        &    \textbf{Περιγραφή λύσης}       \\
$2(x-1)-5=2(x-3)-1$     \quad \quad                 &         \\
\hline 
 $2x-2-5=2x-6-1$                                 & Κάνουμε τις πράξεις (επιμεριστική ιδιότητα)\\ 
                                  & \\                               
\hline
$2x-2x=-6-1+5+2$                                  &  Χωρίζουμε γνωστούς από αγνώστους        \\
                                  & \\                               
 
\hline
$0x=0$                                  & Κάνουμε αναγωγή ομοίων όρων       \\
                                  & \\
\hline
Η εξίσωση είναι ταυτότητα                                  &  Τι παρατηρούμε;     \\
                                  & \\  
\hline 
\end{tabular}
\end{center}
\section*{Άσκηση 6  \hfill \small{10 μονάδες}}
Να λύσετε τις παρακάτω εξισώσεις
\begin{enumerate}[1)]
 \item $2(x+2)+2=2(x+3)$
 \item $5(x+1)+3=2(2x+4)+x$
 \item $5(1-2x)=2(x+1)-12x+3$
 \item $7x+4=2(2+3x)+x$
 \item $\dfrac{x-1}{2}=\dfrac{2x-2}{4}$
\end{enumerate}



\end{document}
