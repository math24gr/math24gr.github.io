
\documentclass[a4paper,11pt]{report}

%\usepackage[landscape]{geometry}
\usepackage[cm-default]{fontspec}
\setromanfont{FreeSerif}
\setsansfont{FreeSans}
\setmonofont{FreeMono}
\usepackage[utf8x]{inputenc}
\usepackage{fontspec}
\usepackage{xunicode}
\usepackage{xltxtra}
\usepackage{xgreek}
\usepackage{amsmath}
\usepackage{unicode-math}
\usepackage{ulem}
\usepackage{color}
\usepackage{verbatim}
\usepackage{nopageno}
\usepackage{graphicx}
\usepackage{textpos}
\setlength{\TPHorizModule}{1cm}
\setlength{\TPVertModule}{1cm}
%\usepackage[colorgrid,texcoord]{eso-pic}
\usepackage[outline]{contour}
\usepackage{wrapfig}
\usepackage{url}
\usepackage{color}
\usepackage{tikz}
\usepackage{fancybox,fancyhdr}
\usepackage{subfigure}
\usepackage{pstricks}
\usepackage{epsfig}
\usepackage{multicol}
\usepackage{listings}
\usepackage{enumerate}
\usepackage{hyperref}
\hypersetup{
  bookmarks=true,
  bookmarksopen=true,
  pdfborder=false,
  pdfpagemode=UseNone,
  raiselinks=true,
  pdfhighlight={/P},
  colorlinks,
  citecolor=black,
  filecolor=black,
  linkcolor=black,
  urlcolor=black
}


\usepackage{fontspec}
\usepackage{xunicode}
\usepackage{xltxtra}


% Margins
%\setlength{\textwidth}{24cm}

%\setlength{\voffset}{0in}
%\setlength{\textheight}{6.8in}
% Colors
\definecolor{cbrown}{rgb}{0.49,0.24,0.07}
\definecolor{cmpez}{rgb}{0.92,0.88,0.79}
\definecolor{cmpez2}{rgb}{0.62,0.33,0.34}
\definecolor{cwhite}{rgb}{1,1,1}
\definecolor{cblack}{rgb}{0,0,0}
\definecolor{cred}{rgb}{0.9,0.15,0.15}
\definecolor{clblue}{rgb}{0.57,0.85,0.97}
\definecolor{corange}{rgb}{0.86,0.49,0.18}
\definecolor{clorange}{rgb}{0.95,0.84,0.61}
\definecolor{cyellow}{rgb}{1,0.95,0.16}
\definecolor{cgreen}{rgb}{0.69,0.8,0.31}
\definecolor{clbrown}{rgb}{0.78,0.67,0.43}
\definecolor{cmagenta}{rgb}{0.79,0.05,0.54}
\definecolor{cgray}{rgb}{0.76,0.73,0.63}
\definecolor{bluesite}{rgb}{0.05,0.43,0.67}
\definecolor{redsite}{rgb}{0.90,0.15,0.15}
\definecolor{backsite}{rgb}{0.07,0.07,0.07}
% ----------- from Costas -----------------------------

%---------------   Main Fonts    -----------------%   

% \setmainfont[Mapping=tex-text]{Calibri}
% \setmainfont[Mapping=tex-text]{Times New Roman}
%\setmainfont[Mapping=tex-text]{Droid Serif}
%\setmainfont[Mapping=tex-text]{Cambria}
%\setmainfont[Mapping=tex-text]{cm-unicode}
% \setmainfont[Mapping=tex-text]{Gentium}
% \setmainfont[Mapping=tex-text]{GFS Didot}
% \setmainfont[Mapping=tex-text]{Comic Sans MS}
 \setmainfont[Mapping=tex-text]{Ubuntu}
% \setmainfont[Mapping=tex-text]{Myriad Pro}
% \setmainfont[Mapping=tex-text]{CMU Concrete}
%\setmainfont[Mapping=tex-text]{DejaVu Sans}
%\setmainfont[Mapping=tex-text]{KerkisSans}
%\setmainfont[Mapping=tex-text]{KerkisCaligraphic}
% \setmainfont[Mapping=tex-text]{Segoe Print}
% \setmainfont[Mapping=tex-text]{Gabriola}

%------------------------------------------------%

% ------------ Mathematics Fonts ----------------%
\setmathfont{Asana-Math.ttf}
% -----------------------------------------------%
% Misc
\author{Κασωτάκη Ε. - Σμαραγδάκης Κ.}
\title{www.math24.gr}
% ---------------- Formation --------------------%


\setlength\topmargin{-0.7cm}
\addtolength\topmargin{-\headheight}
\addtolength\topmargin{-\headsep}
\setlength\textheight{26cm}
\setlength\oddsidemargin{-0.54cm}
\setlength\evensidemargin{-0.54cm}
%\setlength\marginparwidth{1.5in}
\setlength\textwidth{17cm}

\RequirePackage[avantgarde]{quotchap}
\renewcommand\chapterheadstartvskip{\vspace*{0\baselineskip}}
\RequirePackage[calcwidth]{titlesec}
\titleformat{\section}[hang]{\bfseries}
{\Large\thesection}{12pt}{\Large}[{\titlerule[0.9pt]}]
%--------------------------------------------------%
\usepackage{draftwatermark}
\SetWatermarkText{www.math24.gr}
\SetWatermarkLightness{0.9}
\SetWatermarkScale{0.6}
%--------------------------------------------------%


\usepackage{xcolor}
\usepackage{amsthm}
\usepackage{framed}
\usepackage{parskip}

\colorlet{shadecolor}{bluesite!20}

\newtheorem*{orismos}{Ορισμός}

\newenvironment{notation}
  {\begin{shaded}\begin{theorem}}
  {\end{theorem}\end{shaded}}




% Document begins
\begin{document}
\pagestyle{fancy}
\fancyhead{}
\fancyfoot{}
\renewcommand{\headrulewidth}{0pt}
\renewcommand{\footrulewidth}{0pt}


\fancyhead[LO,LE]{
 %\textblockcolor{backsite}
 %\begin{textblock}{5}(-2,-0.55)
  %\rule{0cm}{1cm}
 %\end{textblock}
 \textblockcolor{bluesite}
 \begin{textblock}{5}(-1.5,-0.55)
  \rule{0cm}{1cm}
 \end{textblock}
 %\textblockcolor{bluesite}
 %\begin{textblock}{14}(5.5,-0.55)
 % \rule{0cm}{1cm}
 %\end{textblock}
 \begin{textblock}{0}(-1,-0.25)
 \color{cwhite} \begin{Large}www.math24.gr\end{Large}
 \end{textblock}
\begin{textblock}{0}(16,-0.4)
 \color{cwhite} \includegraphics[height=1.5cm]{math24_logo.png}
 \end{textblock}
\textblockcolor{white}
\begin{textblock}{17}(-1,28)
 \color{backsite} \begin{small}Copyright \textcopyright 2013, Κασωτάκη Ε.(ikasotaki@gmail.com) - Σμαραγδάκης Κ.(kesmarag@gmail.com)\end{small}
 \end{textblock}
}
 
\begin{shaded}
\begin{center}
\huge \textbf{Μαθηματικά Α' Γυμνασίου}\\
\end{center} 
\begin{center}
\textbf{Επαναληπτικές ερωτήσεις θεωρίας} \hfill \textbf{}
\end{center}
%\subsection*{}
\end{shaded}
\vspace{2em}

\begin{itemize}
\item 1.3: Δυνάμεις φυσικών αριθμών
\item 1.4: Ευκλείδεια διαίρεση - διαιρετότητα
\item 1.5: Χαρακτήρες διαιρετότητας - ΜΚΔ - ΕΚΠ - Ανάλυση αριθμού σε γινόμενο πρώτων παραγόντων
\item Κεφάλαιο 2: Τα κλάσματα
\item Κεφάλαιο 5: Ποσοστά
\item 6.3: Ανάλογα ποσά - Ιδιότητες ανάλογων ποσών
\item 6.5: Προβλήματα αναλογιών
\item 6.6: Αντιστρόφως ανάλογα ποσά
\item Β.1.3: Μονάδες μήκους – απόσταση σημείων – μέσο ευθύγραμμου τμήματος 
\item Β.1.5: Διχοτόμος γωνίας 
\item Β.1.6: Είδη γωνιών – Κάθετες γωνίες 
\item Β.1.7: Εφεξής – διαδοχικές γωνίες 
\item Β.1.8: Παραπληρωματικές και συμπληρωματικές γωνίες – κατακορυφήν 
\item Β.1.9: Θέσεις ευθειών στο επίπεδο 
\item Β.1.10: Απόσταση σημείου από ευθεία – Απόσταση παραλλήλων 
\item Β.2.3: Μεσοκάθετος ευθύγραμμου τμήματος 
\item Β.2.6: Παράλληλες ευθείες που τέμνονται από μία άλλη ευθεία
\item Β.3.1: Στοιχεία τριγώνου – Είδη τριγώνων 
\item Β.3.2: Άθροισμα γωνιών τριγώνου – Ιδιότητες ισοσκελούς τριγώνου 
\end{itemize}

\section*{Επαναληπτικές Ερωτήσεις Θεωρίας \hfill \small{}}
\begin{enumerate}
\item Τι ονομάζεται Ελάχιστο Κοινό Πολλαπλάσιο (ΕΚΠ) δύο ή περισσότερων αριθμών;
\item Ποιοι αριθμοί ονομάζονται πρώτοι; 
\item Τι ονομάζεται Μέγιστος Κοινός Διαιρέτης (ΜΚΔ) δύο αριθμών;
\item Πότε δύο αριθμοί λέγονται πρώτοι μεταξύ τους;
\item Πότε ένας φυσικός αριθμός διαιρείται με το $10,100,1000,\cdots$;
%-------------------------------------------------------------05
\item Πότε ένας φυσικός αριθμός διαιρείται με το $2$;
\item Πότε ένας φυσικός αριθμός διαιρείται με το $5$;
\item Πότε ένας φυσικός αριθμός διαιρείται με το $3$;
\item Πότε ένας φυσικός αριθμός διαιρείται με το $9$;
\item Πότε ένας φυσικός αριθμός διαιρείται με το $4$;
%--------------------------------------------------------10
\item Πότε ένας φυσικός αριθμός διαιρείται με το $25$;
\item Πότε δύο κλάσματα λέγονται ισοδύναμα;
\item *Τι ονομάζουμε απλοποίηση κλάσματος; 
\item Πότε ένα κλάσμα λέγεται ανάγωγο;
\item Πότε δύο ή περισσότερα κλάσματα λέγονται ομώνυμα;
%------------------------------------------------------15
\item Πότε δύο ή περισσότερα κλάσματα λέγονται ετερώνυμα;
\item Πότε ένα κλάσμα είναι μεγαλύτερο από το $1$;
\item Πότε ένα κλάσμα είναι μικρότερο από το $1$;
\item Πότε ένα κλάσμα ισούται με $1$;
\item Αν δύο κλάσματα έχουν τον ίδιο παρονομαστή (είναι ομώνυμα) ποιο είναι μεγαλύτερο;
%-------------------------------------------------------20
\item Αν δύο κλάσματα έχουν τον ίδιο αριθμητή ποιο είναι μεγαλύτερο;
\item Πώς συγκρίνουμε δύο ετερώνυμα κλάσματα;
\item *Πώς προσθέτουμε δύο ή περισσότερα ομώνυμα κλάσματα;
\item *Πώς προσθέτουμε ετερώνυμα κλάσματα;
\item *Πώς αφαιρούμε δύο ομώνυμα κλάσματα;
%--------------------------------------------------------25
\item *Πώς αφαιρούμε δύο ετερώνυμα κλάσματα;
\item *Πώς υπολογίζουμε το γινόμενο δύο κλασμάτων;
\item *Πώς υπολογίζουμε το γινόμενο ενός φυσικού αριθμού επί ένα κλάσμα;
\item Πότε δύο κλάσματα λέγονται αντίστροφα;
\item Πότε δύο αριθμοί λέγονται αντίστροφοι;
%-------------------------------------------------------30
\item *Πώς υπολογίζουμε το πηλίκο δύο κλασμάτων;
\item Τι ονομάζονται σύνθετο κλάσμα;
\item *Πώς ονομάζεται και με τι ισούται το σύμβολο $α\%$;
\item *Πώς υπολογίζουμε το ποσοστό $α\%$ ενός αριθμού $β$;
\item Πότε δύο ποσά λέγονται ανάλογα;
%------------------------------------------------------35
\item Τι ονομάζουμε συντελεστή αναλογίας δύο ανάλογων ποσών;
\item Πότε δύο μεγέθη ονομάζονται αντιστρόφως ανάλογα;
\item Ποια είναι η μονάδα μέτρησης του μήκους και ποιες είναι οι υποδιαιρέσεις και ποια τα πολλαπλάσιά της;
\item *Τι ονομάζουμε απόσταση δύο σημείων;
\item Τι ονομάζουμε μέσο ενός ευθύγραμμου τμήματος;
%------------------------------------------------------40
\item *Ποια είναι η μονάδα μέτρησης της γωνίας;
\item Πότε δύο γωνίες είναι ίσες;
\item Τι ονομάζουμε διχοτόμος μιας γωνίας;
\item Ποια γωνία ονομάζεται ορθή;
\item Ποια γωνία ονομάζεται οξεία;
%--------------------------------------------------------------45
\item Ποια γωνία ονομάζεται αμβλεία;
\item Ποια γωνία ονομάζεται ευθεία;
\item Ποια γωνία ονομάζεται κυρτή;
\item Ποια γωνία ονομάζεται μηδενική;
\item Ποια γωνία ονομάζεται πλήρης;
%-------------------------------------------------------------50
\item Πότε δύο ευθείες είναι κάθετες;
\item *Πότε δύο ευθύγραμμα τμήματα λέμε ότι είναι κάθετα;
\item *Πότε δύο ημιευθείες λέμε ότι είναι κάθετες;
\item Πότε δύο γωνίες ονομάζονται εφεξής;
\item Πότε δύο γωνίες ονομάζονται διαδοχικές;
%-------------------------------------------------------------55
\item Πότε δύο γωνίες ονομάζονται παραπληρωματικές;
\item Πότε δύο γωνίες ονομάζονται συμπληρωματικές;
\item Ποιες γωνίες ονομάζονται κατακορυφήν;
\item Πότε δύο ευθείες είναι παράλληλες;
\item Πόε δύο ευθείες ονομάζονται τεμνόμενες;
%---------------------------------------------------------------60
\item *Πότε δύο ευθύγραμμα τμήματα λέγονται παράλληλα;
\item *Τι ονομάζουμε απόσταση ενός σημείου $Α$ από μία ευθεία $ε$;
\item Τι ονομάζουμε απόσταση δύο παραλλήλων ευθειών;
\item Τι ονομάζουμε μεσοκάθετο ευθυγράμμου τμήματος;
\item *Για τις παράλληλες ευθείες $ε_{1}$ και $ε_{2}$ που τέμνονται από μία τρίτη ευθεία $δ$,
       ποιες γωνίες ονομάζονται "εντός", ποιες "εκτός", ποιες "επί τα αυτά" και ποιες "εναλλάξ";
%-----------------------------------------------------------------65
\item Ποια είναι τα κύρια στοιχεία ενός τριγώνου;
\item Πότε ένα τρίγωνο ονομάζεται ορθογώνιο;
\item Πότε ένα τρίγωνο ονομάζεται αμβλυγώνιο;
\item Πότε ένα τρίγωνο ονομάζεται οξυγώνιο;
\item Πότε ένα τρίγωνο ονομάζεται ισόπλευρο;
%---------------------------------------------------------------70
\item Πότε ένα τρίγωνο ονομάζεται ισοσκελές;
\item Πότε ένα τρίγωνο ονομάζεται σκαληνό;
\item *Ποια είναι τα δευτερεύοντα στοιχεία ενός τριγώνου;
\item Τι ονομάζουμε διάμεσο ενός τριγώνου;
\item Τι ονομάζουμε ύψος τριγώνου;
%---------------------------------------------------------------75
\item Τι ονομάζουμε διχοτόμο του τριγώνου;
\item Με τι ισούται το άθροισμα των γωνιών κάθε τριγώνου;
\end{enumerate}



\end{document}
