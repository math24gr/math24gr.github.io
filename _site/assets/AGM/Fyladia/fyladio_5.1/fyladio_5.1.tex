
\documentclass[a4paper,10pt]{report}

%\usepackage[landscape]{geometry}
\usepackage[cm-default]{fontspec}
\setromanfont{FreeSerif}
\setsansfont{FreeSans}
\setmonofont{FreeMono}
\usepackage[utf8x]{inputenc}
\usepackage{fontspec}
\usepackage{xunicode}
\usepackage{xltxtra}
\usepackage{xgreek}
\usepackage{amsmath}
\usepackage{unicode-math}
\usepackage{ulem}
\usepackage{color}
\usepackage{verbatim}
\usepackage{nopageno}
\usepackage{graphicx}
\usepackage{textpos}
\setlength{\TPHorizModule}{1cm}
\setlength{\TPVertModule}{1cm}
%\usepackage[colorgrid,texcoord]{eso-pic}
\usepackage[outline]{contour}
\usepackage{wrapfig}
\usepackage{url}
\usepackage{color}
\usepackage{tikz}
\usepackage{fancybox,fancyhdr}
\usepackage{subfigure}
\usepackage{pstricks}
\usepackage{epsfig}
\usepackage{multicol}
\usepackage{listings}
\usepackage{enumerate}
\usepackage{hyperref}
\hypersetup{
  bookmarks=true,
  bookmarksopen=true,
  pdfborder=false,
  pdfpagemode=UseNone,
  raiselinks=true,
  pdfhighlight={/P},
  colorlinks,
  citecolor=black,
  filecolor=black,
  linkcolor=black,
  urlcolor=black
}


\usepackage{fontspec}
\usepackage{xunicode}
\usepackage{xltxtra}


% Margins
%\setlength{\textwidth}{24cm}

%\setlength{\voffset}{0in}
%\setlength{\textheight}{6.8in}
% Colors
\definecolor{cbrown}{rgb}{0.49,0.24,0.07}
\definecolor{cmpez}{rgb}{0.92,0.88,0.79}
\definecolor{cmpez2}{rgb}{0.62,0.33,0.34}
\definecolor{cwhite}{rgb}{1,1,1}
\definecolor{cblack}{rgb}{0,0,0}
\definecolor{cred}{rgb}{0.9,0.15,0.15}
\definecolor{clblue}{rgb}{0.57,0.85,0.97}
\definecolor{corange}{rgb}{0.86,0.49,0.18}
\definecolor{clorange}{rgb}{0.95,0.84,0.61}
\definecolor{cyellow}{rgb}{1,0.95,0.16}
\definecolor{cgreen}{rgb}{0.69,0.8,0.31}
\definecolor{clbrown}{rgb}{0.78,0.67,0.43}
\definecolor{cmagenta}{rgb}{0.79,0.05,0.54}
\definecolor{cgray}{rgb}{0.76,0.73,0.63}
\definecolor{bluesite}{rgb}{0.05,0.43,0.67}
\definecolor{redsite}{rgb}{0.90,0.15,0.15}
\definecolor{backsite}{rgb}{0.07,0.07,0.07}
% ----------- from Costas -----------------------------

%---------------   Main Fonts    -----------------%   

% \setmainfont[Mapping=tex-text]{Calibri}
% \setmainfont[Mapping=tex-text]{Times New Roman}
%\setmainfont[Mapping=tex-text]{Droid Serif}
%\setmainfont[Mapping=tex-text]{Cambria}
%\setmainfont[Mapping=tex-text]{cm-unicode}
% \setmainfont[Mapping=tex-text]{Gentium}
% \setmainfont[Mapping=tex-text]{GFS Didot}
% \setmainfont[Mapping=tex-text]{Comic Sans MS}
 \setmainfont[Mapping=tex-text]{Ubuntu}
% \setmainfont[Mapping=tex-text]{Myriad Pro}
% \setmainfont[Mapping=tex-text]{CMU Concrete}
%\setmainfont[Mapping=tex-text]{DejaVu Sans}
%\setmainfont[Mapping=tex-text]{KerkisSans}
%\setmainfont[Mapping=tex-text]{KerkisCaligraphic}
% \setmainfont[Mapping=tex-text]{Segoe Print}
% \setmainfont[Mapping=tex-text]{Gabriola}

%------------------------------------------------%

% ------------ Mathematics Fonts ----------------%
\setmathfont{Asana-Math.ttf}
% -----------------------------------------------%
% Misc
\author{Κασωτάκη Ε. - Σμαραγδάκης Κ.}
\title{www.math24.gr}
% ---------------- Formation --------------------%


\setlength\topmargin{-0.7cm}
\addtolength\topmargin{-\headheight}
\addtolength\topmargin{-\headsep}
\setlength\textheight{26cm}
\setlength\oddsidemargin{-0.54cm}
\setlength\evensidemargin{-0.54cm}
%\setlength\marginparwidth{1.5in}
\setlength\textwidth{17cm}

\RequirePackage[avantgarde]{quotchap}
\renewcommand\chapterheadstartvskip{\vspace*{0\baselineskip}}
\RequirePackage[calcwidth]{titlesec}
\titleformat{\section}[hang]{\bfseries}
{\Large\thesection}{12pt}{\Large}[{\titlerule[0.9pt]}]
%--------------------------------------------------%
\usepackage{draftwatermark}
\SetWatermarkText{www.math24.gr}
\SetWatermarkLightness{0.9}
\SetWatermarkScale{0.6}
%--------------------------------------------------%


\usepackage{xcolor}
\usepackage{amsthm}
\usepackage{framed}
\usepackage{parskip}

\colorlet{shadecolor}{bluesite!20}

\newtheorem*{orismos}{Ορισμός}

\newenvironment{notation}
  {\begin{shaded}\begin{theorem}}
  {\end{theorem}\end{shaded}}




% Document begins
\begin{document}
\pagestyle{fancy}
\fancyhead{}
\fancyfoot{}
\renewcommand{\headrulewidth}{0pt}
\renewcommand{\footrulewidth}{0pt}


\fancyhead[LO,LE]{
 %\textblockcolor{backsite}
 %\begin{textblock}{5}(-2,-0.55)
  %\rule{0cm}{1cm}
 %\end{textblock}
 \textblockcolor{bluesite}
 \begin{textblock}{5}(-1.5,-0.55)
  \rule{0cm}{1cm}
 \end{textblock}
 %\textblockcolor{bluesite}
 %\begin{textblock}{14}(5.5,-0.55)
 % \rule{0cm}{1cm}
 %\end{textblock}
 \begin{textblock}{0}(-1,-0.25)
 \color{cwhite} \begin{Large}www.math24.gr\end{Large}
 \end{textblock}
\begin{textblock}{0}(16,-0.4)
 \color{cwhite} \includegraphics[height=1.5cm]{math24_logo.png}
 \end{textblock}
\textblockcolor{white}
\begin{textblock}{17}(-1,28)
 \color{backsite} \begin{small}Copyright \textcopyright 2012, Κασωτάκη Ε.(ikasotaki@gmail.com) - Σμαραγδάκης Κ.(kesmarag@gmail.com)\end{small}
 \end{textblock}
}
 
\begin{shaded}
\begin{center}
\huge \textbf{Φυλλάδιο Ασκήσεων}\\
%Πρόσθεση ρητών αριθμών
\end{center} 
\textbf{Μαθηματικά Α' Γυμνασίου} \hfill \textbf{Ημερομηνία Παράδοσης : \hspace{2em} }
\subsection*{Ονοματεπώνυμο :\hfill  \hspace{5em}}
\end{shaded}
\vspace{2em}
\begin{itemize}
 \item Ποσοστό επί τοις εκατό
 \item Μετατροπή ενός δεκαδικού αριθμού σε ποσοστό επί τοις εκατό
 \item Μετατροπή ενός  κλάσματος σε ποσοστό επί τοις εκατό
 \item Μετατροπή ενός  ποσοστού σε δεκαδικό κλάσμα 
 \item Μετατροπή ενός  ποσοστού σε δεκαδικό αριθμό 
\end{itemize}

\section*{Παράδειγμα \hfill \small{}}
\textbf{Μετατροπή του δεκαδικού αριθμού $0,08$ σε ποσοστό επί τοις εκατό }\\
$0,08 = 0,08\cdot \dfrac{100}{100} = \dfrac{8}{100} =8\%$




\section*{Άσκηση 1  \hfill \small{25 μονάδες}}
Να γράψετε ως ποσοστό επί τοις εκατό τους παρακάτω δεκαδικούς αριθμούς:
\begin{enumerate}[1)]
 \item $0,05$
 \item $0,13$
 \item $0,42$
 \item $0,67$
 \item $0,09$
\end{enumerate}


\section*{Παράδειγμα \hfill \small{}}
\textbf{Μετατροπή του κλάσματος $\dfrac{2}{5}$ σε ποσοστό επί τοις εκατό }
\begin{itemize}
 \item \textbf{α' τρόπος:} $\dfrac{2}{5}=\dfrac{2\cdot 20}{5\cdot 20}=\dfrac{40}{100}=40\%$
 \item \textbf{β' τρόπος:} $\dfrac{2}{5}=0,4=40\%$
\end{itemize}



\section*{Άσκηση 2  \hfill \small{25 μονάδες}}
Να γράψετε ως ποσοστό επί τοις εκατό τα παρακάτω κλάσματα:
\begin{enumerate}[1)]
 \item $\dfrac{37}{100}$
 \item $\dfrac{4}{5}$
 \item $\dfrac{3}{20}$
 \item $\dfrac{3}{4}$
 \item $\dfrac{5}{8}$
\end{enumerate}


\section*{Παράδειγμα \hfill \small{}}
\textbf{Μετατροπή του  ποσοστού $13\%$ σε δεκαδικό κλάσμα }\\
$13\% =\dfrac{13}{100}$



\section*{Άσκηση 3  \hfill \small{25 μονάδες}}
Να μετατρέψετε τα παρακάτω ποσοστά επί τοις εκατό σε δεκαδικά κλάσματα:
\begin{enumerate}[1)]
 \item $19\%$
 \item $23\%$
 \item $7\%$
 \item $1\%$
 \item $95\%$
\end{enumerate}


\section*{Παράδειγμα \hfill \small{}}
\textbf{Μετατροπή του  ποσοστού $20\%$ σε δεκαδικό αριθμό }\\
$20\% =20\cdot \dfrac{1}{100} =\dfrac{20}{100}=0,2$




\section*{Άσκηση 4  \hfill \small{25 μονάδες}}
Να μετατρέψετε τα παρακάτω  ποσοστά επί τοις εκατό σε δεκαδικούς αριθμούς:
\begin{enumerate}[1)]
 \item $51\%$
 \item $62\%$
 \item $71\%$
 \item $8\%$
 \item $9\%$
\end{enumerate}

%\let\thefootnote\relax\footnotetext{Επιμέλεια : Κασωτάκη Ειρήνη (ikasotaki@gmail.com),    Σμαραγδάκης Κώστας (kesmarag@gmail.com)} 


\end{document}
