
\documentclass[a4paper,10pt]{report}

%\usepackage[landscape]{geometry}
\usepackage[cm-default]{fontspec}
\setromanfont{FreeSerif}
\setsansfont{FreeSans}
\setmonofont{FreeMono}
\usepackage[utf8x]{inputenc}
\usepackage{fontspec}
\usepackage{xunicode}
\usepackage{xltxtra}
\usepackage{xgreek}
\usepackage{amsmath}
\usepackage{unicode-math}
\usepackage{ulem}
\usepackage{color}
\usepackage{verbatim}
\usepackage{nopageno}
\usepackage{graphicx}
\usepackage{textpos}
\setlength{\TPHorizModule}{1cm}
\setlength{\TPVertModule}{1cm}
%\usepackage[colorgrid,texcoord]{eso-pic}
\usepackage[outline]{contour}
\usepackage{wrapfig}
\usepackage{url}
\usepackage{color}
\usepackage{tikz}
\usepackage{fancybox,fancyhdr}
\usepackage{subfigure}
\usepackage{pstricks}
\usepackage{epsfig}
\usepackage{multicol}
\usepackage{listings}
\usepackage{enumerate}
\usepackage{hyperref}
\hypersetup{
  bookmarks=true,
  bookmarksopen=true,
  pdfborder=false,
  pdfpagemode=UseNone,
  raiselinks=true,
  pdfhighlight={/P},
  colorlinks,
  citecolor=black,
  filecolor=black,
  linkcolor=black,
  urlcolor=black
}


\usepackage{fontspec}
\usepackage{xunicode}
\usepackage{xltxtra}


% Margins
%\setlength{\textwidth}{24cm}

%\setlength{\voffset}{0in}
%\setlength{\textheight}{6.8in}
% Colors
\definecolor{cbrown}{rgb}{0.49,0.24,0.07}
\definecolor{cmpez}{rgb}{0.92,0.88,0.79}
\definecolor{cmpez2}{rgb}{0.62,0.33,0.34}
\definecolor{cwhite}{rgb}{1,1,1}
\definecolor{cblack}{rgb}{0,0,0}
\definecolor{cred}{rgb}{0.9,0.15,0.15}
\definecolor{clblue}{rgb}{0.57,0.85,0.97}
\definecolor{corange}{rgb}{0.86,0.49,0.18}
\definecolor{clorange}{rgb}{0.95,0.84,0.61}
\definecolor{cyellow}{rgb}{1,0.95,0.16}
\definecolor{cgreen}{rgb}{0.69,0.8,0.31}
\definecolor{clbrown}{rgb}{0.78,0.67,0.43}
\definecolor{cmagenta}{rgb}{0.79,0.05,0.54}
\definecolor{cgray}{rgb}{0.76,0.73,0.63}
\definecolor{bluesite}{rgb}{0.05,0.43,0.67}
\definecolor{redsite}{rgb}{0.90,0.15,0.15}
\definecolor{backsite}{rgb}{0.07,0.07,0.07}
% ----------- from Costas -----------------------------

%---------------   Main Fonts    -----------------%   

% \setmainfont[Mapping=tex-text]{Calibri}
% \setmainfont[Mapping=tex-text]{Times New Roman}
%\setmainfont[Mapping=tex-text]{Droid Serif}
%\setmainfont[Mapping=tex-text]{Cambria}
%\setmainfont[Mapping=tex-text]{cm-unicode}
% \setmainfont[Mapping=tex-text]{Gentium}
% \setmainfont[Mapping=tex-text]{GFS Didot}
% \setmainfont[Mapping=tex-text]{Comic Sans MS}
 \setmainfont[Mapping=tex-text]{Ubuntu}
% \setmainfont[Mapping=tex-text]{Myriad Pro}
% \setmainfont[Mapping=tex-text]{CMU Concrete}
%\setmainfont[Mapping=tex-text]{DejaVu Sans}
%\setmainfont[Mapping=tex-text]{KerkisSans}
%\setmainfont[Mapping=tex-text]{KerkisCaligraphic}
% \setmainfont[Mapping=tex-text]{Segoe Print}
% \setmainfont[Mapping=tex-text]{Gabriola}

%------------------------------------------------%

% ------------ Mathematics Fonts ----------------%
\setmathfont{Asana-Math.ttf}
% -----------------------------------------------%
% Misc
\author{Κασωτάκη Ε. - Σμαραγδάκης Κ.}
\title{www.math24.gr}
% ---------------- Formation --------------------%


\setlength\topmargin{-0.7cm}
\addtolength\topmargin{-\headheight}
\addtolength\topmargin{-\headsep}
\setlength\textheight{26cm}
\setlength\oddsidemargin{-0.54cm}
\setlength\evensidemargin{-0.54cm}
%\setlength\marginparwidth{1.5in}
\setlength\textwidth{17cm}

\RequirePackage[avantgarde]{quotchap}
\renewcommand\chapterheadstartvskip{\vspace*{0\baselineskip}}
\RequirePackage[calcwidth]{titlesec}
\titleformat{\section}[hang]{\bfseries}
{\Large\thesection}{12pt}{\Large}[{\titlerule[0.9pt]}]
%--------------------------------------------------%
\usepackage{draftwatermark}
\SetWatermarkText{www.math24.gr}
\SetWatermarkLightness{0.9}
\SetWatermarkScale{0.6}
%--------------------------------------------------%


\usepackage{xcolor}
\usepackage{amsthm}
\usepackage{framed}
\usepackage{parskip}

\colorlet{shadecolor}{bluesite!20}

\newtheorem*{orismos}{Ορισμός}

\newenvironment{notation}
  {\begin{shaded}\begin{theorem}}
  {\end{theorem}\end{shaded}}




% Document begins
\begin{document}
\pagestyle{fancy}
\fancyhead{}
\fancyfoot{}
\renewcommand{\headrulewidth}{0pt}
\renewcommand{\footrulewidth}{0pt}


\fancyhead[LO,LE]{
 %\textblockcolor{backsite}
 %\begin{textblock}{5}(-2,-0.55)
  %\rule{0cm}{1cm}
 %\end{textblock}
 \textblockcolor{bluesite}
 \begin{textblock}{5}(-1.5,-0.55)
  \rule{0cm}{1cm}
 \end{textblock}
 %\textblockcolor{bluesite}
 %\begin{textblock}{14}(5.5,-0.55)
 % \rule{0cm}{1cm}
 %\end{textblock}
 \begin{textblock}{0}(-1,-0.25)
 \color{cwhite} \begin{Large}www.math24.gr\end{Large}
 \end{textblock}
\begin{textblock}{0}(16,-0.4)
 \color{cwhite} \includegraphics[height=1.5cm]{math24_logo.png}
 \end{textblock}
\textblockcolor{white}
\begin{textblock}{17}(-1,28)
 \color{backsite} \begin{small}Copyright \textcopyright 2011, Κασωτάκη Ε.(ikasotaki@gmail.com) - Σμαραγδάκης Κ.(kesmarag@gmail.com)\end{small}
 \end{textblock}
}
 
\begin{shaded}
\begin{center}
\huge \textbf{Φυλλάδιο Ασκήσεων}\\
%Πρόσθεση ρητών αριθμών
\end{center} 
\textbf{Μαθηματικά Α' Γυμνασίου} \hfill \textbf{Ημερομηνία Παράδοσης : \hspace{2em} }
\subsection*{Ονοματεπώνυμο :\hfill  \hspace{5em}}
\end{shaded}
\vspace{2em}
\begin{itemize}
 \item Η έννοια του κλάσματος
 \item Σύγκριση κλασμάτων με τη μονάδα
 \item Υπολογισμός του μέρους από το όλο (με τη μέθοδο αναγωγής  στη μονάδα)
 \item Υπολογισμός του όλου από το ένα μέρος (με τη μέθοδο αναγωγής  στη μονάδα)
\end{itemize}
\section*{Θεωρία - Η έννοια του κλάσματος\hfill \small{}}
$$ \text{κλάσμα}:\dfrac{\text{πόσα μέρη πήραμε}}{\text{σε πόσα ίσα τμήματα χωρίσαμε}}:\dfrac{\text{αριθμητής}}{\text{παρονομαστής}} $$
\textbf{π.χ} οι $11$ μήνες ενός χρόνου είναι τα $\dfrac{11}{12}$ του χρόνου

\section*{Άσκηση 1  (Η έννοια του κλάσματος)\hfill \small{10 μονάδες}}
Γνωρίζουμε ότι μια ημέρα έχει 24 ώρες. Να βρείτε ποιο μέρος της ημέρας είναι 
\begin{enumerate}[i)]
 \item οι 2 ώρες
 \item οι 4 ώρες
 \item οι 8 ώρες 
 \item οι 6 ώρες 
 \item οι 12 ώρες
\end{enumerate}

\section*{Θεωρία - Σύγκριση κλασμάτων με τη μονάδα \hfill \small{}}
\begin{itemize}
 \item \textbf{Αν} ο αριθμητής ενός κλάσματος είναι μικρότερος από τον παρονομαστή 
        \textbf{τότε} το κλάσμα είναι μικρότερο από τη μονάδα.\\
        \textbf{π.χ} $\dfrac{2}{5}<1$ αφού $2<5$
 \item \textbf{Αν} ο αριθμητής ενός κλάσματος είναι μεγαλύτερος από τον παρονομαστή 
        \textbf{τότε} το κλάσμα είναι μεγαλύτερο από τη μονάδα.\\
        \textbf{π.χ} $\dfrac{8}{3}>1$ αφού $8<3$
 \item \textbf{Αν} ο αριθμητής ενός κλάσματος είναι ίσος με τον παρονομαστή 
        \textbf{τότε} το κλάσμα ισούται με τη μονάδα.\\
        \textbf{π.χ} $\dfrac{4}{4}=1$ αφού $4=4$
\end{itemize}





\section*{Άσκηση 2 (Σύγκριση κλασμάτων με τη μονάδα) \hfill \small{10 μονάδες}}
Να βρείτε ποια από τα παρακάτω κλάσματα είναι μικρότερα, ποια μεγαλύτερα της μονάδα; και ποια είναι ίσα με τη μονάδα;
%\begin{multicols}{2}
\begin{enumerate}[1)]
%\begin{itemize}
 \item $\dfrac{1}{5}$
 \item $\dfrac{2}{8}$
 \item $\dfrac{8}{8}$
 \item $\dfrac{7}{3}$
 \item $\dfrac{7}{7}$
 \item $\dfrac{3}{6}$
 \item $\dfrac{2}{3}$
 \item $\dfrac{3}{3}$
 \item $\dfrac{5}{16}$
 \item $\dfrac{7}{2}$
%\end{itemize}
\end{enumerate}
%\end{multicols}

\section*{Θεωρία - Υπολογισμός του μέρους από το όλο \hfill \small{}}
Για να βρούμε την τιμή του μέρους ξεκινάμε από την τιμή του όλου που είναι η τιμή της μονάδας.
\begin{itemize}
 \item \textbf{Παράδειγμα 1}: Μία δεξαμενή χωράει $1500lt$. Πόσα $lt$ χωράει το $\dfrac{1}{3}$ της;\\
       Τα $\dfrac{3}{3}$ της δεξαμενής χωράνε $1500lt$.\\
       Το $\dfrac{1}{3}$ της δεξαμενής χωράει $\dfrac{1}{3}\cdot 1500lt=500lt$.\\
 \item \textbf{Παράδειγμα 2}: Μία δεξαμενή χωράει $1500lt$. Πόσα $lt$ χωράει το $\dfrac{2}{3}$ της;\\
       Τα $\dfrac{3}{3}$ της δεξαμενής χωράνε $1500lt$.\\
       Το $\dfrac{1}{3}$ της δεξαμενής χωράει $\dfrac{1}{3}\cdot 1500lt=500lt$.\\
       Άρα τα $\dfrac{2}{3}$ της δεξαμενής χωράνε $2\cdot 500lt=1000lt$.
\end{itemize}


\section*{Άσκηση 3 (Υπολογισμός του μέρους από το όλο)\hfill \small{20 μονάδες}}
Ένα πακέτο χωράει $20$ τσίχλες. Κάποια στιγμή βρήκαμε ότι το πακέτο ήταν γεμάτο κατά τα $\dfrac{3}{4}$. 
Πόσες τσίχλες περιείχε τότε το πακέτο;


\section*{Άσκηση 4 (Υπολογισμός του μέρους από το όλο)\hfill \small{20 μονάδες}}
Μία παπουτσοθήκη χωράει $30$ ζευγάρια παπούτσια. Πόσα παπούτσια χωράνε τα  $\dfrac{5}{6}$; 



\section*{Θεωρία - Υπολογισμός του όλου από το ένα μέρος \hfill \small{}}
Για να βρούμε την τιμή του όλου ξεκινάμε από την τιμή του μέρος και υπολογίζουμε  την τιμή της μονάδας 
(αναγωγή στη μονάδα).
\begin{itemize}
 \item \textbf{Παράδειγμα 1}: Τα $\dfrac{3}{4}$ του κιλού τυρί κοστίζουν $9\texteuro$. Πόσο κοστίζει το ένα κιλό;\\
       Τα $\dfrac{3}{4}$ κοστίζουν $9\texteuro$, άρα το $\dfrac{1}{4}$ κοστίζει $9\texteuro:3=3\texteuro$ .\\
       Το ένα κιλό, δηλαδή τα $\dfrac{4}{4}$ κοστίζει $4\cdot 3\texteuro=12\texteuro$.\\
 \item \textbf{Παράδειγμα 2}: Τα $\dfrac{3}{4}$ του κιλού τυρί κοστίζουν $9\texteuro$. 
        Πόσο κοστίζουν τα $\dfrac{5}{6}$ του κιλού;   \\
       Τα $\dfrac{3}{4}$ κοστίζουν $9\texteuro$, άρα το $\dfrac{1}{4}$ κοστίζει $9\texteuro:3=3\texteuro$ .\\
       Το ένα κιλό, δηλαδή τα $\dfrac{4}{4}$ κοστίζει $4\cdot 3\texteuro=12\texteuro$.\\
       Το ένα κιλό, δηλαδή τα $\dfrac{6}{6}$ κοστίζει $12\texteuro$.\\
       Το $\dfrac{1}{6}$ του κιλού κοστίζει $12\texteuro :6=2\texteuro$.\\
       Άρα τα $\dfrac{5}{6}$ του κιλού κοστίζουν $5\cdot 2\texteuro=10\texteuro$
\end{itemize}

\section*{Άσκηση 5 (Υπολογισμός του όλου από το ένα μέρος)\hfill \small{20 μονάδες}}
Τα $\dfrac{2}{6}$ του κιλού μοσχάρι κοστίζουν $4\texteuro.$ Πόσο κοστίζουν τα $\dfrac{2}{3}$ του κιλού;


\section*{Άσκηση 6 (Υπολογισμός του όλου από το ένα μέρος)\hfill \small{20 μονάδες}}
Τα $\dfrac{4}{5}$ ενός ντεπόζιτου χωράνε  $800lt$. Πόσα $lt$ χωράνε τα $\dfrac{9}{10}$;


\end{document}
