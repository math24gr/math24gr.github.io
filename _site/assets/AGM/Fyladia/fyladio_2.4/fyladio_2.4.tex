
\documentclass[a4paper,10pt]{report}

%\usepackage[landscape]{geometry}
\usepackage[cm-default]{fontspec}
\setromanfont{FreeSerif}
\setsansfont{FreeSans}
\setmonofont{FreeMono}
\usepackage[utf8x]{inputenc}
\usepackage{fontspec}
\usepackage{xunicode}
\usepackage{xltxtra}
\usepackage{xgreek}
\usepackage{amsmath}
\usepackage{unicode-math}
\usepackage{ulem}
\usepackage{color}
\usepackage{verbatim}
\usepackage{nopageno}
\usepackage{graphicx}
\usepackage{textpos}
\setlength{\TPHorizModule}{1cm}
\setlength{\TPVertModule}{1cm}
%\usepackage[colorgrid,texcoord]{eso-pic}
\usepackage[outline]{contour}
\usepackage{wrapfig}
\usepackage{url}
\usepackage{color}
\usepackage{tikz}
\usepackage{fancybox,fancyhdr}
\usepackage{subfigure}
\usepackage{pstricks}
\usepackage{epsfig}
\usepackage{multicol}
\usepackage{listings}
\usepackage{enumerate}
\usepackage{hyperref}
\hypersetup{
  bookmarks=true,
  bookmarksopen=true,
  pdfborder=false,
  pdfpagemode=UseNone,
  raiselinks=true,
  pdfhighlight={/P},
  colorlinks,
  citecolor=black,
  filecolor=black,
  linkcolor=black,
  urlcolor=black
}


\usepackage{fontspec}
\usepackage{xunicode}
\usepackage{xltxtra}


% Margins
%\setlength{\textwidth}{24cm}

%\setlength{\voffset}{0in}
%\setlength{\textheight}{6.8in}
% Colors
\definecolor{cbrown}{rgb}{0.49,0.24,0.07}
\definecolor{cmpez}{rgb}{0.92,0.88,0.79}
\definecolor{cmpez2}{rgb}{0.62,0.33,0.34}
\definecolor{cwhite}{rgb}{1,1,1}
\definecolor{cblack}{rgb}{0,0,0}
\definecolor{cred}{rgb}{0.9,0.15,0.15}
\definecolor{clblue}{rgb}{0.57,0.85,0.97}
\definecolor{corange}{rgb}{0.86,0.49,0.18}
\definecolor{clorange}{rgb}{0.95,0.84,0.61}
\definecolor{cyellow}{rgb}{1,0.95,0.16}
\definecolor{cgreen}{rgb}{0.69,0.8,0.31}
\definecolor{clbrown}{rgb}{0.78,0.67,0.43}
\definecolor{cmagenta}{rgb}{0.79,0.05,0.54}
\definecolor{cgray}{rgb}{0.76,0.73,0.63}
\definecolor{bluesite}{rgb}{0.05,0.43,0.67}
\definecolor{redsite}{rgb}{0.90,0.15,0.15}
\definecolor{backsite}{rgb}{0.07,0.07,0.07}
% ----------- from Costas -----------------------------

%---------------   Main Fonts    -----------------%   

% \setmainfont[Mapping=tex-text]{Calibri}
% \setmainfont[Mapping=tex-text]{Times New Roman}
%\setmainfont[Mapping=tex-text]{Droid Serif}
%\setmainfont[Mapping=tex-text]{Cambria}
%\setmainfont[Mapping=tex-text]{cm-unicode}
% \setmainfont[Mapping=tex-text]{Gentium}
% \setmainfont[Mapping=tex-text]{GFS Didot}
% \setmainfont[Mapping=tex-text]{Comic Sans MS}
 \setmainfont[Mapping=tex-text]{Ubuntu}
% \setmainfont[Mapping=tex-text]{Myriad Pro}
% \setmainfont[Mapping=tex-text]{CMU Concrete}
%\setmainfont[Mapping=tex-text]{DejaVu Sans}
%\setmainfont[Mapping=tex-text]{KerkisSans}
%\setmainfont[Mapping=tex-text]{KerkisCaligraphic}
% \setmainfont[Mapping=tex-text]{Segoe Print}
% \setmainfont[Mapping=tex-text]{Gabriola}

%------------------------------------------------%

% ------------ Mathematics Fonts ----------------%
\setmathfont{Asana-Math.ttf}
% -----------------------------------------------%
% Misc
\author{Κασωτάκη Ε. - Σμαραγδάκης Κ.}
\title{www.math24.gr}
% ---------------- Formation --------------------%


\setlength\topmargin{-0.7cm}
\addtolength\topmargin{-\headheight}
\addtolength\topmargin{-\headsep}
\setlength\textheight{26cm}
\setlength\oddsidemargin{-0.54cm}
\setlength\evensidemargin{-0.54cm}
%\setlength\marginparwidth{1.5in}
\setlength\textwidth{17cm}

\RequirePackage[avantgarde]{quotchap}
\renewcommand\chapterheadstartvskip{\vspace*{0\baselineskip}}
\RequirePackage[calcwidth]{titlesec}
\titleformat{\section}[hang]{\bfseries}
{\Large\thesection}{12pt}{\Large}[{\titlerule[0.9pt]}]
%--------------------------------------------------%
\usepackage{draftwatermark}
\SetWatermarkText{www.math24.gr}
\SetWatermarkLightness{0.9}
\SetWatermarkScale{0.6}
%--------------------------------------------------%


\usepackage{xcolor}
\usepackage{amsthm}
\usepackage{framed}
\usepackage{parskip}

\colorlet{shadecolor}{bluesite!20}

\newtheorem*{orismos}{Ορισμός}

\newenvironment{notation}
  {\begin{shaded}\begin{theorem}}
  {\end{theorem}\end{shaded}}




% Document begins
\begin{document}
\pagestyle{fancy}
\fancyhead{}
\fancyfoot{}
\renewcommand{\headrulewidth}{0pt}
\renewcommand{\footrulewidth}{0pt}


\fancyhead[LO,LE]{
 %\textblockcolor{backsite}
 %\begin{textblock}{5}(-2,-0.55)
  %\rule{0cm}{1cm}
 %\end{textblock}
 \textblockcolor{bluesite}
 \begin{textblock}{5}(-1.5,-0.55)
  \rule{0cm}{1cm}
 \end{textblock}
 %\textblockcolor{bluesite}
 %\begin{textblock}{14}(5.5,-0.55)
 % \rule{0cm}{1cm}
 %\end{textblock}
 \begin{textblock}{0}(-1,-0.25)
 \color{cwhite} \begin{Large}www.math24.gr\end{Large}
 \end{textblock}
\begin{textblock}{0}(16,-0.4)
 \color{cwhite} \includegraphics[height=1.5cm]{math24_logo.png}
 \end{textblock}
\textblockcolor{white}
\begin{textblock}{17}(-1,28)
 \color{backsite} \begin{small}Copyright \textcopyright 2011, Κασωτάκη Ε.(ikasotaki@gmail.com) - Σμαραγδάκης Κ.(kesmarag@gmail.com)\end{small}
 \end{textblock}
}
 
\begin{shaded}
\begin{center}
\huge \textbf{Φυλλάδιο Ασκήσεων}\\
%Πρόσθεση ρητών αριθμών
\end{center} 
\textbf{Μαθηματικά Α' Γυμνασίου} \hfill \textbf{Ημερομηνία Παράδοσης : \hspace{2em} }
\subsection*{Ονοματεπώνυμο :\hfill  \hspace{5em}}
\end{shaded}
\vspace{2em}
\begin{itemize}
 \item Πρόσθεση ομώνυμων κλασμάτων
 \item Πρόσθεση ετερώνυμων κλασμάτων
 \item Αφαίρεση ομώνυμων κλασμάτων
 \item Αφαίρεση ετερώνυμων κλασμάτων
 \item Μεικτός αριθμός
\end{itemize}


\section*{Θεωρία - Πρόσθεση ομώνυμων κλασμάτων\hfill \small{}}
Όταν προσθέτουμε δύο ή περισσότερα ομώνυμα κλάσματα το αποτέλεσμα είναι ένα κλάσμα που για 
αριθμητή έχει το άθροισμα των αριθμητών (των προσθετέων κλασμάτων) και για παρονομαστή έχει 
τον κοινό παρονομαστή τους.\\
\textbf{π.χ} $\dfrac{2}{3}+\dfrac{5}{3}=\dfrac{2+5}{3}=\dfrac{5}{4}$\\
\textbf{π.χ} $\dfrac{6}{20}+\dfrac{11}{20}=\dfrac{6+11}{20}=\dfrac{17}{20}$\\
\textbf{π.χ} $\dfrac{5}{4}+\dfrac{15}{4}=\dfrac{5+15}{4}=\dfrac{20}{4}=5$



\section*{Άσκηση 1  \hfill \small{10 μονάδες}}
Να προσθέσετε τα παρακάτω κλάσματα και να απλοποιήσετε το αποτέλεσμα αν το κλάσμα 
που έχει προκύψει δεν είναι ανάγωγο.
\begin{enumerate}[1)]
\begin{multicols}{2}
 \item $\dfrac{2}{5}+\dfrac{7}{5}$
 \item $\dfrac{2}{13}+\dfrac{4}{13}+\dfrac{20}{13}$
 \item $\dfrac{2}{10}+\dfrac{3}{10}$
 \item $\dfrac{5}{16}+\dfrac{4}{16}+\dfrac{3}{16}$
 \item $\dfrac{2}{6}+\dfrac{4}{6}+\dfrac{7}{6}$
 \item $\dfrac{23}{5}+\dfrac{27}{5}$
 \item $\dfrac{9}{11}+\dfrac{12}{11}$
 \item $\dfrac{7}{40}+\dfrac{13}{40}+\dfrac{5}{40}$
 \item $\dfrac{6}{7}+\dfrac{5}{7}+\dfrac{10}{7}$
 \item $\dfrac{3}{8}+\dfrac{5}{8}+\dfrac{7}{8}+\dfrac{9}{8}$
\end{multicols}
\end{enumerate}


\section*{Θεωρία - Πρόσθεση ετερώνυμων κλασμάτων\hfill \small{}}
Για να προσθέσουμε δύο ή περισσότερα ετερώνυμα κλάσματα πρέπει πρώτα να τα μετατρέψουμε σε ομώνυμα. \\
\textbf{π.χ} $\dfrac{2}{4}+\dfrac{1}{2}=\dfrac{2}{4}+\dfrac{2}{4}=\dfrac{2+2}{4}\dfrac{4}{4}=1$\\
\textbf{π.χ} $\dfrac{5}{3}+\dfrac{1}{4}=\dfrac{20}{12}+\dfrac{3}{12}=\dfrac{23}{12}$



\section*{Άσκηση 2  \hfill \small{10 μονάδες}}
Να προσθέσετε τα παρακάτω κλάσματα και να απλοποιήσετε το αποτέλεσμα αν το κλάσμα 
που έχει προκύψει δεν είναι ανάγωγο.
\begin{enumerate}[1)]
\begin{multicols}{2}
 \item $\dfrac{7}{6}+\dfrac{5}{3}$
 \item $\dfrac{3}{5}+\dfrac{6}{10}$
 \item $\dfrac{7}{4}+\dfrac{3}{5}$
 \item $\dfrac{6}{2}+\dfrac{1}{3}$
 \item $\dfrac{1}{2}+\dfrac{6}{5}$
 \item $\dfrac{1}{2}+\dfrac{3}{2}+\dfrac{5}{4}$
 \item $\dfrac{3}{2}+\dfrac{2}{3}+\dfrac{2}{6}$
 \item $\dfrac{3}{3}+\dfrac{2}{5}+\dfrac{1}{15}$
 \item $\dfrac{3}{2}+\dfrac{1}{3}+\dfrac{5}{4}$
 \item $\dfrac{1}{4}+\dfrac{1}{3}+\dfrac{1}{6}$
\end{multicols}
\end{enumerate}

\section*{Θεωρία - Αφαίρεση ομώνυμων κλασμάτων\hfill \small{}}
Η αφαίρεση δύο ομώνυμων κλασμάτων μας δίνει ένα κλάσμα που έχει τον ίδιο παρονομαστή με τα αρχικά 
κλάσματα και για παρονομαστή έχει τη διαφορά των αριθμητών τους.  \\
\textbf{π.χ} $\dfrac{5}{3}-\dfrac{2}{3}=\dfrac{5-2}{3}=\dfrac{3}{3}=1$\\
\textbf{π.χ} $\dfrac{7}{13}-\dfrac{6}{13}=\dfrac{7-6}{13}=\dfrac{1}{13}$



\section*{Άσκηση 3  \hfill \small{10 μονάδες}}
Να υπολογίσετε τις παρακάτω διαφορές  και να απλοποιήσετε το αποτέλεσμα όπου αυτό δεν είναι  ανάγωγο κλάσμα.
\begin{enumerate}[1)]
\begin{multicols}{2}
 \item $\dfrac{7}{2}-\dfrac{6}{2}$
 \item $\dfrac{8}{3}-\dfrac{5}{3}$
 \item $\dfrac{10}{4}-\dfrac{2}{4}$
 \item $\dfrac{20}{10}-\dfrac{5}{10}$
 \item $\dfrac{18}{6}-\dfrac{15}{6}$
 \item $\dfrac{32}{24}-\dfrac{20}{24}$
 \item $\dfrac{6}{5}-\dfrac{4}{5}$
 \item $\dfrac{9}{7}-\dfrac{1}{7}$
 \item $\dfrac{100}{40}-\dfrac{20}{40}$
 \item $\dfrac{52}{71}-\dfrac{30}{71}$
\end{multicols}
\end{enumerate}


\section*{Θεωρία - Αφαίρεση ετερώνυμων κλασμάτων\hfill \small{}}
Για να αφαιρέσουμε δύο ετερώνυμα κλάσματα πρέπει πρώτα να τα μετατρέψουμε σε ομώνυμα. \\
\textbf{π.χ} $\dfrac{17}{10}-\dfrac{2}{5}=\dfrac{17}{10}-\dfrac{4}{10}=\dfrac{17-4}{10}=\dfrac{3}{10}$\\
\textbf{π.χ} $\dfrac{9}{2}-\dfrac{2}{3}=\dfrac{27}{6}-\dfrac{4}{6}=\dfrac{27-4}{6}=\dfrac{23}{6}$



\section*{Άσκηση 4  \hfill \small{10 μονάδες}}
Να υπολογίσετε τις παρακάτω διαφορές  και να απλοποιήσετε το αποτέλεσμα όπου αυτό δεν είναι  ανάγωγο κλάσμα.
\begin{enumerate}[1)]
\begin{multicols}{2}
 \item $\dfrac{19}{6}-\dfrac{4}{3}$
 \item $\dfrac{30}{6}-\dfrac{5}{2}$
 \item $\dfrac{14}{4}-\dfrac{7}{2}$
 \item $\dfrac{8}{42}-\dfrac{1}{21}$
 \item $\dfrac{3}{4}-\dfrac{1}{5}$
 \item $\dfrac{6}{4}-\dfrac{4}{3}$
 \item $\dfrac{5}{3}-\dfrac{3}{5}$
 \item $\dfrac{2}{6}-\dfrac{3}{12}$
 \item $\dfrac{9}{3}-\dfrac{2}{7}$
 \item $\dfrac{4}{20}-\dfrac{1}{30}$
\end{multicols}
\end{enumerate}

\section*{Θεωρία - Μεικτός αριθμός\hfill \small{}}
\textbf{Μεικτός αριθμός} ονομάζεται το άθροισμα ενός ακέραιου αριθμού με ένα κλάσμα μικρότερο της μονάδας. \\
\textbf{π.χ} ο μεικτός αριθμός $1\dfrac{3}{4}$ σημαίνει $1+\dfrac{3}{4}$\\
\textbf{π.χ} ο μεικτός αριθμός $2\dfrac{6}{11}$ σημαίνει $2+\dfrac{6}{11}$
\begin{itemize}
 \item \textbf{Παράδειγμα} - Μετατροπή ενός κλάσματος μεγαλύτερου της μονάδας σε μεικτό αριθμό\\
       Αν θέλουμε να μετατρέψουμε το κλάσμα $\dfrac{17}{3}$ σε μεικτό αριθμό \\εκτελούμε την 
       ευκλείδεια διαίρεση $17=3\cdot 5+2$ \\και έχουμε: 
       $\dfrac{17}{3}=\dfrac{3\cdot 5}{3}+\dfrac{2}{3}=5+\dfrac{2}{3}=5\dfrac{2}{3}$
 \item \textbf{Παράδειγμα} - Μετατροπή ενός μεικτού αριθμού σε κλάσμα\\
        $5\dfrac{2}{3}=5+\dfrac{2}{3}=\dfrac{15}{3}+\dfrac{2}{3}=\dfrac{17}{3}$
\end{itemize}

\section*{Άσκηση 5  \hfill \small{20 μονάδες}}
Να μετατρέψετε τα παρακάτω κλάσματα σε μεικτούς αριθμούς:
\begin{enumerate}[1)]
\begin{multicols}{2}
 \item $\dfrac{19}{4}$
 \item $\dfrac{25}{3}$
 \item $\dfrac{49}{5}$
 \item $\dfrac{7}{2}$
 \item $\dfrac{14}{3}$
 \item $\dfrac{27}{8}$
 \item $\dfrac{19}{7}$
 \item $\dfrac{33}{6}$
 \item $\dfrac{29}{5}$
 \item $\dfrac{33}{10}$
\end{multicols}
\end{enumerate}

\newpage
\section*{Άσκηση 6  \hfill \small{20 μονάδες}}
Να μετατρέψετε τους παρακάτω μεικτούς αριθμούς σε κλάσματα
\begin{enumerate}[1)]
\begin{multicols}{2}
 \item $3\dfrac{7}{8}$
 \item $4\dfrac{1}{5}$
 \item $6\dfrac{3}{5}$
 \item $1\dfrac{2}{7}$
 \item $2\dfrac{3}{4}$
 \item $3\dfrac{2}{3}$
 \item $1\dfrac{9}{10}$
 \item $1\dfrac{6}{7}$
 \item $1\dfrac{2}{9}$
 \item $2\dfrac{2}{11}$
\end{multicols}
\end{enumerate}

\section*{Άσκηση 7  \hfill \small{20 μονάδες}}
Να υπολογίσετε τις παρακάτω αριθμητικές παραστάσεις.
\begin{enumerate}[1)]
 \item $1+\dfrac{1}{2}+\dfrac{1}{4}-\dfrac{3}{4}$
 \item $5+\dfrac{1}{3}-\dfrac{2}{3}$
 \item $\biggr(\dfrac{7}{2}+\dfrac{4}{3}\biggl)-\biggr(\dfrac{1}{4}+\dfrac{1}{2}\biggl)$
 \item $\biggr(\dfrac{6}{4}-\dfrac{1}{2}\biggl)-\biggr(\dfrac{7}{5}-1\biggl)$
 \item $\biggr(2\dfrac{2}{3}-\dfrac{2}{6}\biggr)-\biggr(\frac{2}{4}-\dfrac{1}{3}\biggl)$
\end{enumerate}


\end{document}
