
\documentclass[a4paper,10pt]{report}

%\usepackage[landscape]{geometry}
\usepackage[cm-default]{fontspec}
\setromanfont{FreeSerif}
\setsansfont{FreeSans}
\setmonofont{FreeMono}
\usepackage[utf8x]{inputenc}
\usepackage{fontspec}
\usepackage{xunicode}
\usepackage{xltxtra}
\usepackage{xgreek}
\usepackage{amsmath}
\usepackage{unicode-math}
\usepackage{ulem}
\usepackage{color}
\usepackage{verbatim}
\usepackage{nopageno}
\usepackage{graphicx}
\usepackage{textpos}
\setlength{\TPHorizModule}{1cm}
\setlength{\TPVertModule}{1cm}
%\usepackage[colorgrid,texcoord]{eso-pic}
\usepackage[outline]{contour}
\usepackage{wrapfig}
\usepackage{url}
\usepackage{color}
\usepackage{tikz}
\usepackage{fancybox,fancyhdr}
\usepackage{subfigure}
\usepackage{pstricks}
\usepackage{epsfig}
\usepackage{multicol}
\usepackage{listings}
\usepackage{enumerate}
\usepackage{hyperref}
\hypersetup{
  bookmarks=true,
  bookmarksopen=true,
  pdfborder=false,
  pdfpagemode=UseNone,
  raiselinks=true,
  pdfhighlight={/P},
  colorlinks,
  citecolor=black,
  filecolor=black,
  linkcolor=black,
  urlcolor=black
}


\usepackage{fontspec}
\usepackage{xunicode}
\usepackage{xltxtra}


% Margins
%\setlength{\textwidth}{24cm}

%\setlength{\voffset}{0in}
%\setlength{\textheight}{6.8in}
% Colors
\definecolor{cbrown}{rgb}{0.49,0.24,0.07}
\definecolor{cmpez}{rgb}{0.92,0.88,0.79}
\definecolor{cmpez2}{rgb}{0.62,0.33,0.34}
\definecolor{cwhite}{rgb}{1,1,1}
\definecolor{cblack}{rgb}{0,0,0}
\definecolor{cred}{rgb}{0.9,0.15,0.15}
\definecolor{clblue}{rgb}{0.57,0.85,0.97}
\definecolor{corange}{rgb}{0.86,0.49,0.18}
\definecolor{clorange}{rgb}{0.95,0.84,0.61}
\definecolor{cyellow}{rgb}{1,0.95,0.16}
\definecolor{cgreen}{rgb}{0.69,0.8,0.31}
\definecolor{clbrown}{rgb}{0.78,0.67,0.43}
\definecolor{cmagenta}{rgb}{0.79,0.05,0.54}
\definecolor{cgray}{rgb}{0.76,0.73,0.63}
\definecolor{bluesite}{rgb}{0.05,0.43,0.67}
\definecolor{redsite}{rgb}{0.90,0.15,0.15}
\definecolor{backsite}{rgb}{0.07,0.07,0.07}
% ----------- from Costas -----------------------------

%---------------   Main Fonts    -----------------%   

% \setmainfont[Mapping=tex-text]{Calibri}
% \setmainfont[Mapping=tex-text]{Times New Roman}
%\setmainfont[Mapping=tex-text]{Droid Serif}
%\setmainfont[Mapping=tex-text]{Cambria}
%\setmainfont[Mapping=tex-text]{cm-unicode}
% \setmainfont[Mapping=tex-text]{Gentium}
% \setmainfont[Mapping=tex-text]{GFS Didot}
% \setmainfont[Mapping=tex-text]{Comic Sans MS}
 \setmainfont[Mapping=tex-text]{Ubuntu}
% \setmainfont[Mapping=tex-text]{Myriad Pro}
% \setmainfont[Mapping=tex-text]{CMU Concrete}
%\setmainfont[Mapping=tex-text]{DejaVu Sans}
%\setmainfont[Mapping=tex-text]{KerkisSans}
%\setmainfont[Mapping=tex-text]{KerkisCaligraphic}
% \setmainfont[Mapping=tex-text]{Segoe Print}
% \setmainfont[Mapping=tex-text]{Gabriola}

%------------------------------------------------%

% ------------ Mathematics Fonts ----------------%
\setmathfont{Asana-Math.ttf}
% -----------------------------------------------%
% Misc
\author{Κασωτάκη Ε. - Σμαραγδάκης Κ.}
\title{www.math24.gr}
% ---------------- Formation --------------------%


\setlength\topmargin{-0.7cm}
\addtolength\topmargin{-\headheight}
\addtolength\topmargin{-\headsep}
\setlength\textheight{26cm}
\setlength\oddsidemargin{-0.54cm}
\setlength\evensidemargin{-0.54cm}
%\setlength\marginparwidth{1.5in}
\setlength\textwidth{17cm}

\RequirePackage[avantgarde]{quotchap}
\renewcommand\chapterheadstartvskip{\vspace*{0\baselineskip}}
\RequirePackage[calcwidth]{titlesec}
\titleformat{\section}[hang]{\bfseries}
{\Large\thesection}{12pt}{\Large}[{\titlerule[0.9pt]}]
%--------------------------------------------------%
\usepackage{draftwatermark}
\SetWatermarkText{www.math24.gr}
\SetWatermarkLightness{0.9}
\SetWatermarkScale{0.6}
%--------------------------------------------------%


\usepackage{xcolor}
\usepackage{amsthm}
\usepackage{framed}
\usepackage{parskip}

\colorlet{shadecolor}{bluesite!20}

\newtheorem*{orismos}{Ορισμός}

\newenvironment{notation}
  {\begin{shaded}\begin{theorem}}
  {\end{theorem}\end{shaded}}




% Document begins
\begin{document}
\pagestyle{fancy}
\fancyhead{}
\fancyfoot{}
\renewcommand{\headrulewidth}{0pt}
\renewcommand{\footrulewidth}{0pt}


\fancyhead[LO,LE]{
 %\textblockcolor{backsite}
 %\begin{textblock}{5}(-2,-0.55)
  %\rule{0cm}{1cm}
 %\end{textblock}
 \textblockcolor{bluesite}
 \begin{textblock}{5}(-1.5,-0.55)
  \rule{0cm}{1cm}
 \end{textblock}
 %\textblockcolor{bluesite}
 %\begin{textblock}{14}(5.5,-0.55)
 % \rule{0cm}{1cm}
 %\end{textblock}
 \begin{textblock}{0}(-1,-0.25)
 \color{cwhite} \begin{Large}www.math24.gr\end{Large}
 \end{textblock}
\begin{textblock}{0}(16,-0.4)
 \color{cwhite} \includegraphics[height=1.5cm]{math24_logo.png}
 \end{textblock}
\textblockcolor{white}
\begin{textblock}{17}(-1,28)
 \color{backsite} \begin{small}Copyright \textcopyright 2011, Κασωτάκη Ε.(ikasotaki@gmail.com) - Σμαραγδάκης Κ.(kesmarag@gmail.com)\end{small}
 \end{textblock}
}
 
\begin{shaded}
\begin{center}
\huge \textbf{Φυλλάδιο Ασκήσεων}\\
%Πρόσθεση ρητών αριθμών
\end{center} 
\textbf{Μαθηματικά Α' Γυμνασίου} \hfill \textbf{Ημερομηνία Παράδοσης : \hspace{2em} }
\subsection*{Ονοματεπώνυμο :\hfill  \hspace{5em}}
\end{shaded}
\vspace{2em}
\begin{itemize}
 \item Πρόσθεση 2 ακέραιων αριθμών
 \item Πρόσθεση περισσότερων από 2 ακέραιων αρνητικών αριθμών
 \item Πρόσθεση ακέραιων αριθμών
\end{itemize}
\section*{Θεωρία - Πρόσθεση 2 ακέραιων αριθμών\hfill \small{}}
\begin{itemize}
 \item \textbf{Ομόσημοι} λέγονται οι αριθμοί που έχουν το ίδιο πρόσημο. 
 \item \underline{ Κανόνας για την πρόσθεση 2 ομόσημων ρητών αριθμών:} \\
       Για να \textbf{προσθέσουμε 2 ομόσημους} ακέραιους αριθμούς, 
       \textbf{προσθέτουμε} τις απόλυτες τιμές τους και στο άθροισμα βάζουμε το πρόσημό τους.\\
       \textbf{π.χ} $(+3)+(+1)=+4$\\
       \textbf{π.χ} $(-5)+(-1)=-6$
 \item \textbf{Ετερόσημοι} λέγονται οι αριθμοί που έχουν διαφορετικό πρόσημο. 
 \item \underline{ Κανόνας για την πρόσθεση 2 ετερόσημων ακέραιων αριθμών:} 
        Για να \textbf{προσθέσουμε 2 ετερόσημους} ακέραιους αριθμούς, 
       \textbf{αφαιρούμε} από τη μεγαλύτερη τη μικρότερη απόλυτη τιμή και στη διαφορά βάζουμε το πρόσημο του 
       ρητού με τη μεγαλύτερη απόλυτη τιμή.\\
       \textbf{π.χ} $(-9)+(+2)=-7$\\
       \textbf{π.χ} $(+3)+(-10)=-7$\\
       \textbf{π.χ} $(+5)+(-1)=+4$\\
       \textbf{π.χ} $(-1)+(+7)=+6$
\end{itemize}


\section*{Άσκηση 1  (Πρόσθεση 2 ακέραιων αριθμών)\hfill \small{30 μονάδες}}
Να υπολογίσετε τα παρακάτω αθροίσματα :
\begin{enumerate}[1)]
\begin{multicols}{3}
 \item $(+3)+(+4)$
 \item $(+13)+(+7)$
 \item $5+1$
 \item $2+7$
 \item $16+20$
 \item $(-5)+(-1)$
 \item $(-3)+(-7)$
 \item $(-4)+(-5)$
 \item $(-10)+(-20)$
 \item $(-1)+(-3)$
 \item $(-9)+(+1)$
 \item $(-2)+(+3)$
 \item $(-7)+(+3)$
 \item $(-23)+(+3)$
 \item $(-27)+(+5)$
 \item $(+7)+(-14)$
 \item $(+1)+(-14)$
 \item $(+3)+(-10)$
 \item $(+11)+(-22)$
 \item $(+12)+(-30)$
 \item $(+5)+(-2)$
 \item $(+10)+(-7)$
 \item $(+21)+(-3)$
 \item $(+27)+(-13)$
 \item $(+10)+(-14)$
 \item $(-1)+(+7)$
 \item $(-3)+(+8)$
 \item $(-13)+(+30)$
 \item $(-2)+(+10)$
 \item $(-3)+(+14)$
\end{multicols}
\end{enumerate}

\section*{Θεωρία - Πρόσθεση περισσότερων από 2 αρνητικών αριθμών\hfill \small{}}
\textbf{\underline{Κανόνας}}: Για να \textbf{προσθέσουμε} περισσότερους από 2 \textbf{αρνητικούς αριθμούς}, 
\textbf{προσθέτουμε} τις απόλυτες τιμές τους και στο άθροισμα βάζουμε το πρόσημό τους.
\begin{itemize}
 \item \textbf{π.χ} $(-3)+(-2)+(-3)=-8$
 \item \textbf{π.χ} $(-4)+(-5)+(-6)+(-3)=-18$
 \item \textbf{π.χ} $(-3)+(-2)+(-4)=-9$
\end{itemize}


\section*{Άσκηση 2 (Πρόσθεση περισσότερων από 2 αρνητικών αριθμών) \hfill \small{30 μονάδες}}
Να υπολογίσετε τα παρακάτω αθροίσματα :
\begin{enumerate}[1)]
 \item $(-3)+(-1)+(-7)$
 \item $(-3)+(-1)+(-8)$
 \item $(-10)+(-20)+(-3)$
 \item $(-13)+(-3)+(-10)$
 \item $(-7)+(-8)+(-1)$
 \item $(-4)+(-7)+(-3)+(-5)$
 \item $(-3)+(-8)+(-7)+(-2)$
 \item $(-1)+(-1)+(-3)+(-3)$
 \item $(-7)+(-8)+(-5)+(-7)+(-3)$
 \item $(-8)+(-3)+(-5)+(-2)+(-4)$
\end{enumerate}


\section*{Θεωρία - Πρόσθεση Ακέραιων Αριθμών\hfill \small{}}
\textbf{\underline{Παράδειγμα}}: θέλουμε να υπολογίσουμε το άθροισμα 
$(+5)+(+8)+(-3)+(-4)+(+10)$
\begin{itemize}
 \item \textbf{1ο Βήμα} Χωρίζουμε τους θετικούς από τους αρνητικούς, 
                        δηλαδή \\$(+5)+(+8)+(+10)+(-3)+(-4)$
 \item \textbf{2ο Βήμα} Προσθέτουμε χωριστά τους θετικούς και χωριστά τους αρνητικούς, δηλαδή παίρνουμε \\
                         $(+23)+(-7)$
 \item \textbf{3ο Βήμα} Τώρα έχουμε να προσθέσουμε έναν θετικό με έναν αρνητικό αριθμό 
                        (δηλαδή έχουμε να προσθέσουμε 2 ετερόσημους αριθμούς), δηλαδή παίρνουμε \\
                         $(+16)$
\end{itemize}


\section*{Άσκηση 3 (Πρόσθεση Ακέραιων Αριθμών) \hfill \small{30 μονάδες}}
Να υπολογίσετε τις τιμές των παρακάτω παραστάσεων:
\begin{enumerate}[1)]
 \item $(+1)+(-7)+(-3)+(+5)$
 \item $(-8)+(-3)+(+4)+(-2)$
 \item $(-16)+(+1)+(-4)$
 \item $(-20)+(+1)+(-3)+(+3)$
 \item $(-30)+(+15)+(-5)+(+7)+(-8)$
 \item $(+50)+(-10)+(-20)+(+23)+(-5)$
 \item $(+1)+(-3)+(+4)+(-5)$
 \item $(+1)+(-1)+(+2)+(-3)$
 \item $(-1)+(-3)+(+4)+(+7)$
 \item $(-12)+(+2)+(-7)+(+1)$
\end{enumerate}



\section*{Άσκηση 4 (Πρόβλημα από την καθημερινή ζωή)\hfill \small{10 μονάδες}}
Η μητέρα της Ρηνούλας της έδωσε 10\texteuro. Η Ρηνούλα πήγε στο βιβλιοπωλείο και πλήρωσε 5\texteuro \quad για να αγοράσει 
τετράδια. Μετά πήγε στο μπακάλικο και έδωσε 2\texteuro \quad για να πάρει ένα παγωτό. 
Λίγο πριν να επιστρέψει στο σπίτι πέρασε να δει τη γιαγιά της και εκείνη της έδωσε χαρτζιλίκι 5\texteuro.\\
 Πόσα χρήματα είχε η Ρηνούλα όταν επέστρεψε στο σπίτι;



\end{document}
